\chapter{Analysis based on Kronecker products}

\section{\textit{Q} matrix representation with Kronecker
  products}

The dynamic behaviour of the superposed generalized stochastic Petri
net $\SGSPN$ can be studied using the continous-time Markov whose
state space is tangible reachability set $\TRS$ of the $\SGSPN$ and
whose infinitesimal generator matrix $Q$ is
\begin{empheq}[left=\empheqlbrace]{gather}
  q_{O}[i, j] = \sumprime_{M_i \xrightarrow{t_k} M'} \lambda_k
  p_{\text{immedidate}}(M', M_j)
  \text, \tag{\ref{eq:intro:qd-element} revisited} \\
  Q = Q_{O} + Q_{D} = Q_{O} - \diag Q_O \vec{e}^T \text.
  \tag{\ref{eq:intro:qd-qo} revisited}
\end{empheq}
Unfortunately, this matrix requires space proportional to the square
of $\TRS$ if dense matrix storage is employed, and reductions os
required storage space by sparse matrix storage methods may not be
sufficient due to the large number of transitions that may be enabled
in a given state.

By exploiting the partitioned structure of superposed \gspn s, the
required space can be reduced. Consider the \emph{restriction} of
$\TRS$, $\TRS\loc{j} \subset \Bag(P\loc{j})$, which is the set of
possible markings the local net $\LN\loc{j}$ can have when the
superposed net $\SGSPN$ is in a tangible reachable state. Moreover,
let $M\loc{j} \in \Bag(P\loc{j})$ denote the restriction of the marking
$M$ to $\LN\loc{j}$. In set-builder notation,
\begin{equation}
  \TRS\loc{j} = \{ M\loc{j} : M \in \TRS \} \text.
\end{equation}

\begin{obs}\label{obs:kronecker:pimmediate-j}
  Let $p\loc{j}_{\text{immediate}}(M\loc{j}, M^{\prime (j)})$ denote
  the probability of state change from $M\loc{j} \in \Bag(P\loc{j})$ to
  $M^{\prime (j)} \in \TRS\loc{j}$ due to immediate transitions in
  $\LN\loc{j}$. Every immediate transition is local in an \sgspn and
  can cause state changes only in its local net, independently of
  other local nets. Therefore, we may write the global immediate
  state-change probabilities as
  \begin{equation}
    p_{\text{immediate}}(M, M') = \prod_{j = 1}^J
    p\loc{j}_{\text{immediate}}(M\loc{j}, M^{\prime (j)})
    \text. \label{eq:kronecker:pimmediate-j} 
  \end{equation}
\end{obs}

\subsection{Tangible reachability sets with Descartes product form}

Suppose that the tangible reachability set may be written as
\begin{equation}
  \TRS = \TRS\loc{1} \times \TRS\loc{2} \times \cdots \times
  \TRS\loc{j} \text,
\end{equation}
in other words, every combination of local net markings is tangible
and reachable. Then the off-diagonal part of $Q$ can be written as
\begin{equation}
  Q_O = \bigoplus_{j = 1}^J Q_L\loc{j} + \sum_{t \in \TS} \bigotimes_{j
  = 1}^J Q_t\loc{j} \text, \label{eq:kronecker:simple-local-global}
\end{equation}
where $Q_L\loc{j}$ describes the effect of \emph{local} transitions on
$\LN\loc{j}$, while $Q_t\loc{j}$ corresponds to a synchronized
transition $t \in \TS = \bigcup_{j = 1}^J \TS\loc{j}$ in
$\LN\loc{j}$. The matrices $Q_L\loc{j}$ and $Q_t\loc{j}$ are of size
$n_j \times n_j = \lvert \TRS\loc{j} \rvert \times \lvert \TRS\loc{j}
\rvert$ and together represent the $n \times n = \lvert \TRS \rvert
\times \lvert \TRS \rvert$ matrix $Q_O$. We have used \emph{Kronecker
  sum} notation to write the contribution of local transitions, i.e.
\begin{equation}
  \begin{aligned}
    \bigoplus_{j = 1}^J Q_L\loc{j} &= \sum_{j = 1}^J I^{n_1 \times
      n_1} \otimes I^{n_2 \times n_2} \otimes \cdots \otimes I^{n_{j -
        1} \times n_{j - 1}} \otimes Q_L\loc{j} \otimes I^{n_{j + 1}
      \times
      n_{j + 1}} \otimes \cdots \otimes I^{n_J \times n_J} \\
    &= \sum_{j = 1}^J I^{\prod_{k = 1}^{j - 1} n_k \times \prod_{k =
        1}^{j - 1}} \otimes Q_L\loc{j} \otimes I^{\prod_{k = j +
        1}^{J} n_k \times \prod_{k = j + 1}^{J}} \text.
  \end{aligned}
\end{equation}

Terms describing local transitions are the Kronecker products of all
but one identity matrices, because local transitions cause no state
change in local nets other than the one they belong to. That is,
\begin{equation}
  q_L\loc{j}[x, y] = \sumprime_{M_x \xrightarrow{t_k}
      M'} \lambda_k p_{\text{immediate}}\loc{j}(M', M_y) \text{, where
      $t_k \in \LN\loc{j},$} \label{eq:kronecker:ql}
\end{equation}
$M_x, M_y \in \TRS\loc{j}$ and the summation is done over all
enabled timed transitions local to $\LN_j$. In $\LN_j$ contains no
immediate transitions, \cref{eq:kronecker:ql} can be simplified to
\begin{equation}
  q_L\loc{j}[x, y] = \sumprime_{M_x \xrightarrow{t_k} M_y} \lambda_k
  \text{, where $t_k \in \LN\loc{j}$.}
\end{equation}

If the transition rates, are marking dependent, that is, $\Lambda: T
\times \Bag(P) \to \mathbb{R}^{+}$, then we can write
\begin{equation}
  q_L\loc{j}[x, y] = \sumprime_{M_x \xrightarrow{t_k}
    M'} \lambda_k(M_x) p_{\text{immediate}}\loc{j}(M', M_y) \text{, where
    $t_k \in \LN\loc{j},$}
\end{equation}
as long as the rate $\lambda_k$ only depends the marking of places in
$P\loc{j}$. Thus, the Kronecker product decomposition framework can be
also employed if (limited) marking dependency is used in the model.

The synchronized transition $\TS$ can be treated as follows. First,
let us write the transition rate of $t_k \in \TS$ as
\begin{equation}
  \lambda_k = \lambda\loc{j_1}_k \lambda\loc{j_2}_k \cdots
  \lambda\loc{j_\ell}_k \label{eq:kronecker:lambdak}
\end{equation}
where $1 \le j_1 < j_2 < \cdots < j_\ell \le J$ are the indices of all
the local transitions sets $P\loc{j_i}$ in which $t_k$ appears. That
is, there is $\lambda\loc{j_i}_k$ corresponding to each local net the
transition $t_k$ synchronizes. If there is no marking dependency in
the model, we can set $\lambda\loc{j_1}_k = \lambda_k$ while all the
others are $1$. If marking dependency is used, each
$\lambda\loc{j_i}_k$ is allowed to depend on the restriction of the
marking to $P\loc{j_i}$, so
\begin{equation}
  \lambda_k(M) = \lambda\loc{j_1}_k(M\loc{j_1})\;
  \lambda\loc{j_2}_k(M\loc{j_2}) \;\cdots\;
  \lambda\loc{j_\ell}_k(M\loc{j_\ell}) \text.
\end{equation}
Now the matrices corresponding to synchronized transitions can be
expressed as
\begin{equation}
  \begin{cases}
    q_{t_k}\loc{j}[x, y] = \lambda_k\loc{j} p_{\text{immediate}}\loc{j}(M', M_y)
    & \text{if $t_k \in T\loc{j}$ and $\exists M' : M_x \xrightarrow{t_k} M'$,} \\
    q_{t_k}\loc{j}[x, y] = 0
    & \text{if $t_k \in T\loc{j}$ and $M_x \not\xrightarrow{t_k}$,} \\
    Q_{t_k}\loc{j} = I^{n_j \times n_j} & \text{if $t_k \notin T\loc{j}$.}
  \end{cases}\label{eq:kronecker:qt}
\end{equation}

Recalling multi-index notation from \vref{sec:intro:tensors}, we may
verify that
\begin{multline}
  q_O[\vec{x}, \vec{y}] = q_O[x\loc{1}, x\loc{2}, \ldots, x\loc{J},
  y\loc{1}, y\loc{2}, \ldots, y\loc{J}] = \\ \sum_{j = 1}^J
  q_L\loc{j}[x\loc{j}, y\loc{j}] + \sum_{t \in \TS} \prod_{j = 1}^J
q_t\loc{j}[x\loc{j}, y\loc{j}] \text. \label{eq:kronecker:qd-multiindex}
\end{multline}

\subsubsection{Contribution from local transitions}

The first term can be treated easily,
\begin{align}
   \sum_{j = 1}^J q_L\loc{j}[x\loc{j}, y\loc{j}] &= \sum_{j = 1}^J\,
  \sumprime_{\substack{M_x \xrightarrow{t_k} M' \\ t_k \in
  \TL\loc{j}}} \lambda_k p_{\text{immediate}}\loc{j}(M^{\prime(j)},
  M_y\loc{j}) \label{eq:kronecker:tl-doublesum}\\
  &= \sumprime_{\substack{M_x \xrightarrow{t_k} M' \\ t_k \in \TL}}
  \lambda_k p_{\text{immediate}}(M', M_y)
  \text, \label{eq:kronecker:tl-singlesum}
\end{align}
where $\TL = \bigcup_{j = 1}^J \TL\loc{j}$. Each local transition
appears exactly once in the double summation in
\cref{eq:kronecker:tl-doublesum}, because it belongs to exactly one
local net, so the simplification \labelcref{eq:kronecker:tl-singlesum}
is possble. A local transition cannot cause immediate state change in
any local net other than its own, since timed transitions cannot be
synchronized, therefore
$p_{\text{immediate}}\loc{j}(M^{\prime(j)}, M_y\loc{j}) =
p_{\text{immediate}}(M', M_y)$ holds for local transitions.

\subsubsection{Contribution from synchronized transitions}

One may notice that the factors
\begin{equation}
  \prod_{j = 1}^J q_t\loc{j}[x\loc{j}, y\loc{j}]
\end{equation}
on the right side of \vref{eq:kronecker:qd-multiindex} are nonzero
only if $t$ is enabled in $M_x$. Additionally, if $M_x$ is a tangible
marking and $M_x \xrightarrow{t_k} M'$, $M^{\prime (j)}$ must be
tangible for all $j$ for which $t_k \notin T\loc{j}$, because the
firing of $t_k$ causes no marking change in such local nets. This means that
$M_x\loc{j} \xrightarrow{t_k} M^{\prime (j)} = M_x\loc{j}$ for
$t_k \notin T\loc{j}$ and
$p_{\text{immediate}}\loc{j}(M^{\prime (j)}, M_y\loc{j}) = 1$ if
$M_x\loc{j} = M^{\prime (j)} = M_y\loc{j}$, otherwise $0$.

Thus, we can rewrite the second term as
\begin{align}
  &\sum_{t_k \in \TS} \prod_{j = 1}^J q_{t_k}\loc{j}[x\loc{j}, y\loc{j}] =
    \sumprime_{\substack{M_x \xrightarrow{t_k} M' \\ t_k \in \TS}} \,
  \prod_{j = 1}^J q_t\loc{j}[x\loc{j}, y\loc{j}] \\
  &\qquad=\sumprime_{\substack{M_x \xrightarrow{t_k} M' \\ t_k \in
  \TS}} \, \prod_{t_k \in T\loc{j}} \lambda_k\loc{j}
  p_{\text{immediate}}\loc{j}(M^{\prime (j)}, M_y\loc{j}) \, \prod_{t_k
  \notin T\loc{j}} \delta_{x\loc{j}, y\loc{j}} \\
  &\qquad=\sumprime_{\substack{M_x \xrightarrow{t_k} M' \\ t_k \in
  \TS}} \, \prod_{t_k \in T\loc{j}} \lambda_k\loc{j}
  p_{\text{immediate}}\loc{j}(M^{\prime (j)}, M_y\loc{j}) \, \prod_{t_k
  \notin T\loc{j}} p_{\text{immediate}}\loc{j}(M^{\prime (j)},
  M_y\loc{j}) \text, \\
  \intertext{where we can use the factorization of $\lambda_k$
  from \vref{eq:kronecker:lambdak}
  and \cref{eq:kronecker:pimmediate-j}
  from \vref{obs:kronecker:pimmediate-j} to conclude}
  &\qquad=\sumprime_{\substack{M_x \xrightarrow{t_k} M' \\ t_k \in
  \TS}} \, \prod_{t_k \in T\loc{j}} \lambda_k\loc{j} \,\, \prod_{j = 1}^J
  p_{\text{immediate}}\loc{j} (M^{\prime (j)}, M_y\loc{j}) \\
  &\qquad= \sumprime_{\substack{M_x \xrightarrow{t_k} M' \\ t_k \in
  \TS}} \lambda_k p_{\text{immediate}}(M', M_y)
  \text, \label{eq:kronecker:qo-ts-result}
\end{align}
where $\delta_{x,y}$ is the $[x, y]$\thinspace th element of the
identity matrix.

\begin{thm}
  The decomposition \labelvref{eq:kronecker:simple-local-global},
  where the matrices $Q_L\loc{j}$ and $Q_t\loc{j}$ are defined by
  \cref{eq:kronecker:ql,eq:kronecker:qt}, is equivalent to the
  explicit generator matrix from
  \fullref{eq:intro:qd-element,eq:intro:qd-qo}.
\end{thm}

\begin{proof}
  Because $T = \TL \uplus \TS$ is a disjoint partition of the
  transitions, the addition of \vref{eq:kronecker:tl-singlesum} and
  \vref{eq:kronecker:qo-ts-result} yields
  \begin{equation}
    \sumprime_{\substack{M_x \xrightarrow{t_k} M' \\ t_k \in
        \TL \uplus \TS}} \lambda_k p_{\text{immediate}}(M', M_y) =
    \sumprime_{M_x \xrightarrow{t_k} M'} \lambda_k
    p_{\text{immediate}}(M', M_y) \text.
  \end{equation}
  This is a summation over all enabled timed transitions, as required
  by \cref{eq:intro:qd-element}.
\end{proof}

\fbreak

\begin{dfn}
  The \emph{potential state space} of an \sgspn is the Descartes
  product of the tangible reachability sets of its local nets,
  \begin{equation}
    \PSS = \TRS\loc{1} \times \TRS\loc{2} \times \cdots \times
    \TRS\loc{J} \subseteq \TRS \text.
  \end{equation}
\end{dfn}

Now we turn to the case when $\PSS \ne \TRS$, that is, when the
tangible reachability space of the \sgspn\ is \emph{not} in Descartes
product form. Approaches developed for this kind of systems may be
divided into the following three broad categories
\citet{DBLP:conf/ipps/BenoitPS03}:
\begin{enumerate}
\item Local nets may be aggregated (joined into larger local nets)
  until the resulting \sgspn's $\TRS$ becomes a Descartes
  product. Unfortunately, is is possible that the resulting aggregated
  \sgspn\ has only a single local net, therefore the approach gives no
  benefit over explicit treatment of the \gspn of the resulting local
  nets have state spaces $\TRS\loc{j}$ so large that computation
  become infeasible.
\item The infinitesimal generator 
\end{enumerate}

\section{Efficient vector--Kronecker product multiplication}

Both steady-state and transient analysis of \ctmc s, such as
\vref{eq:intro:diffeq-initial,eq:intro:steadystate}, often require the
multiplication of a probability distribution vector $\vec{\uppi}$ with
the infinitesimal generator matrix $Q$.

Because the vector--matrix product is associative,
\begin{align}
  \vec{\pi} Q &= \vec{\uppi} \, \mleft( \bigoplus_{j = 1}^J Q_L\loc{j}
                + \sum_{t \in \TS} \bigotimes_{j = 1}^J Q_t\loc{j}
                \mright) \\
              &= \sum_{j = 1}^J \vec{\uppi} \, I^{n_1 \times n_1} \otimes
                \cdots \otimes Q_L\loc{j} \otimes \cdots \otimes
                I^{n_J \times n_J} + \sum_{t \in \TS} \vec{\uppi}
                \bigotimes_{j = 1}^J Q_t\loc{j} \text,
\end{align}
thus only vector--Kronecker product multiplications and vector
additions are requred for probability vector--$Q$ matrix
multiplications.

An efficient vector--Kronecker product multiplication algorithm must
evaluate the product
\begin{equation}
  \vec{v} \bigotimes_{j = 1}^J A\loc{j}
  \text, \label{eq:kronecker:vector-kronecker-product}
\end{equation}
where each $A\loc{j}$ is of size $m_j \times n_j$, without explicitly
evaluating and storing the
$m \times n = \prod_{j = 1}^J m_j \times \prod_{j = 1}^J n_j$ matrix
$A = \bigotimes_{j = 1}^J A\loc{j}$. The simplest of these algorithm
is the \emph{extended shuffle algorithm} by
\citet{DBLP:conf/ipps/BenoitPS03}. The algorithm is an extension of
another, rather, \emph{extended} refers to the fact that the vector
$\vec{v}$ is stored in an \emph{extended} or \emph{dense} form by
explicitly storing all the zero elements in addition to the
nonzeroes. In contrast, the \emph{partially reduced} and \emph{fully
  reduced shuffle algoritms}
\citep{DBLP:conf/ipps/BenoitPS03,DBLP:journals/fgcs/BenoitPS06} can
handle multiplication of vectors is sparse format and thus may be
employed in potential state-space Kronecker methods even when the size
of $\PSS$ prevents explicit dense storage.

The shuffle algoritms are based on a rewriting of
\cref{eq:kronecker:vector-kronecker-product}
\begin{align}
  \vec{v} \bigotimes_{j = 1}^J A\loc{j}
  &= \vec{v} \prod_{j = 1}^J
    I^{n_1 \times n_1} \otimes \cdots \otimes
    I^{n_{j - 1} \times n_{j - 1}} \otimes A\loc{j} \otimes I^{m_{j + 1}
    \times m_{j + 1}} \otimes \cdots \otimes I^{m_J \otimes m_J} \\
  &= \vec{v} \prod_{j = 1}^J I^{\textit{nleft}_j \times
    \textit{nleft}_j} \otimes A\loc{j} \otimes I^{\textit{mright}_j
    \times \textit{mright}_j} \text,
\end{align}
where
\begin{equation}
  \begin{aligned}
    \textit{nleft}_j &= \textstyle \prod_{k = 1}^{j - 1} n_k \text,
    & \textit{nright}_j &= \textstyle \prod_{k = j + 1}^{J} m_k \text. \\
  \end{aligned}
\end{equation}

\Vref{alg:kronecker:shuffle} shows the pseudocode of the extended
shuffle algorithm.

\begin{algorithm}
  \caption{Extended shuffle
    \citep[Algorithm~2.1]{DBLP:conf/ipps/BenoitPS03}}
  \label{alg:kronecker:shuffle}
  \begin{algorithmic}[1]
    \Function{Shuffle}{$\vec{v}, A\loc{1}, A\loc{2}, \ldots,
      A\loc{J}$}
    \State Allocate buffers $\vec{x}$ and $\vec{y}$ of length $\max_{0
    \le j \le J} \prod_{k = 1}^j n_k \prod_{k = j}^J m_k$ each
    \State Copy $\vec{v}$ to the first $m$ elements of $\vec{x}$
    \State $\textit{nleft} = 1$, $\textit{mright} = \prod_{j = 2}^J m_j$
    \For{$j \gets 1$ \To $J$}
    \Comment{For each Kronecker factor}
    \For{$l \gets 0$ \To $\textit{nleft} - 1$}
    \Comment{For each ``slice'' of $\vec{x}$}
    \For{$r \gets 0$ \To $\textit{mright} - 1$}
    \Comment{For each ``subslice''}
    \State Allocate a vector $\vec{z}_{\textrm{in}}$ of length $m_j$
    \State $\textit{index} \gets l \cdot \textit{mright} \cdot m_j + r$
    \Comment{Find the input subslice}
    \For{$k \gets 1$ \To $m_j - 1$}
    \Comment{Copy elements to be multiplied to $\vec{z}_{\textrm{in}}$}
    \State $z_{\textrm{in}}[k] \gets x[\textit{index}]$
    \State $\textit{index} \gets \textit{index} + \textit{mright}$
    \EndFor
    \State $\vec{z}_{\textrm{out}} \gets \vec{z}_{\textrm{in}}
    A\loc{j}$
    \Comment{Perform the multiplication}
    \State $\textit{index} \gets l \cdot \textit{mright} \cdot n_j +
    r$
    \Comment{Find the output subslice}
    \For{$k \gets 1$ \To $n_j - 1$}
    \Comment{Copy results from $\vec{z}_{\textrm{out}}$ to $\vec{y}$}
    \State $y[\textit{index}] \gets z_{\textrm{out}}[k]$
    \State $\textit{index} \gets \textit{index} + \textit{mright}$
    \EndFor
    \EndFor
    \EndFor
    \State Swap $\vec{x}$ and $\vec{y}$
    \State $\textit{nleft} \gets \textit{nleft} \cdot n_j$
    \Comment{Update $\textit{nleft}$ and $\textit{mright}$}
    \State \algorithmicif\ $j < J$ \algorithmicthen\
    $\textit{mright} \gets \textit{mright} / m_{j + 1}$
    \EndFor
    \State Copy the first $n$ elements of $\vec{x}$ into a new vector
    $\vec{w}$
    \State \Return $\vec{w} = \bigotimes_{j = 1}^J A\loc{j}$
    \EndFunction
  \end{algorithmic}
\end{algorithm}

%%% Local Variables:
%%% mode: latex
%%% TeX-master: "markov"
%%% End:
