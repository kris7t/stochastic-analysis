\begin{otherlanguage}{magyar}

  \paragraph*{Összefoglaló}
  \phantomsection
  \addcontentsline{toc}{chapter}{Összefoglaló}
  \thispagestyle{plain}

  A kritikus rendszerek --~biztonságkritikus,
  elosztott és felhőalkalmazások~-- helyességének
  biztosításához szükséges a funkcionális és nemfunkcionális
  követelmények matematikai igényességű ellenőrzése. Számos,
  szolgáltatásbiztonsággal és teljesítményvizsgálattal kapcsolatos
  tipikus kérdés általában sztochasztikus analízis segítségével
  válaszolható meg.

  A kritikus rendszerek elosztott és aszinkron tulajdonságai az
  \emph{állapottér robbanás} jelenségéhez vezetnek. Emiatt méretük és
  komplexitásuk gyakran megakadályozza a sikeres sztochasztikus
  analízist, melynek számításigénye nagyban függ a lehetséges
  viselkedések számától. A modellek komponenseinek jellegzetes időbeli
  viselkedése a számításigény további jelentős növekedését okozhatja.

  A szolgáltatásbiztonsági és teljesítményjellemzők kiszámítása
  markovi modellek állandósult állapotbeli és tranziens megoldását
  igényli. Számos eljárás ismert ezen problémák kezelésére, melyek
  eltérő reprezentációkat és numerikus algoritmusokat alkalmaznak; ám
  a modellek változatos tulajdonságai miatt nem választható ki olyan
  eljárás, mely minden esetben hatékony lenne.

  A markovi analízishez szükséges a modell lehetséges viselkedéseinek,
  azaz állapotterének felderítése, illetve tárolása, mely szimbolikus
  módszerekkel hatékonyan végezhető el. Ezzel szemben a sztochasztikus
  algoritmusokban használt vektor- és indexműveletek szimbolikus
  megvalósítása nehézkes. Munkánk célja egy olyan, integrált
  keretrendszer fejlesztése, mely lehetővé teszi a komplex
  sztochasztikus rendszerek kezelését a szimbolikus módszerek,
  hatékony mátrix reprezentációk és numerikus algoritmusok előnyeinek
  ötvözésével.

  Egy teljesen szimbolikus algoritmust javasolunk a sztochasztikus
  viselkedéseket leíró mátrix-dekompozíciók előállítására a
  szimbolikus formában adott állapottérből kiindulva. Ez az eljárás
  lehetővé teszi a temporális logikai kifejezéseken alapuló
  szimbolikus technikák használatát.

  A keretrendszerben megvalósítottuk a konfigurálható sztochasztikus
  analízist: megközelítésünk lehetővé teszi a különböző
  mátrix reprezentációk és numerikus algoritmusok kombinált
  használatát. Az implementált algoritmusokkal állandósult állapotbeli
  költség- és érzékenység analízis, tranziens költséganalízis és első
  hiba várható bekövetkezési idő analízis végezhető el sztochasztikus
  Petri-háló~\paren{\textls{SPN}} alapú markovi költségmodelleken. Az
  elkészített eszközt integráltuk a \textsc{PetriDotNet} modellező
  szoftverrel. Módszerünk gyakorlati alkalmazhatóságát szintetikus és
  ipari modelleken végzett mérésekkel igazoljuk.

\end{otherlanguage}

\cleardoublepage

\paragraph*{Abstract}
\phantomsection
\addcontentsline{toc}{chapter}{Abstract}
\thispagestyle{plain}

Ensuring the correctness of critical systems --~such as
safety-critical, distributed and cloud applications~-- requires the
rigorous analysis of the functional and extra-functional properties of
the system. A large class of typical quantitative questions regarding
dependability and performability are usually addressed by stochastic
analysis.

Recent critical systems are often distributed/asynchronous, leading to
the well-known phenomenon of \emph{state space explosion}. The size
and complexity of such systems often prevents the success of the
analysis due to the high sensitivity to the number of possible
behaviors. In addition, temporal characteristics of the components can
easily lead to huge computational overhead.

Calculation of dependability and performability measures can be
reduced to steady-state and transient solutions of Markovian
models. Various approaches are known in the literature for these
problems differing in the representation of the stochastic behavior of
the models or in the applied numerical algorithms. The efficiency of
these approaches are influenced by various characteristics of the
models, therefore no single best approach is known.

The prerequisite of Markovian analysis is the exploration of the state
space, i.e.~the possible behaviors of the system. Symbolic approaches
provide an efficient state space exploration and storage technique,
however their application to support the vector operations and index
manipulations extensively used by stochastic algorithms is cumbersome.
The goal of our work is to introduce a framework that facilitates the
analysis of complex, stochastic systems by combining the
advantages of symbolic algorithms, compact matrix representations and
various numerical algorithms.

We propose a fully symbolic method to explore and describe the
stochastic behaviors. A new algorithm is introduced to transform the
symbolic state space representation into a decomposed linear algebraic
representation. This approach allows leveraging existing symbolic
techniques, such as the specification of properties with
\emph{Computational Tree Logic}~\paren{\textls{CTL}} expressions.

The framework provides configurable stochastic analysis: an approach
is introduced to combine the different matrix representations with
numerical solution algorithms. Various algorithms are implemented for
steady-state reward and sensitivity analysis, transient reward
analysis and mean-time-to-first-failure analysis of stochastic models
in the \emph{Stochastic Petri Net}~\paren{\textls{SPN}} based Markov reward
model formalism. The analysis tool is integrated into the \textsc{PetriDotNet}
modeling application.  Benchmarks and industrial case studies are used
to evaluate the applicability of our approach.
