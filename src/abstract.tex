\begin{otherlanguage}{magyar}

  \chapter{Összefoglaló}

  A technológia fejlődésével a számítógépes rendszerek alkalmazási
  köre ma már olyan biztonságkritikus rendszerekre is kiterjed,
  amelyek helyes működésétől sokszor teljes vállalatok sorsa, vagy
  akár emberéletek is függhetnek. Az ilyen rendszerek méretének és
  bonyolultságának növekedtével szükségessé vált a megbízható
  automatikus módszerek kifejlesztése a rendszer (biztonság
  szempontjából) kritikus tulajdonságai teljesülésének ellenőrzésére,
  illetve a megbízhatósági és teljesítményjellemzők kiszámítására. Míg
  a formális módszerekkel matematikai igényességgel igazolható a
  tervezési folyamat helyessége, a sztochastikus technikák lehetővé
  teszik a modell kibővítését dinamikus, kvantiatív jellemzőkkel.

  A modellek métetének és bonyolultságának növekedése nagyságrendekkel
  emelheti meg a modell lehetséges futásidejű konfigurációinak
  számát. Ez a jelenség az \emph{állapottér-robbanás}, mely mind a
  modellelenőrzés alapú formális verifikáció, mind a sztochasztikus
  analízís során nehézséget jelent. A modellellenőrzésnél, mely a
  formalizált követelmények verifikálásához felderíti a modell
  állapothalmazát, a \emph{szaturációs} használó szimbolikus technikák
  segítik a rendkívül nagy méretű állapotterek hatékony kezelését. A
  sztochastikus analízis erre a célra általában lineáris algebrai
  mátrix-dekompozíciókat alkalmaz.

\end{otherlanguage}

\chapter{Abstract}

Ensuring correctness of critical systems, like safety-critical, distributed and cloud applications necessitates the analysis of functional and also qualitative aspects. Typical qualitative questions regarding dependability and performability are answered by stochastic analysis. However, the complexity of such systems raised the need for rigorous analysis methods. 
Stochastic analysis methods are highly sensitive to the number of possible behaviours of the systems. Recent critical systems are used to be distributed/asynchronous leading to the well-known phenomenon of state space explosion often preventing the success of stochastic analysis. In addition, temporal characteristics of the components can easily lead to huge 
computational overhead. 
Calculation of dependability and performability measures can be reduced to steady-state and transient solutions of Markovian models. Various approaches are known in the literature for these problems differing in the representation of the stochastic behaviour of the model and also differing in the applied numerical algorithms. Various characteristics of the models influence the overheads associated with these approaches, therefore no single best approach is known.
 
Various solutions were developed in the history for computing the 

The operation of entire corporations or even human life may depend on
the correctness of safety-critical systems which became a prominent
application area of computer systems due to the advancements in
technology. The size and complexity of these systems is increasing,
which creates a need for the development of trustable, automatic methods
for verification of critical system properties and estimation of
dependability and performability measures. Formal methods guarantee
the correctness of the design process with mathematical rigor, while
stochastic modeling allows extending models with quantitative
properties.

The increase of model size and complexity can cause orders of
magnitudes growth in the space of modeled possible runtime
configurations. This phenomenon, called \emph{state space explosion},
poses a difficulty for both formal verification based on model
checking and for quantitative performance analysis. In model checking,
which enumerates the possible states of the model to verify formalized
system properties, symbolic approaches based on the \emph{saturation}
algorithm can overcome even exceptionally large state spaces with
efficiency. In stochastic analysis decompositions based on linear
algebra are used.

We propose a fully symbolic method of creating a description of the
stochastic behavior of the system to be used in analysis calculations
in decomposed form. This approach allows leveraging existing symbolic
techniques of model checking for complex systems, such as the
specification of the analyzed model properties with
\emph{continous-time temporal logic} \paren{\textls{CTL}} expressions.

The decomposition method is implemented for steady-state reward and
sensitivity analysis, transient reward analysis and
mean-time-to-first-failure analysis of stochastic models in the
\emph{stochastic Petri net} \paren{\textls{SPN}} Markov reward model
formalism. The analysis tool is integrated into the Petridotnet~1.3%
\footnote{\url{https://inf.mit.bme.hu/research/tools/petridotnet}}
modeling and analysis applications along with a flexible linear
algebra optimized for decompositions and calculations in used
stochastic analysis which allows the user to fine-tune the analysis
for the model under study.

The performance of analysis algorithms and decompositions is studied
for multiple benchmark models and case studies and compared to other
available analysis tools.
