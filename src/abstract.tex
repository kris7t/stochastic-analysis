\begin{otherlanguage}{magyar}

  \addcontentsline{toc}{chapter}{Összefoglaló}
  \thispagestyle{plain}
  \paragraph*{Összefoglaló}

  A technológia fejlődésével a számítógépes rendszerek
  alkalmazási köre ma már olyan biztonságkritikus rendszerekre is
  kiterjed, amelyek helyes működésétől sokszor teljes vállalatok
  sorsa, vagy akár emberéletek is függhetnek. Az ilyen rendszerek
  méretének és bonyolultságának növekedtével szükségessé vált a
  megbízható automatikus módszerek kifejlesztése a rendszer (biztonság
  szempontjából) kritikus tulajdonságai teljesülésének ellenőrzésére,
  illetve a megbízhatósági és teljesítményjellemzők kiszámítására. Míg
  a formális módszerekkel matematikai igényességgel igazolható a
  tervezési folyamat helyessége, a sztochastikus technikák lehetővé
  teszik a modell kibővítését dinamikus, kvantiatív jellemzőkkel.

  A modellek métetének és bonyolultságának növekedése nagyságrendekkel
  emelheti meg a modell lehetséges futásidejű konfigurációinak
  számát. Ez a jelenség az \emph{állapottér-robbanás}, mely mind a
  modellelenőrzés alapú formális verifikáció, mind a sztochasztikus
  analízís során nehézséget jelent. A modellellenőrzésnél, mely a
  formalizált követelmények verifikálásához felderíti a modell
  állapothalmazát, a \emph{szaturációs} használó szimbolikus technikák
  segítik a rendkívül nagy méretű állapotterek hatékony kezelését. A
  sztochastikus analízis erre a célra általában lineáris algebrai
  mátrix-dekompozíciókat alkalmaz.

\end{otherlanguage}

\cleardoublepage

\addcontentsline{toc}{chapter}{Abstract}
\thispagestyle{plain}
\paragraph*{Abstract}

Ensuring the correctness of critical systems---such as
safety-critical, distributed and cloud applications---requires the
rigorous analysis of the functional and extra-functional properties of
the system. A large class of typical quantitative questions regarding
dependability and performability are usually addressed by stochastic
analysis.

However, the size and complexity of such systems often prevents the
success of the analysis as they are highly sensitive to the number of
possible behaviors. Recent critical systems are used to be
distributed/asynchronous leading to the well-known phenomenon of state
space explosion. In addition, temporal characteristics of the
components can easily lead to huge computational overhead.

Calculation of dependability and performability measures can be
reduced to steady-state and transient solutions of Markovian
models. Various approaches are known in the literature for these
problems differing in the representation of the stochastic behavior
of the models and also differing in the applied numerical
algorithms. Various characteristics of the models influence overheads
associated with these approaches, therefore no single best approach is
known.

The prerequisite of Markovian analysis is the exploration of the set
of reachable states, i.e.~the behaviors of the system. Symbolic
approaches provide an efficient state space exploration and storage
technique, however their application to support the vector operations
and index manipulations extensively used by stochastic algorithms is
cumbersome.  The goal of our work is to introduce a framework that
facilitates the analysis of complex, stochastic systems by combining
together the advantages of symbolic algorithms, compact matrix
representations and various numerical algorithms.

We propose a fully symbolic method to describe the stochastic
behaviors. A new algorithm is introduced to transform the symbolic
state space representation into a decomposed linear algebraic
representation. This approach allows leveraging existing symbolic
techniques, such as the specification of properties with
\emph{Computational Tree Logic}~\paren{\textls{CTL}} expressions.

The framework provides configurable stochastic analysis: an approach
is introduced to combine the various matrix representations with the
numerical solution algorithms. Various algorithms are implemented for
steady-state reward and sensitivity analysis, transient reward
analysis and mean-time-to-first-failure analysis of stochastic models
in the \emph{Stochastic Petri Net} \paren{\textls{SPN}} Markov reward
model formalism. The analysis tool is integrated into the PetriDotNet
modeling application.  Benchmarks and industrial case studies are used
to evaluate the applicability of our approach.
