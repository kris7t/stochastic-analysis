\begin{otherlanguage}{magyar}

  \paragraph*{Kivonat}
  \phantomsection
  \addcontentsline{toc}{chapter}{Kivonat}
  \thispagestyle{plain}

  A kritikus rendszerek --~biztonságkritikus,
  elosztott és felhőalkalmazások~-- helyességének
  biztosításához szükséges a funkcionális és nemfunkcionális
  követelmények matematikai igényességű ellenőrzése. Számos,
  szolgáltatásbiztonsággal és teljesítményvizsgálattal kapcsolatos
  tipikus kérdés általában sztochasztikus analízis segítségével
  válaszolható meg.

  A kritikus rendszerek elosztott és aszinkron tulajdonságai az
  \emph{állapottér robbanás} jelenségéhez vezetnek. Emiatt méretük és
  komplexitásuk gyakran megakadályozza a sikeres sztochasztikus
  analízist, melynek számításigénye nagyban függ a lehetséges
  viselkedések számától. A modellek komponenseinek jellegzetes időbeli
  viselkedése és leginkább eltérő karakterisztikája a számításigény
  további jelentős növekedését okozhatja.

  A szolgáltatásbiztonsági és teljesítményjellemzők kiszámítása
  markovi modellek állandósult állapotbeli és tranziens megoldását
  igényli. Számos eljárás ismert ezen problémák kezelésére, melyek
  eltérő reprezentációkat és numerikus algoritmusokat alkalmaznak; ám
  a modellek változatos tulajdonságai miatt nem választható ki olyan
  eljárás, mely minden esetben hatékony lenne.

  A dolgozatban bemutatjuk az irodalomban ismert, markovi
  sztochasztikus rendszerek állandósult állapotbeli és tranziens
  viselkedésének vizsgálatára alkalmas numerikus algoritmusokat. Az
  algoritmusokat konfigurálható adatstruktúrával és lineáris algebrai
  műveletekkel valósítottuk meg.

  A bevezetett konfigurálható sztochasztikus analízis keretrendszer
  lehetővé teszi a sztochasztikus viselkedéseket leíró különböző
  mátrix-dekompozíciók és az analízis algoritmusok használatát
  állandósult állapotbeli, tranziens, első hiba várható idő és
  érzékenységvizsgálatok elvégzésére. Az elkészített eszközt
  integráltuk a \textsc{PetriDotNet} modellező szoftverrel.

  Módszerünk gyakorlati alkalmazhatóságát szintetikus és ipari
  modelleken végzett mérésekkel igazoljuk.

  \paragraph{Kulcsszavak} aszinkron rendszerek, teljesítményvizsgálat,
  sztochasztikus modell, numerikus módszerek, érzékenységvizsgálat
\end{otherlanguage}

\cleardoublepage

\paragraph*{Abstract}
\phantomsection
\addcontentsline{toc}{chapter}{Abstract}
\thispagestyle{plain}

Ensuring the correctness of critical systems --~such as
safety-critical, distributed and cloud applications~-- requires the
rigorous analysis of the functional and extra-functional properties of
the system. A large class of typical quantitative questions regarding
dependability and performability are usually addressed by stochastic
analysis.

Recent critical systems are often distributed/asynchronous, leading to
the well-known phenomenon of \emph{state space explosion}. The size
and complexity of such systems often prevents the success of the
analysis due to the high sensitivity to the number of possible
behaviors. In addition, temporal characteristics of the components can
easily lead to huge computational overhead.

Calculation of dependability and performability measures can be
reduced to steady-state and transient solutions of Markovian
models. Various approaches are known in the literature for these
problems differing in the representation of the stochastic behavior of
the models or in the applied numerical algorithms. The efficiency of
these approaches are influenced by various characteristics of the
models, therefore no single best approach is known.

In this thesis we present numerical solution algorithms for the steady
state and transient analysis of Markovian models. Various algorithms
were implemented with configurable data structure and linear algebra
operations.

Our framework provides configurable stochastic analysis: an approach
is introduced to combine different matrix representations of stochastic
behaviors with numerical solution algorithms for steady-state,
transient, mean-time-to-first-failure and sensitivity problems. 

The goal of our work is to introduce a framework that facilitates the
analysis of complex, stochastic systems by combining the advantages of
compact matrix representations and various numerical algorithms. The
analysis tool is integrated into the \textsc{PetriDotNet} modeling
application.

Benchmarks and industrial case studies are used to evaluate the
applicability of our approach.

\paragraph{Keywords} asynchronous systems, performance analysis,
stochastic model, numeric methods, sensitivity analysis
