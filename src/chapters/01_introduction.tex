\chapter{Introduction}
\label{chap:introduction}

The growing need for ensuring the correctness of critical systems
--~such as safety-critical, distributed and cloud applications~--
requires the rigorous analysis of the functional and extra-functional
properties.  A large class of typical quantitative questions regarding
dependability and performability are usually addressed by stochastic
analysis.

Recent critical systems are often distributed/asynchronous, leading to
the well-known phenomenon of \emph{state space explosion}. The size
and complexity of such systems often prevents the success of the
analysis due to the high sensitivity to the number of possible
behaviors. In addition, temporal characteristics of the components can
easily lead to huge computational overhead or prevent algorithms from
convergence.

Calculation of dependability and performability measures can be
reduced to steady-state and transient solutions of Markovian
models. Various approaches are known in the literature for these
problems differing in the representation of the stochastic behavior of
the models or in the applied numerical algorithms. The efficiency of
these approaches are influenced by various characteristics of the
models, therefore no single best approach is known.

In this paper our goal is to propose a numerical backed for the
solution of the various problems occurring in stochastic analysis of
complex systems.

In \cite{TDK2015_Klenik_Marussy} we introduced the concept of
configurable stochastic analysis. We developed a framework to support
the combination of
\begin{itemize}
\item various state space exploration techniques with
\item decomposition algorithms and representation
  techniques for the stochastic behavior of the systems 
\item various numerical algorithms to solve the steady-state and
  transient analysis problem,
\item computation of high level measures such as various reward,
  sensitivity and mean time to failure values.
\end{itemize}

Various problems were solved during our work: an approach is
introduced to combine the different matrix representations with
numerical solution algorithms. The approach consists of a flexible
data structure, which is complemented by a set of linear algebra
operations configurable at runtime.

A diverse set of algorithms are implemented for steady-state reward
and sensitivity analysis, transient reward analysis and
mean-time-to-first-failure analysis of stochastic models. Several
optimizations and improvements were applied to provide efficient
algorithms. Most of the developed algorithms are parallelized to
exploit the modern multicore architectures. Benchmarks and industrial
case studies are used to evaluate the applicability of our approach.

The algorithm development and optimization includes preliminary work
on integrating the \textls{IDR}($s$)\textls{STAB}($\ell$) numerical
linear equations solver into our stochastic analysis framework,
including modifications and tuning for matrices arising in
steady-state Markovian analysis problems. To our best knowledge, ours
is the first preliminary result in this area.

The analysis framework is integrated into the \textsc{PetriDotNet}
modeling application for stochastic models in the \emph{Stochastic
  Petri Net}~\paren{\textls{SPN}} formalism.

More than $78\,000$ unit tests are generated with a combinatorial
interface testing approach to ensure the correctness of the data
structure. To validate the stochastic analysis pipeline and the
implemented algorithms through software redundancy, $588$
mathematically consistent configurations of the pipeline are executed
and evaluated for several models. More than $150$ benchmark runs were
performed with large models and industrial case studies to gather
information about the performance and memory utilization
characteristics of the analysis tool. In addition, $10\,000$ shorter
benchmark runs were used to study the convergence behavior of our
modification of the \textls{IDR}($s$)\textls{STAB}($\ell$) algorithm.

The remainder of this paper is structured as follows:
\Cref{chap:background} reviews some preliminaries of the stochastic
analysis of stochastic Petri nets. \Cref{chap:overview} presents the
configurable stochastic analysis pipeline and its numerical
backend. \Cref{chap:operations} describes the developed data structure
and configurable linear algebra operations. \Cref{chap:algorithms}
presents numerical steady-state and transient analysis algorithms and
their implementations in our framework, with special attention to the
homogeneous linear equation systems arising from steady-state
analysis. In \vref{ssec:algorithms:krylov}, we present the Krylov
subspace solvers utilized, including our preliminary work integrating
\textls{IDR}($s$)\textls{STAB}($\ell$) on
page~\pageref{ssec:algorithms:idrstab}. After describing the testing
and validation methodologies applied to our framework in
\cref{chap:evaluation} as well as the benchmark results, we conclude
our thesis in \cref{chap:conclusion}.
