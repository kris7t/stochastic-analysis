\chapter{Algorithms for stochastic analysis}
\label{chap:algorithms}

Steady state, transient, accumulated and sensitivity analysis problems
pose several numerical challanges, especially when the state space of
the \CTMC\ and the vectors and matrices involved in the computation are
extemely large.

In steady-state and sensitivty analysis, linear equations of the form
$\vec{x} A = \vec{b}$ are solved, such as%
~\vref{eq:background:ctmc:steadystate,eq:background:ctmc:sensitvity:s}. The
steady-state probability vector is the solution of the linear system
\begin{equation}
  \frac{\dd \vec{\uppi}}{\dd t} = \vec{\uppi} \, Q = \vec{0},
  \quad \vec{\uppi} \vec{1}^\T = 1 \text,
  \tag{\ref{eq:background:ctmc:steadystate}~revisited}
\end{equation}
where the infinitesimal generator $Q$ is a rank-deficient
matrix. Therefore, steady-state solution methods must handle various
generator matrix decompositions and homogenous linear equation with
rank deficient matrices. Convergence and computation times of linear
equations solvers depend on the numerical properties of the $Q$
matrices, thus different solvers may be preferred for different
models.

In transient analysis, initial value problems with first-order linear
differetial equations such as%
~\vref{eq:background:ctmc:diffeq,eq:background:ctmc:L-ivp} are
considered. The decomposed generator matrix $Q$ must be also handled
efficiently. Another difficulty is caused by the \emph{stiffness} of
differential equations arising from some models, which may
significantly increase computation times.

To facilitate configurable stochastic analysis, we developed several
linear equation solvers and transient analysis methods. Where it is
reasonable, the implementation is independent of the form of the
generator matrix $Q$.

The implementation of low-level linear algebra operations is also
decoupled from the numerical algorithms and data structure. This
strategy enables further configurability by replacing the operations
at runtime, as described in~\cref{chap:operations}.

In this chapter, we describe the algorithms implemented in our
stochastic analysis framework. The pseudocode of the algorithms is
annotated with the low level operations performed on the configurable
data structure by the high level algorithms.

\section{Linear equation solvers}
\label{sec:algorithms:solvers}

\subsection{Explicit solution by \textls[30]{LU} decomposition}
\label{ssec:algorithms:lu}

\textls{LU} decomposition is a direct method for solving linear
equations with forward and backward substitution, i.e.~it does not
require iteration to reach a given precision.

The decomposition computes the lower triangular matrix $L$ and upper
triangular matrix $U$ such that
\begin{equation}
  A = LU \text.
\end{equation}
To solve the equation $\vec{x} A = \vec{x} LU = \vec{b}$ forward
substitution is applied first to find $\vec{z}$ in
$\vec{z} U = \vec{b}$, then $\vec{x}$ is computed by back substitution
from \vec{x} L = \vec{b}\text.

\begin{algorithm}
  \KwIn{the matrix $A \in \RR^{n \times n}$ operated on in-place}
  \KwOut{$L, U \in \RR^{n \times n}$ such that $A = LU$, $u[i, i] = 1$
    for all $i = 0, 1, \ldots, n - 1$}
  \For{$i \gets 0$ \KwTo $n - 1$}{
    \lFor{$j \gets 0$ \KwTo $i$}{%
      $a[i,j] \gets a[i,j] - \sum_{k = 0}^{j - 1} a[i, k] a[k, j]$ 
    }
    \lFor{$j \gets i + 1$ \KwTo $n - 1$}{%
      $a[i, j] \gets \bigl( a[i, j] - \sum_{k = 0}^{i - 1} a[i, k]
      a[i, j] \bigr) \big/ a[i, i]$ 
    }
  }
  Let $A_L$, $A_D$ and $A_U$ refer to the strictly lower triangular,
  diagonal and strictly upper triangular parts of $A$, respectively.\;
  $L \gets A_L + A_D$\;
  $U \gets A_U + I$\;
  \KwRet{$L, U$}\;
  \caption{Crout's \textls{LU} decomposition without pivoting.}
  \label{alg:algorithms:lu:crout}
\end{algorithm}

We used Crout's \textls{LU} decomposition%
~\citep[Section~2.3.1]{press2007numerical}, presented in
\vref{alg:algorithms:lu:crout}, which ensures
\begin{equation}
  u[i, i] = 1 \text{ for all $i = 0, 1, \ldots, n - 1$,}
\end{equation}
i.e.~the diagonal of the $U$ matrix is uniformly $1$. The matrix is
filled in during the decomposition even if it was initially sparse,
therefore it should first be copied to a dense array storage for
efficiency reasons. This considerably limits the size of Markov chains that
can be analysed by direct solution due to memory requirements. Our
data structure allows access to upper and lower diagonal parts to
matrices and linear combinations, therefore no additional storage is
needed other than $A$ itself.

\begin{algorithm}
  \KwIn{$U, L \in \RR^{n \times n}$, right vector $\vec{b} \in \RR^n$}
  \KwOut{solution of $\vec{x}LU = \vec{b}$}
  \KwAllocate{$\vec{x}, \vec{z} \in \mathbb{R}^n$}\;
  \lIf(\tcp*[f]{Skip forward substitution for homogenous equations})
  {$\vec{b} = \vec{0}$}{$\vec{z} \gets \vec{0}$}
  \KwSty{else} \lFor{$j \gets 0$ \KwTo $n - 1$}{%
    $z[j] \gets b[j] \cdot \sum_{i = 0}^{j - 1} u[i, j]$
  }
  \If{$l[n - 1, n - 1] \approx 0$}{
    \lIf(\tcp*[f]{Set the free parameter to 1})
    {$z[n - 1] \approx 0$}{$x[n - 1] \gets 0$}
    \lElse{%
      \label{ln:algorithms:lu:substitution:not-in-image}
      \KwError{inconsistent linear equation system}
    }
  }
  \lElse{$x[n - 1] \gets z[n - 1] / l[n - 1, n - 1]$}
  \For{$j \gets n - 2$ \KwDownto $0$}{
    \lIf{$l[j, j] \approx 0$}{%
      \label{ln:algorithms:lu:substitution:rank-too-low}
      \KwError{more than one free parameter}
    }
    $x[j] \gets \bigl( z[i] - \sum_{i = j + 1}^{n - 1} x[i] l[i, j]
    \bigr) \big/ l[j, j]$\;
  }
  \KwRet{$\vec{x}$}\;
  \caption{Forward and back substitution.}
  \label{alg:algorithms:lu:substitution}
\end{algorithm}

The forward and back substitution process is shown in
\vref{alg:algorithms:lu:substitution}. If multiple equations are
solver with the same matrix, its \textls{LU} decomposition may be
cached.

\subsubsection{Matrices of less than full rank}

If the matrix $Q$ is of rank $n - 1$, the element $l[n - 1, n - 1]$ in
Crout's \textls{LU} decomposition will be $0$. In this case,
$x[n - 1]$ is a free parameter and will be set to $1$ to yield a
nonzero solution vector when $z[n - 1] = 0$. If $z[n - 1] \ne 0$, the
equation $\vec{x} L = \vec{z}$ does not have a solution and the error
condition in line~\ref{ln:algorithms:lu:substitution:not-in-image} is
triggered. A matrix of rank less than $n - 1$ triggers the error
condition in line~\ref{ln:algorithms:lu:substitution:rank-too-low}.

In practice, the algorithm can be used to solve homogenous equations
in Markovian analysis, because the infinitesimal generator matrix $Q$
of an irreducible \CTMC\ is always of rank $n - 1$. The solution
vector $\vec{x}$ is not a probability vector in general, so it must be
normalized as $\vec{\pi} = \vec{x} / \vec{x} \vec{1}^\T$ to get a
stationary probability distribution vector.

\subsection{Iterative methods}

\begin{algorithm}
  \KwIn{matrix $A \in \RR^{n \times n}$, right vector $\vec{b} \in
    \RR^n$, initial guess $\vec{x} \in \RR^n$, tolerance $\tau > 0$}
  \KwOut{approximate solution of $\vec{x}A = \vec{b}$ and its residual
    norm}
  \KwAllocate{$\vec{x}' \in \mathbb{R}^n$}
  \tcp*{Previous iterate for convergence test}
  \Repeat{$\|\vec{x}' - \vec{x}\| \le \tau$}{
    \nllabel{ln:algorithms:iterative:basic:loopbeg}
    $\vec{x}' \gets \vec{x}$\tcp*{Save the previous vector}
    $\vec{x} \gets f(\vec{x}')$\;
  }
  \nllabel{ln:algorithms:iterative:basic:loopend}
  \KwRet{$\vec{x}$ and $\| \vec{x}Q - \vec{b} \|$}\;
  \caption{Basic iterative scheme for solving linear equations.}
  \label{alg:algorithms:iterative:basic}
\end{algorithm}

Iterative methods express the solution of the linear equation $\vec{x}
A = \vec{b}$ as a recurrence
\begin{equation}
  \vec{x}_{k} = f(\vec{x}_{k - 1}) \text,
\end{equation}
where $\vec{x}_0$ is an initial guess vector. The iteration converges
to a solution vector when $\lim_{k \to \infty} \vec{x}_k = \vec{x}$
exists and $\vec{x}$ equals the true solution vector $\vec{x}^*$. The
iteration is illustrated in \vref{alg:algorithms:iterative:basic}.

The process is assumed to have converged if subsequent iterates are
sufficiently close, i.e.~the stopping criterion at the $k$th iteration
is
\begin{equation}
  \label{eq:algorithms:iterative:convergence}
  \| \vec{x}_{k} - \vec{x}_{k - 1} \| \le \tau
\end{equation}
for some prescribed tolerance $\tau$. In our implementation, we
selected the $L^1$-norm
\begin{equation}
  \| \vec{x}_{k} - \vec{x}_{k - 1} \| = \sum_{i} \bigl\lvert x_{k}[i] - x_{k -
    1}[i] \bigr\rvert
\end{equation}
as the vector norm used for detecting convergence.

Premature termination may be avoided if iterates spaced $m > 1$
iterations apart are used for convergence test
($\| \vec{x}_k - \vec{x}_{k - m} \| \le \tau$), but only at the
expense of additional memory required for storing $m$ previous
iterates. In order to handle large Markov chains with reasonable
memory consumption, we only used the convergence test with a single
previous iterate.

Correctness of the solution can be checked by observing the norm of
the residual $\vec{x}_{k} A - \vec{b}$, since the error vector
$\vec{x}_k - \vec{x}^*$ is generally not available. Because the
additional matrix multiplication may make the latter check costly, it
is performed only after detecting convergence by
\vref{eq:algorithms:iterative:convergence}. Unfortunately, the
residual norm may not be representative of the error norm if the
problem is ill-conditioned.

For a detailed discussion stopping criterions and iterate
normalization in steady-state \CTMC\ analysis, we refer
to~\citep[Section~10.3.5]{stewart2009probability}.

\subsubsection{Power iteration}
\label{ssec:algorithms:power}

Power iteration~\citep[Section~10.3.1]{stewart2009probability} is the
one of the simplest iterative methods for Markovian analysis. Its
iteration function has the form
\begin{equation}
  \vec{x}_k = f(\vec{x}_{k - 1}) = \vec{x}_{k - 1} + \frac{1}{\alpha} \mleft(
  \vec{x}_{k - 1} A - \vec{b} \mright) \text.
\end{equation}

The iteration converges if the diagonal elements $a[i,i]$ of $A$ are
strictly negative, the off-diagonal elements $a[i,j]$ are nonnegative
and $\alpha \ge \max_{i} \lvert a[i,i] \rvert$. The matrix $A$
satisfies these properties if it is an inifinitesimal generator matrix
of an irreducible \CTMC. The fastest convergence is achieved when
$\alpha = \min_{i} \lvert a[i,i] \rvert$.

\begin{algorithm}
  $\alpha^{-1} \gets 1 / \max_{i} \lvert a[i,i] \rvert$\;
  \Repeat{$\epsilon \le \tau$}{
    $\vec{x}' \gets \vec{x} A$
    \Operation*{VectorMatrixMultiplyFromLeft}
    $\vec{x}' \gets \vec{x}' + (-1) \cdot \vec{x}$
    \Operation*{\InPlace VectorAdd}
    $\epsilon \gets \alpha^{-1} \| \vec{x}' \|$
    \Operation*{VectorL1Norm}
    $\vec{x} \gets \vec{x} + \alpha^{-1} \vec{x}'$
    \Operation*{\InPlace VectorAdd}
  }
  \caption{Power iteration.}
  \label{alg:algorithms:iterative:power}
\end{algorithm}

Power iteration can be realized by replacing lines%
~\ref{ln:algorithms:iterative:basic:loopbeg}%
--\ref{ln:algorithms:iterative:basic:loopend} in
\vref{alg:algorithms:iterative:basic} with the loop in
\vref{alg:algorithms:iterative:power}.

This realization uses memory efficiently, because it only requires the
allocation of a single vector $\vec{x}'$ in addition to the initial
guess $\vec{x}$.

\begin{obs}
  \label{obs:algorithms:iterative:power-keepnorm}
  If $\vec{b} = 0$ and $A$ is an inifitesimal generator matrix, then
  \begin{align}
    \vec{x}_k \vec{1}^\T &= \mleft[ \vec{x}_{k - 1} + \frac{1}{\alpha}
                           \mleft( \vec{x}_{k - 1} A - \vec{b}
                           \mright) \mright] \vec{1}^\T \\
                         &= \vec{x}_{k - 1} \vec{1}^\T +
                           \frac{1}{\alpha} \vec{x}_{k - 1} A
                           \vec{1}^T - \vec{b} \vec{1}^\T \\
                         &= \vec{x}_{k - 1} \vec{1}^\T +
                           \frac{1}{\alpha} \vec{x}_{k - 1} \vec{0}^\T
                           - \vec{0} \vec{1}^\T = \vec{x}_{k - 1} \vec{1}^\T.
  \end{align}
  This means the sum of the elements of the result vector $\vec{x}$
  and the initial guess vector $\vec{x}_0$ are equal, because the
  iteration leaves the sum unchanged.
\end{obs}

To solve an equation of the form
\begin{equation}
  \label{eq:algorithms:iterative:homogenous-prob}
  \vec{x} Q = \vec{0},\quad \vec{x} \vec{1}^\T = 1
\end{equation}
where $Q$ is an infinitesimal generator matrix, the initial guess
$\vec{x}_0$ is selected such that $\vec{x}_0 \vec{1}^\T = 1$. If the
\CTMC\ described by $Q$ is irreducible, we may select
\begin{equation}
  \label{eq:algorithms:iterative:power-init}
  x_0[i] \equiv \frac{1}{n} \text,
\end{equation}
where $n$ is the dimensionality of $\vec{x}$. After the initial guess
is selected, the equation $\vec{x} \vec{1}^\T$ may be ignored to solve
$\vec{x} Q = \vec{0}$ with the power method. This process yields the
solution of the original problem%
~\eqref{eq:algorithms:iterative:homogenous-prob}.

\subsubsection{Jacobi and Gauss--Seidel iteration}
\label{ssec:algorithms:jgs}

Jordan and Gauss--Seidel iterative methods%
~\citep[Section~10.3.2--3]{stewart2009probability} repeatedly solve a
system of simultaneous equations of a specific form.

In Jordan iteration, the system
{\small\begin{equation} \left.
      \begin{array}{r@{${} = {}$}*{2}{r@{\kern1pt}l@{${} + {}$}}c@{${} + {}$}r@{\kern1pt}l}
        b[0] & x_{k}[0] & a[0,0] & x_{k - 1}[1] & a[1,0] & \cdots & x_{k - 1}[n - 1] & a[n - 1,0] \text,\\
        b[1] & x_{k - 1}[0] & a[0,1] & x_{k}[1] & a[1,1] & \cdots & x_{k - 1}[n - 1] & a[n - 1,1] \text,\\
        \multicolumn{8}{c}{$\vdots$} \\
        b[n - 1] & x_{k - 1}[0] & a[0,n - 1] & x_{k - 1}[1] & a[1,n - 1] & \cdots & x_{k}[n - 1] & a[n - 1,n - 1] \text,\\
      \end{array}
    \right\}
  \end{equation}}%
is solved for $\vec{x}_{k}$ at each iteration, i.e.~there is a
single unknown in each row and the rest of the variables are taken
from the previous iterate. In vector form, the iteration can be
expressed as
\begin{equation}
  \vec{x}_{k} = A_D^{-1} (\vec{b} - A_O \vec{x}_{k - 1}) \text,
\end{equation}
where $A_D$ and $A_O$ are the diagonal (all off-diagonal elements are
zero) and off-diagonal (all diagonal elements are zero) parts of $A =
A_D + A_O$.

In Gauss--Seidel iteration, the linear system
{\small\begin{equation}
    \left.
      \begin{array}{r@{${} = {}$}*{2}{r@{\kern1pt}l@{${} + {}$}}c@{${} + {}$}r@{\kern1pt}l}
        b[0] & x_{k}[0] & a[0,0] & x_{k - 1}[1] & a[1,0] & \cdots & x_{k - 1}[n - 1] & a[n - 1,0] \text,\\
        b[1] & x_{k}[0] & a[0,1] & x_{k}[1] & a[1,1] & \cdots & x_{k - 1}[n - 1] & a[n - 1,1] \text,\\
        \multicolumn{8}{c}{$\vdots$} \\
        b[n - 1] & x_{k}[0] & a[0,n - 1] & x_{k}[1] & a[1,n - 1] & \cdots & x_{k}[n - 1] & a[n - 1,n - 1] \text,\\
      \end{array}
    \right\}
  \end{equation}}%
is considered, i.e.~the $i$th equation contains the first $i$ elements
of $\vec{x}_k$ as unknowns. The equations are solved for successive
elements of $\vec{x}_k$ from top to bottom.

\begin{algorithm}
  \KwIn{matrix $A \in \RR^{n \times n}$, right vector $\vec{b} \in
    \RR^n$, initial guess $\vec{x} \in \RR^n$, tolerance $\tau > 0$,
    over-relaxation parameter $\omega > 0$}
  \KwOut{approximate solution of $\vec{x}A = \vec{b}$}
  \KwAllocate{$\vec{x}' \in \RR^n$}\;
  Let $A_O$ refer to the off-diagonal part of $A$.\;
  \Repeat{$\epsilon \le \tau$}{
    $\vec{x}' \gets \vec{x} A_O$
    \Operation*{VectorMatrixMultiplyFromLeft}\label{ln:algorithms:jacobi:product}
    $\vec{x}' \gets \vec{x}' + (-1) \cdot \vec{b}$\label{ln:algorithms:jacobi:addition}
    \Operation*{\InPlace VectorAdd}
    $\epsilon \gets 0$\;
    \For{$i \gets 0$ \KwTo $n - 1$}{
      \label{ln:algorithms:jacobi:innerloop}
      $y \gets (1 - \omega) x[i] - \omega x'[i] / a[i, i]$\;
      $\epsilon \gets \epsilon + \lvert y - x[i] \rvert$\;
      $x[i] \gets y$\;
    }
  }
  \KwRet{$\vec{x}$}\;
  \caption{Jacobi over-relaxation.}
  \label{alg:algorithms:iterative:jacobi}
\end{algorithm}

Jacobi over-relaxation, a generalized form of Jacobi iteraion, is
realized in \vref{alg:algorithms:iterative:jacobi}. The value $1$ of
the over-relaxation paramter $\omega$ corresponds to ordinary Jacobi
iteration. Values $\omega > 1$ may accelerate convergence, while
$0 < \omega < 1$ may help diverging Jacobi iteration converge.

Jacobi over-relaxation has many parallelization opportunities. The
matrix multiplication in line~\ref{ln:algorithms:jacobi:product} and
the vector addition in line~\ref{ln:algorithms:jacobi:addition} can be
parallelized, as well as the for loop in
line~\ref{ln:algorithms:jacobi:innerloop}. Our implementation takes
advantage of the configurable linear algebra operations framework to
execute lines~\ref{ln:algorithms:jacobi:product} and
\ref{ln:algorithms:jacobi:addition} with possible paralellization
considering the structures of both the vectors $\vec{x}, \vec{x}'$ and
the matrix $A$. However, the inner loop is left sequential to reduce
implementation complexity, as it represents only a small fraction of
execution time compared to the matrix-vector product.

\begin{algorithm}
  \KwIn{matrix $A \in \RR^{n \times n}$, right vector $\vec{b} \in
    \RR^n$, initial guess $\vec{x} \in \RR^n$, tolerance $\tau > 0$,
    over-relaxation parameter $\omega > 0$}
  \KwOut{approximate solution of $\vec{x}A = \vec{b}$}
  \KwAllocate{$\vec{x}' \in \RR^n$}\;
  Let $A_O$ refer to the off-diagonal part of $A$.\;
  \Repeat{$\epsilon \le \tau$}{
    $\epsilon \gets 0$\;
    \For{$i \gets 0$ \KwTo $n - 1$}{
      $\textit{scalarProduct} \gets \vec{x} \cdot \vec{a}_O[\cdot, i]$
      \Operation*{VectorMatrixScalarProductWithColumn}
      $y \gets \omega (b[i] - \textit{scalarProduct}) / a[i, i] + (1 -
      \omega) \cdot x[i]$\;
      $\epsilon \gets \epsilon + \lvert y - x[i] \rvert$\;
      $x[i] \gets y$\;
    }
  }
  \KwRet{$\vec{x}$}\;
  \caption{Gauss--Seidel successive over-relaxatation.}
  \label{alg:algorithms:iterative:gs}
\end{algorithm}

\Vref{alg:algorithms:iterative:gs} shows an implementation of
successive over-relaxation for Gauss--Seidel iteration, where the
notation $\vec{a}_O[\cdot, i]$ refers to the $i$th column of $A_O$.

Gauss--Seidel iteration cannot easily be parallelized, because
calculation of successive elements $x[0], x[1], \ldots$ depend on all
of the prior elements. However, in contrast with Jacobi iteration, no
memory is required in addition to the vectors $\vec{x}$, $\vec{b}$ and
the matrix $X$, which makes the algorithm suitable for very large
vectors and memory-constrained situations. In addition, convergence is
often significantly faster.

\begin{figure}
\begin{equation}
  \label{eq:algorithms:iterative:normalized-linear-system}
  \vec{x} \begin{pmatrix}
    q[0,0] & q[0,1] & \cdots & q[0,n - 2] & 1 \\
    q[1,0] & q[1,1] & \cdots & q[1,n - 2] & 1 \\
    \vdots & \vdots & \ddots & \vdots & \vdots \\
    q[n - 2,0] & q[n - 2,1] & \cdots & q[n - 2,n - 2] & 1 \\
    q[n - 1,0] & q[n - 1,1] & \cdots & q[n - 2,n - 1] & 1 \\
  \end{pmatrix} = \begin{pmatrix}
    0 \\ 0 \\ \vdots \\ 0 \\ 1
  \end{pmatrix} \text,
\end{equation}
\end{figure}

The sum of elements $\vec{x} \vec{1}^\T$ does not stay constant during
Jacobi or Gauss--Seidel iteration. Thus, when solving equations of the
form $\vec{x} Q = \vec{0}, \vec{x} \vec{1}^\T = 1$, normalization
cannot be entierly handled by the initial guess. We instead transform
the equation into the form in
\cref{eq:algorithms:iterative:normalized-linear-system}, where we take
advantage of the fact that the infinitesimal generator matrix is not
of full rank, therefore one of the columns is redundant and can be
replaced with the condition $\vec{x} \vec{1}^\T = 1$.

While this transformation may affect the convergence behavior of the
algorithm, it allows uniform handling of homogenous and non-homogenous
linear equations.

\subsection{Group iterative methods}
\label{sec:algorithms:group-jgs}

\emph{Group} or \emph{block} iterative methods%
~\citet[Section~10.4]{stewart2009probability} assume the block
structure for the vectors $\vec{x}$, $\vec{b}$ and the matrix $A$
\begin{equation}
  \vec{x}[i] \in \RR^{n_i}, \vec{b}[j] \in \RR^{n_j}, A[i, j] \in
  \RR^{n_i \times n_j} \text{ for all $i, j \in \{0, 1, \ldots,
    N - 1\}$,}
\end{equation}
Infinitesimal generator matrices in the block Kronecker decomposition
along with appropriately partitioned vectors match this structure (see
\vref{eq:genstor:explicit:block:block}). Each block of $\vec{x}$
corresponds to a group a variables that are simultaneously solved for.

Group Jacobi iteration solves the linear system
{\small\begin{equation}
    \left.
      \begin{array}{r@{${} = {}$}*{2}{r@{\kern1pt}l@{${} + {}$}}c@{${} + {}$}r@{\kern1pt}l}
        \vec{b}[0] & \vec{x}_{k}[0] & A[0,0] & \vec{x}_{k - 1}[1] & A[1,0] & \cdots & \vec{x}_{k - 1}[n - 1] & A[n - 1,0] \text,\\
        \vec{b}[1] & \vec{x}_{k - 1}[0] & A[0,1] & \vec{x}_{k}[1] & A[1,1] & \cdots & \vec{x}_{k - 1}[n - 1] & A[n - 1,1] \text,\\
        \multicolumn{8}{c}{$\vdots$} \\
        \vec{b}[n - 1] & \vec{x}_{k - 1}[0] & A[0,n - 1] & \vec{x}_{k - 1}[1] & A[1,n - 1] & \cdots & \vec{x}_{k}[n - 1] & A[n - 1,n - 1] \text,\\
      \end{array}
    \right\}
  \end{equation}}%
while group Gauss--Seidel considers
{\small\begin{equation}
    \left.
      \begin{array}{r@{${} = {}$}*{2}{r@{\kern1pt}l@{${} + {}$}}c@{${} + {}$}r@{\kern1pt}l}
        \vec{b}[0] & \vec{x}_{k}[0] & A[0,0] & \vec{x}_{k - 1}[1] & A[1,0] & \cdots & \vec{x}_{k - 1}[n - 1] & A[n - 1,0] \text,\\
        \vec{b}[1] & \vec{x}_{k}[0] & A[0,1] & \vec{x}_{k}[1] & A[1,1] & \cdots & \vec{x}_{k - 1}[n - 1] & A[n - 1,1] \text,\\
        \multicolumn{8}{c}{$\vdots$} \\
        \vec{b}[n - 1] & \vec{x}_{k}[0] & A[0,n - 1] & \vec{x}_{k}[1] & A[1,n - 1] & \cdots & \vec{x}_{k}[n - 1] & A[n - 1,n - 1] \text.\\
      \end{array}
    \right\}
  \end{equation}}

\begin{algorithm}
  \KwIn{block matrix $A$, block right vector $\vec{b}$, block initial
    guess $\vec{n}$, tolerance $\tau > 0$, over-relaxation parameter
    $\omega > 0$}
  \KwOut{approximate solution of $\vec{x}A = \vec{b}$ and its residual
    norm}
  \KwAllocate{$\vec{x}'$ and $\vec{c}$ with the same block structure
    as $\vec{x}$ and $\vec{b}$}\;
  Let $A_{OB}$ represent the off-diagonal part of the block matrix $A$
  with the blocks along the diagonal set to zero.\;
  \Repeat{$\epsilon \le \tau$}{
    $\vec{x}' \gets \vec{x}, \vec{c} \gets \vec{b}$\;
    $\vec{c} \gets \vec{c} + (-1) \cdot \vec{x}' A_{OB}$
    \Operation*{VectorMatrixAccumulateMultiplyFromLeft}
    \ParFor(\tcp*[f]{Loop over all blocks}){$i \gets 0$ \KwTo $N - 1$}{
      Solve $\vec{x}[i] A[i,i] = \vec{c}[i]$ for $\vec{x}[i]$\;
    }
    $\epsilon \gets 0$\;
    \For(\tcp*[f]{Loop over all elements}){$k \gets 0$ \KwTo $n - 1$}{
      $y \gets \omega x[k] + (1 - \omega) x'[k]$\;
      $\epsilon \gets \epsilon + \lvert y - x'[k] \rvert$\;
      $x[k] \gets y$\;
    }
  }
  \caption{Group Jacobi over-relaxation.}
  \label{alg:algorithms:iterative:group-jacobi}
\end{algorithm}

\begin{algorithm}
  \KwIn{block matrix $A$, block right vector $\vec{b}$, block initial
    guess $\vec{n}$, tolerance $\tau > 0$, over-relaxation parameter
    $\omega > 0$}
  \KwOut{approximate solution of $\vec{x}A = \vec{b}$ and its residual
    norm}
  \KwAllocate{$\vec{x}'$ and $\vec{c}$ large enough to store a single
    block of $\vec{x}$ and $\vec{b}$.}\;
  \Repeat{$\epsilon \le \tau$}{
    $\epsilon \gets 0$\;
    \For(\tcp*[f]{Loop over all blocks}){$i \gets 0$ \KwTo $N - 1$}{
      $\vec{x}' \gets \vec{x}[i], \vec{c} \gets \vec{b}[i]$\;
      \For{$j \gets 0$ \KwTo $N - 1$}{
        \If(\Operation*[f]{VectorMatrixAccumulateMultiplyFromLeft}){$i \ne j$}{
          $\vec{c} \gets \vec{c} + (-1) \cdot \vec{x}[j] A[i, j]$\;
        }
      }
      Solve $\vec{x}[i] A[i,i] = \vec{c}$ for $\vec{x}[i]$\;
      \For{$k \gets 0$ \KwTo $n_i - 1$}{
        $y \gets \omega x[i][k] + (1 - \omega) x'[k]$\;
        $\epsilon \gets \epsilon + \lvert y - x'[k] \rvert$\;
        $x[i][k] \gets y$\;
      }
    }
  }
  \caption{Group Gauss--Seidel successive over-relaxation.}
  \label{alg:algorithms:iterative:group-gs}
\end{algorithm}

Implementations of group Jacobi over-relaxation and group Gauss--Seidel
successive over-relaxation are shown in
\vref{alg:algorithms:iterative:group-jacobi,alg:algorithms:iterative:group-gs}.
The inner linear equations of the form $\vec{x}[i] A[i,i] = \vec{c}$
may be solved by any algorithm, for example, LU decomposition,
iterative methods, or even block-iterative methods if $A$ has a
two-level block structure. The choice of the inner algorithm may
significantly affect performance and care must be taken to avoid
diverging inner solutions in an iterative solver is used.

In Jacobi over-relaxation, paralellization of both the matrix
multiplication and the inner loop is possible. However, two vectors of
the same size as $\vec{x}$ are required for temporary storage.

Gauss--Seidel successive over-relaxation cannot be parallelized
easily.  However it requires only two temporary vectors of size equal
to the largest block of $\vec{x}$, much less than Jacobi
over-relaxation. Moreover, it often requires fewer steps to converge,
making it preferable over Jacobi iteration.

Because the inner solver may be selected by the user and thus its
convergence behaviour varies widely, we do not perform the
transformation for homogenous equations
\eqref{eq:algorithms:iterative:normalized-linear-system}. Instead, the
normalization $\vec{\uppi} = \vec{x} / \vec{x}\vec{1}^\T$ is performed
only after finding any nonzero solution of $\vec{x} Q = \vec{0}$.

For a detailed analysis of the convergence behaviour of group
iterative methods, we refer to
\citet[Chapter~14]{greenbaum1997iterative} and
\citet{courtois1986block}.

\subsection{Krylov subspace methods}

Projectional iterative methods are iterative linear equation solvers
that produce a sequence of approximate solutions $\vec{x}_k$ of the
linear equation $\vec{x} A = \vec{b}$ that satisfy the
Petrov--Galerkin conditions~\citep[Section~5.1.1]{saad2003iterative}
\begin{equation}
  \label{eq:algorithms:krylov:galerkin}
  \vec{x}_k \in \mathcal{K}_k, \quad \vec{r}_k = \vec{b} - \vec{x}_k A
  \perp \mathcal{L}_k \text,
\end{equation}
where $\mathcal{K}_k$ and $\mathcal{L}_k$ are two subspaces of
$\RR^n$ and $\vec{r}_k$ is residual in the $k$th iteration.

Krylov subspace iterative methods correspond to the choice
\begin{equation}
  \mathcal{K}_k = \mathcal{K}_{k}(A, \vec{r}_0) = \Span \{ \vec{r}_0,
  \vec{r}_0 A, \vec{r}_0 A^2, \ldots, \vec{r}_0 A^{k - 1} \} \text,
\end{equation}
where $\mathcal{K}_{k}(A, \vec{r}_0)$ is the \emph{$k$th Krylov
  subspace} of $A$ and the initial residual $\vec{r}_0 = \vec{b} -
\vec{x}_0 Q$.

The smallest $m \in \NN$ such that
$\dim \mathcal{K}_m(A, \vec{r}_0) = \dim \mathcal{K}_{m + 1}(A,
\vec{r}_0)$
is called the \emph{grade} of $A$ with respect to $\vec{r}_0$. Hence
$k \le m$ implies $\dim \mathcal{K}_k(A, \vec{r}_0) = k$. Krylov
subspace solvers usually suppose that the algorithm terminates at some
iteration $k^*$ such that $k^* \le m$, therefore the dimension of
$\mathcal{K}_k$ increases with each iteration. The contary situation
leads to stagnation, because
$\mathcal{K}_k \subseteq \mathcal{K}_{k + 1}$ together with
$\dim \mathcal{K}_k = \dim \mathcal{K}_{k + 1}$ ($k \ge m$) implies
$\mathcal{K}_k = \mathcal{K}_{k + 1}$.

The subspace $\mathcal{L}_k$ also must be a $k$-dimensional subspace
of $\RR^n$. Conceptually, while the Krylov subspace $\mathcal{K}_k$
``expands'' in dimensionality every iteration, the subspace
$\mathcal{L}_k$ likewise fills the space to make additional residuals
forbidden by the Petrov--Galerkin condition~%
\eqref{eq:algorithms:krylov:galerkin}.

If $A \in \RR^{n \times n}$ is of full rank and grade, Krylov subspace
solvers find the exact solution of the linear equation in at most $n$
iterations with exact arithmetic. The only possible orthogonal
residual is the zero vector $\vec{0}$ if $\mathcal{L}_n = \RR^n$
holds. While $n$ is usually too large for this to be practical,
convergence often happens with suitable accuracy after a small number
of iterations.

Note that problems may arise when $A$ is singular, which may worsen
the convergence behaviour. This is the case in \textls{CTMC} analysis,
where the infinitesimal generator matrix $Q$ is of rank $n - 1$.

Some Krylov subspace methods for nonsymmetric matrices in wide use are
Generalized Minimum Residual~(\textls{GMRES})~\citep{saad1986gmres},
Bi-Conjugate Gradient Stabilized~(\textls{BiCGSTAB})
\citep{van1992bi}, Conjugate Gradient Squared%
~(\textls{CGS})~\citep{sonneveld1989cgs} and
\textls{IDR$(s)$}~\citep{sonneveld2008idr}.

\subsubsection{Generalized Minimal Residual (\textls[30]{GMRES})}
\label{ssec:algorithms:gmres}

Generalized Minimal Residual~(\textls{GMRES})%
~\mkbibbrackets{\nakedcite[Section~6.5.1]{saad2003iterative};~\nakedcite{saad1986gmres}}
a Krylov subspace method for nonsymmetric linear systems. It is based
on the choice
\begin{equation}
  \label{eq:algorithms:gmres:lspace}
  \mathcal{L}_k = \mathcal{K}_k A = \{ \vec{r}_0A, \vec{r}_0A^2,
  \ldots, \vec{r}_0 A^k \} \text.
\end{equation}

With this choice, the Petrov--Galerkin condition~%
\eqref{eq:algorithms:krylov:galerkin} minimizes the Euclidean norm of
the residuals in each iteration, i.e.
\begin{equation}
  \label{eq:algorithms:gmres:minres}
  \vec{x}_k \in \mathcal{K}_k \text{ such that } \vec{r}_k = \vec{b} - \vec{x}_k A \perp
  \mathcal{L_k} \quad\Longleftrightarrow\quad \vec{x}_k = \argmin_{\vec{x} \in
    \mathcal{K}_k} \| \vec{b} - \vec{x} Q \|_2 \text.
\end{equation}

Unfortunately, the solution of \cref{eq:algorithms:gmres:minres}
requires the storage of a basis of $\mathcal{K}_k$, which is a $k$
dimensional subspace of $\RR^n$. Thus, each iteration requires the
allocation of an additional vector. Solution of a linear system with
\textls{GMRES} requires up to $n$ additional floating-point vectors of
$n$ elements each, i.e. $O(n^2)$ floating-point numbers. This property
makes \textls{GMRES} a ``long recurrence'' algorithm.

The high memory requirements may be alleviated by discarding the basis
of $\mathcal{K}_k$ and restarting the iteration from another initial
guess $\vec{x}_0$ if no solution is obtained after $\ell$
iterations. The resulting algorithm is called \textls{GMRES}($\ell$).

The convergence behaviour of full \textls{GMRES} is often
excellent. However, due to impractical memory requirements, we did not
implement \textls{GMRES} as a numerical solver in our framefork.  We
instead use \textls{BiCGSTAB} and
\textls{IDR}($s$)\textls{STAB($\ell$}), Krylov subspace solvers
incorporating \textls{GMRES}($\ell$)-like steps.

\subsubsection{Bi-Conjugate Gradient Stabilized (\textls[30]{BiCGSTAB})}
\label{ssec:algorithms:bicgstab}

Bi-Conjugate Gradient Stabilized~(\textls{BiCGSTAB})%
~\mkbibbrackets{\nakedcite[Section~7.4.2]{saad2003iterative};~\nakedcite{van1992bi}}
is a Krylov subspace method where~\citep{simoncini2010interpreting}
\begin{equation}
  \label{eq:algorithms:bicgstab:stabpoly}
  \mathcal{L}_k = \mathcal{K}_k(A^\T, \tilde{\vec{r}}_0) \cdot
  (\Omega_k(A)^\T)^{-1}, \quad
  \Omega_k(A) = \begin{cases}
    \Omega_{k - 1}(A) \cdot (I - \omega_k A) & \text{if $k \ge 1$,} \\
    I & \text{if $k = 0$.}
  \end{cases}
\end{equation}
The \emph{initial shadow residual} $\tilde{\vec{r}}_0$ must satisfy
$\vec{r}_0 \tilde{\vec{r}}_0^\T \ne 0$ and must not be an eigenvector
of $Q^\T$. Usually, $\tilde{\vec{r}}_0 = \vec{r}_0$, which is the
convention we use in our implementation.

Equivalently, \textls{BiCGSTAB} is a Krylov subspace method which
produces residuals
\begin{equation}
  \label{eq:algorithms:bicgstab:sonneveld}
  \vec{r}_k \in \mathcal{G}_k, \quad
  \mathcal{G}_k = \begin{cases}
    (\mathcal{G}_{k} \cap \tilde{\vec{r}}_0^{\perp}) (I - \omega_k
    A) & \text{if $k \ge 1$}, \\
    \RR^n & \text{if $k = 0$,}
  \end{cases}
\end{equation}
where $\mathcal{A}^{\perp}$ is the set of vector orthogonal to
$\mathcal{A}$. It can be shown that%
~\citep{sleijpen2010exploiting}
\begin{equation}
  \mathcal{G}_k = \mathcal{S}(\Omega_k, A, \tilde{\vec{r}}_0) = \{
  \vec{v} \cdot \Omega_k(A) : \vec{v} \perp \mathcal{K}_k(A^\T,
  \tilde{\vec{r}}_0) \} \text,
\end{equation}
where $\mathcal{S}(\Omega, A, \tilde{\vec{r}}_0)$ is the
$\Omega$\kern0.5pt th \emph{Sonneveld subspace} generated by $A$ and
$\tilde{\vec{r}}_0)$ of order $k = \deg \Omega$. Hence
$\mathcal{L}_k = \mathcal{G}_k^{\perp}$, which makes \textls{BiCGSTAB}
equivalent to another Krylov subspace method, Induced Dimensionality
Reduction (\textls{IDR})~\citep{sleijpen2010exploiting} in exact
arithmetic.

\textls{BiCGSTAB} is a ``short recurrence'', that is, the number of
allocated intermediate vectors does not depend on the number of
variables in equation system.

\paragraph{Implementation}

We selected \textls{BiCGSTAB} as the first Krylov subspace solver
integrated into our framework because of its good convergence
behaviour and low memory requirements. \textls{BiCGSTAB} only requires
the storage of 7 vectors, which makes it suitable even for large state
spaces with large state vectors.

\begin{algorithmpage}
  \KwIn{matrix $A \in \RR^{n \times n}$, right vector $\vec{b} \in
    \RR^n$, initial guess $\vec{x} \in \RR^n$, tolerance $\tau > 0$}
  \KwOut{approximate solution of $\vec{x}A = \vec{b}$}
  \KwAllocate{$\vec{r}, \tilde{\vec{r}}_0, \vec{v}, \vec{p}, \vec{s}, \vec{t}
    \in \RR^n$}\;
  $\vec{r} \gets \vec{b}$
  \Operation*{VectorSet}
  $\vec{r} \gets \vec{r} + (-1) \cdot \vec{x} A$
  \Operation*{VectorMatrixAccumulateMultiplyFromLeft}
  \If{$\| \vec{r} \| \le \tau$}{
    \KwMessage{initial guess is correct, skipping iteration}\;
    \KwRet{$\vec{x}$}\;
  }
  $\tilde{\vec{r}}_0 \gets \vec{r}, \vec{v} \gets \vec{0}, \vec{p} \gets
  \vec{0}, \rho' \gets 1, \alpha \gets 1, \omega \gets 1$\;
  \While{\KwTrue}{
    \AlgSection{\textls[30]{Bi-CG} step}
    $\rho \gets \vec{r_0} \cdot \vec{r}$\label{nl:algorithms:bicgstab:bicg-start}
    \Operation*{VectorScalarProduct}
    \lIf{$\rho \approx 0$}{%
      \KwError{breakdown: $\vec{r} \perp \tilde{\vec{r}}_0$}
    }
    $\beta \gets \rho / \rho' \cdot \alpha / \omega$\;
    $\vec{p} \gets \vec{r} + \beta \cdot \vec{p}$
    \Operation*{VectorAdd}
    $\vec{p} \gets \vec{p} + (-\beta \omega) \cdot \vec{v}$
    \Operation*{\InPlace VectorAdd}
    $\vec{v} \gets \vec{p} Q$
    \Operation*{VectorMatrixMultiplyFromLeft}
    $\alpha \gets \rho / (\tilde{\vec{r}}_0 \cdot \vec{v})$
    \Operation*{VectorScalarProduct}
    $\vec{r} \gets \vec{s} + (-\alpha) \cdot \vec{s}$
    \Operation*{VectorAdd}
    \If{$\| \vec{s} \| < \tau$}{
      $\vec{x} \gets \vec{x} + \alpha \cdot \vec{p}$
      \Operation*{\InPlace VectorAdd}
      \KwMessage{early return with vanishing $\vec{s}$}\;
      \KwRet{$\vec{x}$}\label{nl:algorithms:bicgstab:bicg-end}\;
    }
    \AlgSection{\textls[30]{GMRES}(1) step}
    $\vec{t} \gets \vec{s} A$\label{nl:algorithms:bicgstab:gm-start}
    \Operation*{VectorMatrixMultiplyFromLeft}
    $\textit{tLengthSquared} \gets \vec{t} \cdot \vec{t}$
    \Operation*{VectorScalarProduct}
    \lIf{$\textit{tLengthSquared} \approx 0$}{%
      \KwError{breakdown: $\vec{t} \approx \vec{0}$}
    }
    $\omega \gets (\vec{t} \cdot \vec{s}) / \textit{tLengthSquared}$
    \Operation*{VectorScalarProduct}
    \lIf{$\omega \approx 0$}{%
      \KwError{breakdown: $\omega \approx 0$}
    }
    $\epsilon \gets 0$\;
    \For{$i \gets 0$ \KwTo $n - 1$}{
      $\textit{change} \gets \alpha p[i] + \omega s[i]$,
      $\epsilon \gets \epsilon + \lvert \textit{change} \rvert$.
      $x[i] \gets x[i] + \textit{change}$\;
    }
    \lIf{$\epsilon \le \tau$}{%
      \KwRet{$\vec{x}$}
    }
    $\vec{s} \gets \vec{t} + (-\omega) \cdot \vec{r}$
    \Operation*{VectorAdd}
    $\rho' \gets \rho$\label{nl:algorithms:bicgstab:gm-end}\;
  }
  \caption{\textls{BiCGSTAB} iteration without preconditioning.}
  \label{alg:algorithms:iterative:bicgstab}
\end{algorithmpage}

\Vref{alg:algorithms:iterative:bicgstab} shows the pseudocode for
\textls{BiCGSTAB}. Our implementation is based on the \textsc{Matlab}
code\footnote{\url{http://www.netlib.org/templates/matlab/bicgstab.m}}
by \citet{barrett1994templates}.

The inner loop of \textls{BiCGSTAB} is the composed of two
procedures. The bi-conjucate gradient (\textls{Bi-CG}) part in lines~%
\ref{nl:algorithms:bicgstab:bicg-start}--\ref{nl:algorithms:bicgstab:bicg-end}
calculates a residual $\vec{t} \in \mathcal{G}_{k - 1} A$ and its
associated approximate solution $\vec{x} \in \mathcal{K}_k$. The
\textls{GMRES}($1$) part in lines~%
\ref{nl:algorithms:bicgstab:gm-start}--\ref{nl:algorithms:bicgstab:gm-end}
selects $\omega_k \in \RR$ and calculates a new residual
$\vec{r} \in \mathcal{G}_{k}$ such that the Euclidean norm
$\| \vec{r} \|_2$ is minimized. This part improves convergence over
the original Bi-Conjucate Gradient algorithm.

Solving preconditioned equations in the form
$\vec{x} A M^{-1} = \vec{b} M^{-1}$ could improve convergence, but was
omitted from our current implementation. As the choice of appropriate
preconditioner matrices $M$ is not trivial%
~\citep{DBLP:journals/informs/LangvilleS04}, implementation and sudy
of preconditioners for Markov chains, especially with block Kronecker
decomposition, is in the scope of our future work.

Because six vectors are allocated in addition to $\vec{x}$ and
$\vec{b}$, the amount of available memory may be a significant
bottleneck.

Similar to \vref{obs:algorithms:iterative:power-keepnorm}, it can be
seen that the sum $\vec{x} \vec{1}^\T$ stays constant throughout
\textls{BiCGSTAB} iteration. Thus, we can find probability vectors
satisfying homogenous equations by the initialization in%
~\vref{eq:algorithms:iterative:power-init}.

\subsubsection{Induced Dimensionality Reduction Stabilized (\textls[30]{IDRSTAB})}
\label{ssec:algorithms:idrstab}

Induced Dimensionality Reduction Stabilized%
~(\textls{IDR}($s$)\textls{STAB}($\ell$))%
~\citep{sleijpen2010exploiting} is Krylov subspace solver that
generalizes \textls{BiCGSTAB} and \textls{IDR} techniques to provide
converge behaviors closely matching \textls{GMRES} while maintaining
the short recurrence property.

As the algorithm developed relatively recently in
\citeyear{sleijpen2010exploiting}, high performance implementations of
\textls{IDR}($s$)\textls{STAB}($\ell$) are not widely available. To
our best knowledge, \textls{IDR}($s$)\textls{STAB}($\ell$) was not
investigated for use in \textls{CTMC} analysis despite its promising
results solving differential equations arising from finite element
problems. Therefore, we are currently focusing research and
development effort into integrating
\textls{IDR}($s$)\textls{STAB}($\ell$) into our stochastic analysis.
Special attention is paid to its behaviour on steady-state equations
with infinitesimal generator matrices and other linear systems arising
from \textls{CTMC} analysis.

\textls{IDR}($s$)\textls{STAB}($\ell$) merges two generalizations of
\textls{BiCGSTAB}:

\begin{itemize}
\item The first idea comes from \textls{IDR}($s$)%
  ~\citep{sonneveld2008idr}, a Krylov subspace solver based on
  Sonneveld subspaces. A block version of
  \cref{eq:algorithms:bicgstab:sonneveld} constraints the residual
  $\vec{r}_k$
  \begin{equation}
    \vec{r}_k \in \mathcal{G}_k = \mathcal{S}(\Omega_k, A, \widetilde{R}_0) =
    \{ \vec{v} \cdot \Omega_k(A) : \vec{v} \perp \mathcal{K}_k(A,
    \tilde{R}_0) \}\text,
  \end{equation}
  where $\mathcal{K}_k(A, \tilde{R}_0)$ is the $k$th \emph{row} Krylov
  subspace of $A \in \RR^{n \times n}$ with respect to
  $\tilde{R}_0 \in \RR^{s \times n}$
  \begin{equation}
    \mathcal{K}_k(A, \widetilde{R}_0) = \Span \{ \tilde{\vec{r}}_0[i],
    \tilde{\vec{r}}_0[i] A, \ldots, \tilde{\vec{r}}_0[i] A^{k - 1} : i =
    0, 1, \ldots, s - 1 \}
  \end{equation}
  and $\tilde{\vec{r}}_0[i]$ is the $i$th row $\tilde{R}_0$.

  Higher values of $s$, i.e.~higher dimensional initial shadow spaces,
  may accelerate convergence, at the cost of allocation additional
  intermediate vectors.

\item The second generalization, which is called
  \textls{BiCGSTAB}($\ell$)~\citep{sleijpen1993bicgstab}, replaces the
  stabilizer polynomial $\Omega_k$ from
  \cref{eq:algorithms:bicgstab:stabpoly} with
  \begin{equation}
    \Omega_k(A) = \Omega_{k - 1}(A) \cdot (I - \gamma[0] A - \gamma[1] A^2
    - \cdots - \gamma[\ell - 1] A^{\ell}) \text,
  \end{equation}
  i.e.~degree of the stabilizer polynomial $\Omega$ increases by $\ell$
  instead of $1$ every iteration. The increase is described by the
  vector $\vecarrow{\gamma} \in \RR^\ell$.

  The higher-order stabilization, also called a \textls{GMRES}($\ell$)
  step, improves convergence behavior with unsymmetrix matrices that
  have complex spectrum. However, the number of intermediate vectors,
  thus the amount of required memory, also grows.
\end{itemize}

A single dimensional initial shadow space ($s = 1$) and first-order
stabilization ($\ell = 1$) make
\textls{IDR}($s$)\textls{STAB}($\ell$) identical to
\textls{BiCGSTAB}. Moreover, $\ell = 1$ results is behavior equivalent
to \textls{IDR}($s$), while $s = 1$ results in behavior equivalent to
\textls{BiCGSTAB}($\ell$).

These correspondences make \textls{IDR}($s$)\textls{STAB}($\ell$) a
promising candidate for use in configurable stochastic analysis, as
different settings of $(s, \ell)$ bring the power of multiple
algorithms to the modelers' disposal.

\paragraph{Implementation}

\begin{algorithm}
  \KwIn{matrix $A \in \RR^{n \times n}$, right vector $\vec{b} \in
    \RR^n$, initial guess $\vec{x} \in \RR^n$, tolerance $\tau > 0$}
  \KwOut{approximate solution of $\vec{x}A = \vec{b}$}
  \KwAllocate{$\vec{R} \in \RR^{(\ell + 1) \times n}, \vec{U}, \vec{V}
    \in \RR^{((\ell + 2) \times s) \times n}, \widetilde{\vec{R}}_0 \in \RR^{s
    \times n}$}\;
  \KwAllocate{$\largesigma \in \RR^{s \times s}, \vecarrow{m} \in
    \RR^s, \vecarrow{\alpha} \in \RR^s, \vecarrow{\beta} \in \RR^s, G \in
    \RR^{\ell \times \ell}, \vecarrow{\rho} \in \RR^\ell,
    \vecarrow{\gamma} \in \RR^\ell$}\;
  \HlNl{
    \KwAllocate{$\vec{C}, \vec{D} \in \RR^{s \times n}$}\;
  }
  \AlgSection{Initialize shadow residuals}
  \For{$j = 0$ \KwTo $s - 1$}{
    Sample $\tilde{\vec{r}}_0[j]$ from an $n$-dimensional standard
    normal distribution\;
    \HlNl{
      $\tilde{\vec{r}}_0[j] \gets \tilde{\vec{r}}_0[j] + \mleft( - \sum_{k
        = 0}^{n - 1} \tilde{r}_0[j][k] / n \mright) \, \vec{1}$
      \label{nl:algorithms:idrstab:init-normalize}
    }
    \Operation*{\InPlace VectorAdd}
    \For{$q = 0$ \KwTo $s - 1$}{%
      $\tilde{\vec{r}}_0[j] \gets \tilde{\vec{r}}_0[j] +
      (-\tilde{\vec{r}}_0[j] \cdot \tilde{\vec{r}}_0[q])
      \, \tilde{\vec{r}}_0[q]$
      \Operation*{\InPlace VectorAdd}
    }
    $\tilde{\vec{r}}_0[j] \gets (1 / \| \tilde{\vec{r}}_0[j] \|_2 )
    \, \tilde{\vec{r}}_0[j]$
    \Operation*{\InPlace VectorScale}
  }
  \AlgSection{Initialize residuals and intermediate vectors}
  $\vec{r}[0] \gets \vec{b}$,
  $\vec{r}[0] \gets \vec{r}[0] + (-1) \, \vec{x} A$
  \Operation*{VectorMatrixAccumulateMultiplyFromLeft}
  \For{$q \gets 0$ \KwTo $s - 1$}{%
    \HlNl{
      \lIf(\Operation*[f]{VectorSet}){$q = 0$}{%
        $\vec{c}[0] \gets (-1) \, \vec{x}$,
        $\vec{u}[0, 0] \gets \vec{r}[0]$%
      }
      \lElse(\Operation*[f]{VectorSet}){%
        $\vec{c}[q] \gets \vec{u}[0, q - 1]$,
        $\vec{u}[0, q] \gets \vec{u}[1, q - 1]$%
      }
    }
    $U[1, q] \gets U[0, q] \, A$
    \Operation*{VectorMatrixMultiplyFromLeft}
    \For{$k \gets 0$ \KwTo $q - 1$}{
      $\textit{proj} \gets \vec{u}[0, k] \cdot \vec{u}[0, q]$
      \Operation*{VectorScalarProduct}
      \HlNl{
        $\vec{c}[q] \gets \vec{c}[q] + (-\textit{proj}) \, \vec{c}[k]$
        \Operation*{\InPlace VectorAdd}
      }
      $\vec{u}[0, q] \gets U[0, q] + (-\textit{proj}) \, \vec{u}[0, k]$,
      $\vec{u}[1, q] \gets U[1, q] + (-\textit{proj}) \, \vec{u}[1, k]$\;
    }
    $\textit{norm} \gets \| \vec{u}[0, q] \|_2$
    \Operation*{VectorScalarProduct}
    \HlNl{
      $\vec{c}[q] \gets (1 / \textit{norm}) \, \vec{c}[q]$,
      \Operation*{\InPlace VectorScale}
    }
    $\vec{u}[0, q] \gets (1 / \textit{norm}) \, \vec{u}[0, q]$,
    $\vec{u}[1, q] \gets (1 / \textit{norm}) \, \vec{u}[1, q]$\;
  }
  \AlgSection{Iteration}
  \While{$\| \vec{r}[0] \| > \epsilon$}{
    Perform \textls{IDR} step from \cref{alg:algorithms:idrstab:idr}\;
    Perform \textls{GMRES}($\ell$) step from
    \cref{alg:algorithms:idrstab:gmres}\;
  }
  \KwRet{$\vec{x}$}
  \caption{\textls{IDR}($s$)\textls{STAB}($\ell$).}
  \label{alg:algorithms:idrstab:idrstab}
\end{algorithm}

\begin{algorithm}
  \For(\tcp*[f]{For every repetition step}){$j \gets 1$ \KwTo $\ell$}{
    %\AlgSection{Caclulate the new residuals}
    \For{$k \gets 0$ \KwTo $s - 1$}{
      \lFor(\Operation*[f]{VectorScalarProduct}){$q \gets 0$ \KwTo $s - 1$}{%
        $\sigma[q, k] \gets \vec{u}[j, q] \cdot \tilde{\vec{r}}_0[k]$%
      }
      $m[k] \gets \vec{r}[j - 1] \cdot \tilde{\vec{r}}_0[k]$
      \Operation*{VectorScalarProduct}
    }
    Solve $\vecarrow{\alpha} \largesigma = \vecarrow{m}$ for
    $\vecarrow{\alpha}$\label{nl:algorithms:idrstab:solve-sigma}\;
    \For{$q \gets 0$ \KwTo $s - 1$}{
      $\vec{x} \gets \vec{x} + \alpha[q] \, \vec{u}[0, q]$
      \Operation*{\InPlace VectorAdd}
      \lFor{$k \gets 0$ \KwTo $j - 1$}{%
        $\vec{r}[k] \gets \vec{r}[k] + (-\alpha[q]) \, \vec{u}[k +
        1][q]$%
      }
    }
    $\vec{r}[j - 1] \gets \vec{r} \, A$
    \Operation*{VectorMatrixMultiplyFromLeft}
    \For(\tcp*[f]{Build a new basis for the shadow space}){%
      $q \gets 0$ \KwTo $s - 1$}{
      %\AlgSection{Calculate the next basis vector by projection}
      \eIf{$q = 0$}{
        \HlNl{
          $\vec{d}[0] \gets (-1) \, \vec{x}$
          \Operation*{VectorScale}
        }
        \lFor(\Operation*[f]{VectorSet}){$k \gets 0$ \KwTo $j - 1$}{%
          $\vec{v}[k, 0] \gets \vec{r}[k]$%
        }
      }{
        \HlNl{
          $\vec{d}[q] \gets \vec{v}[0, q - 1]$
          \Operation*{VectorSet}
        }
        \lFor(\Operation*[f]{VectorSet}){$k \gets 0$ \KwTo $j - 1$}{%
          $\vec{v}[k, q] \gets \vec{v}[k + 1, q - 1]$%
        }
      }
      \lFor(\Operation*[f]{VectorScalarProduct}){%
        $k \gets 0$ \KwTo $s - 1$}{%
        $m[k] \gets \vec{v}[j, q] \cdot \tilde{\vec{r}}_0[k]$%
      }
      Solve $\vecarrow{\beta} \largesigma = \vecarrow{m}$ for
      $\vecarrow{\beta}$\;
      \For{$i \gets 0$ \KwTo $s - 1$}{%
        \HlNl{
          $\vec{d}[q] \gets \vec{d}[q] + (-\beta[i]) \, \vec{c}[i]$
          \Operation*{\InPlace VectorAdd}
        }
        \lFor{$k \gets 0$ \KwTo $j$}{%
          $\vec{v}[k, q] \gets \vec{v}[k, q] + (-\beta[i]) \,
          \vec{u}[k, i]$%
        }
      }
      $\vec{v}[j + 1, q] \gets \vec{v}[j, q] \, A$
      \Operation*{VectorMatrixMultiplyFromLeft}
      %\AlgSection{Gram--Schmidt orthonormalization}
      \For(\tcp*[f]{Attempt orthonormalization}){%
        $i \gets 0$ \KwTo $q - 1$}{
        \label{nl:algorithms:idrstab:idr:gs-start}
        $\textit{proj} \gets \vec{v}[j, q] \cdot \vec{v}[j, i]$
        \Operation*{VectorScalarProduct}
        \HlNl{
          $\vec{d}[q] \gets \vec{d}[q] + (-\textit{proj}) \, \vec{d}[i]$
          \Operation*{\InPlace VectorAdd}
        }
        \lFor{$k \gets 0$ \KwTo $j + 1$}{%
          $\vec{v}[k, q] \gets \vec{v}[k, q] + (-\textit{proj}) \, \vec{v}[k, i]$%
        }
      }
      $\textit{norm} \gets \| \vec{v}[j, q] \|_2$
      \Operation*{VectorScalarProduct}
      \HlNl{
        \If(\tcp*[f]{Gram--Schmidt breakdown}){%
          $\textit{norm} < \epsilon$}{
          \KwMessage{early exit with $\vec{v}[j, q] \approx \vec{0}$}\;
          $\textit{sum} \gets \sum_{k = 0}^{n - 1}{d[q][k]}$,
          $\vec{x} \gets (1 / \textit{sum}) \, d[q][k]$
          \Operation*{\InPlace VectorScale}
          \KwRet{$\vec{x}$}\;
        }
        $\vec{d}[q] \gets (1 / \textit{norm}) \, \vec{d}[q]$
        \Operation*{\InPlace VectorScale}
      }
      \lFor(\Operation*[f]{\InPlace VectorScale}){%
        $k \gets 0$ \KwTo $j + 1$}{%
        $\vec{v}[k, q] \gets (1 / \textit{norm}) \, \vec{v}[k, q]$%
        \label{nl:algorithms:idrstab:idr:gs-end}%
      }
    }
    \HlNl{
      Swap the references to $\vec{C}$ and $\vec{D}$\;
    }
    Swap the references to $\vec{U}$ and $\vec{V}$\;
  }
  \caption{\textls{IDR}($s$)\textls{STAB}($\ell$) \textls{IDR} step.}
  \label{alg:algorithms:idrstab:idr}
\end{algorithm}

\begin{algorithm}
  \AlgSection{Find $\argmin_{\vecarrow{\gamma} \in \RR^\ell} \|
    \vec{r}[0] - R[1:\ell] \vecarrow{\gamma}^\T\|_2$ by solving the
    normal equation}
  \For{$j \gets 1$ \KwTo $s$}{
    \lFor(\Operation*[f]{VectorScalarProduct}){$i \gets 1$ \KwTo $s$}{%
      $g[i, j] \gets \vec{r}[i] \cdot \vec{r}[j]$%
    }
    $\rho[j] \gets \vec{r}[0] \cdot \vec{r}[j]$
    \Operation*{VectorScalarProduct}
  }
  Solve $\vecarrow{\gamma} G = \vecarrow{\rho}$ for
  $\vecarrow{\gamma}$\;
  \AlgSection{Calculate the minimal residual}
  \For{$j \gets 0$ \KwTo $j - 1$}{
    $\vec{x} \gets \vec{x} + \gamma[j] \, \vec{r}[j]$
    \Operation*{\InPlace VectorAdd}
    $\vec{r}[0] \gets \vec{r}[0] + (-\gamma[j]) \, \vec{r}[j + 1]$
    \Operation*{\InPlace VectorAdd}
    \For{$q \gets 0$ \KwTo $s - 1$}{
      \HlNl{
        $\vec{c}[q] \gets \vec{c}[q] + (-\gamma[j]) \, \vec{u}[j, q]$
        \Operation*{\InPlace VectorAdd}
      }
      $\vec{u}[j, q] \gets \vec{u}[j, q] + (-\gamma[j]) \, \vec{u}[j + 1, q]$
      \Operation*{\InPlace VectorAdd}
      $\vec{u}[j + 1, q] \gets \vec{u}[j + 1, q] + (-\gamma[j]) \,
      \vec{u}[j + 2, q]$
      \Operation*{\InPlace VectorAdd}
    }
  }
  \caption{\textls{IDR}($s$)\textls{STAB}($\ell$)
    \textls{GMRES}($\ell$) step.}
  \label{alg:algorithms:idrstab:gmres}
\end{algorithm}

The pseudocode of our implementation of
\textls{IDR}($s$)\textls{STAB}($\ell$), which is based on the
pseudocode of \citet{sleijpen2010exploiting}, is show in
\cref{alg:algorithms:idrstab:idrstab,alg:algorithms:idrstab:idr,%
alg:algorithms:idrstab:gmres}. Modification to the algorithm to obtain
better convergence properties with \textls{CTMC} steady-state analysis
are highlighed with \HlNode{shaded} line numbers.

For convenient representation of memory requirements, we employ two
different typographical styles for vectors. Vectors in bold,
e.g.~$\vec{x} \in \RR^n$ are ``long'' vectors, while vectors with
arrows, e.g.~$\vecarrow{\gamma}$ are ``short'' vectors of length $s$
or $\ell \ll n$. Storage space of long vectors dominated memory
requirements and their manipulations including vector--matrix products
dominate computation time.

The algorithm works with three arrays of vectors,
$\vec{R} \in \RR^{(\ell + 1) \times n}$,
$\vec{U}, \vec{V} \in \RR^{((\ell + 2) \times s) \times n}$,
i.e.~$\vec{r}[j'], \vec{u}[j, q], \vec{v}[j, q] \in \RR^n$ for all
$j' = 0, 1, \ldots, \ell$; $j = 0, 1, \ldots, \ell + 1$;
$q = 0, 1, \ldots, s - 1$.

\newbox\IdrStabProjectionsBox
\savebox\IdrStabProjectionsBox{\parbox{\linewidth}%
  {\small\begin{equation}
      \newcommand*{\rightproj}[1]{\overset{\smash{\clap{%
              \raisebox{3pt}{$\mathsurround=0pt\Pi_{#1}$}}}}\rightarrow}%
      \newcommand*{\neproj}[1]{{\nearrow}\mathrlap{\,\Pi_{#1}}}%
      \newcommand*{\mulA}{{\downarrow}\mathrlap{\,A}}%
      \newcommand*{\noMulA}{{\color{black!50}\mulA}}%
      \newcommand*{\noEquation}{{\color{black!50}\downarrow}}%
      \begin{array}{*{7}{c}@{\,\,}c@{\,\,}c}
        \vec{x}_{-} & \rightarrow & \vec{x} & & \vec{v}[0,0] & & \vec{v}[0,1] & \cdots & \vec{v}[0, s - 1] \text, \\
                    & & \noEquation & \neproj{0} & \noMulA & \neproj{0} & \noMulA & & \noMulA \\
        \vec{r}_{-}[0] & \rightproj{1} & \vec{r}[0] & & \vec{v}[1,0] & & \vec{v}[1,1] & \cdots & \vec{v}[1, s - 1] \mathrlap{\text,} \\
                    & & \noMulA & \neproj{1} & \noMulA & \neproj{1} & \noMulA & & \noMulA \\
        \vec{r}_{-}[1] & \rightproj{2} & \vec{r}[1] & & \vec{v}[2,0] & & \vec{v}[2,1] & \cdots & \vec{v}[2, s - 1] \mathrlap{\text,}, \\
        \vdots & & \vdots & & \vdots & & \vdots & & \vdots \\
        \vec{r}_{-}[j - 2] & \rightproj{j - 1} & \vec{r}[j - 2] & & \vec{v}[j - 1,0] & & \vec{v}[j - 1,1] & \cdots & \vec{v}[j - 1, s - 1] \mathrlap{\text,} \\
                    & & \noMulA & \neproj{j - 1} & \noMulA & \neproj{j - 1} & \noMulA & & \noMulA \\
        \vec{r}_{-}[j - 1] & \rightproj{j} & \vec{r}[j - 1] & & \vec{v}[j,0] & & \vec{v}[j,1] & \cdots & \vec{v}[j, s - 1] \mathrlap{\text,} \\
                    & & \mulA & \neproj{j} & \mulA & \neproj{j} & \mulA & & \mulA \\
                    & & \boxed{\vec{r}[j]} & & \boxed{\vec{v}[j + 1,0]} & & \boxed{\vec{v}[j + 1,1]} & \cdots & \boxed{\vec{v}[j + 1, s - 1]}
      \end{array}\label{eq:algorithms:idrstab:projections}
    \end{equation}}}%

\begin{figure}
  \centering\usebox\IdrStabProjectionsBox
  \caption{Repetition step of the \textls{IDR}($s$) part of \textls{IDR}($s$)\textls{STAB}($\ell$).}
  \label{fig:algorithms:idrstab:projections}
\end{figure}

The \textls{IDR} part performs the projections, called a ``repetition
step'', shown in \cref{fig:algorithms:idrstab:projections}, for
$j = 1, 2, \ldots, \ell$ every iteration.

After a repetition step is complete, $\vec{U}$ and $\vec{V}$ are
swapped and the process starts again with increased $j$ or the
\textls{GMRES}($\ell$) part commences. Symbols with a subscript
``$-$'' sign refer to vectors from the previous repetition step.

The projections $\Pi_i$ ($i = 0, 1, \ldots, j$) are defined as
\begin{equation}
  \label{eq:algorithms:idrstab:projection}
  \Pi_i = I - A^{j - i} \widetilde{R}_0^\T \largesigma^{-1} U[i, \cdot],
  \quad \largesigma = \widetilde{R}_0 (U[j, \cdot])^\T \text,
\end{equation}
where the matrix $U[k, \cdot]$ is the $s \times n$ matrix that has the
vector $\vec{u}[k, q]$ as its $q$th row. They ensure that the rows of
$U[j, \cdot]$ form a basis of the Krylov subspace
$\mathcal{K}_s(A \Pi_j, \vec{r}[j] \Pi_j)$ after the $j$th repetition
step.

The relationships $\vec{r}[i + 1] = \vec{r}{i} A$,
$\vec{u}[i + 1, q] = \vec{u}[i, q] A$,
$\vec{v}[i + 1, q] = \vec{v}[i, q]$ are maintained throughout the
algorithm via the projections. This is signified by the gray
{\color{black!50}$\downarrow A$} arrows in
\cref{eq:algorithms:idrstab:projections}. Notice that this means
$\vec{r}[0] A^i = \vec{r}[i]$ and $U[0, \cdot] A^i = U[i, \cdot]$.

In the case of $\vec{r}[j]$ and $\vec{v}[j + 1, q]$, a matrix
multiplication is performed as shown by the $\downarrow A$
arrows. Vectors generated by matrix multiplication are shown in
borders.

For improving numerical properties, \citet{sleijpen2010exploiting}
recommend performing Gram--Schmidt ortonormalization on $U[j, \cdot]$.
The same subtractions and normalization operations must be performed
on the rows of $U[i, \cdot]$, $i \ne j$ that are performed in
$U[j, \cdot]$ in order to maintain their relationships. We realized
the orthonormalization by the modified Gram--Schmidt process in lines%
~\ref{nl:algorithms:idrstab:idr:gs-start}%
--\ref{nl:algorithms:idrstab:idr:gs-end} of
\cref{alg:algorithms:idrstab:idr}.

The storage of $\vec{R}$, $\vec{U}$ and $\vec{V}$ requires
$(\ell + 1) + 2 \cdot (\ell + 1) \cdot s$ vectors of length $n$, while
the initial shadow residual matrix $\widetilde{R}_0$ requires space
equal to $s$ vectors of length $n$. Thus, $1 + \ell + 3s + 2\ell s$
intermediate vectors are needed in addition to the initial guess
$\vec{x}_0$ and the right vectors $\vec{b}$. Although the memory
requiremens of \textls{IDR}($s$)\textls{STAB}($\ell$) are quite high,
$s$ and $\ell$ can be selected to ensure that the solution fits in the
available memory.

A different formulation of the \textls{IDR}($s$)\textls{STAB}($\ell$)
principles is \textls{GBi-CGSTAB}($s, \ell$)~\citep{tanio2010gbi},
which avoids the allocation of $\vec{V}$ by updating $\vec{U}$ in
place, albeit with lesser numerical properties due to the lack of
orthonormalization steps. Another variant by
\textcite{aihara2014variant} replaces some vector updates with matrix
multiplications to improve accuracy.

\paragraph{Numerical problems and breakdown}

Unfortunately, our initial experintments with
\textls{IDR}($s$)\textls{STAB}($\ell$) in Markovian steady-state
analysis did not lead to success when one of the parameters $s$ or
$\ell$ were set to values strictly larger than $1$. The only case that
managed to decrease the norm of the residual reliably, $s = \ell = 1$
is equivalent to \textls{BiCGSTAB}, but due to properties of the
\textls{IDR} formulation,

In steady-state analysis, equations of the form
\begin{equation}
  \label{eq:algorithms:idrstab:equation}
  \vec{x} Q = \vec{0}, \quad \vec{x} \vec{1}^\T = 1
\end{equation}
are solved, where the infinitesimal generator matrix
$Q \in \RR^{n \times n}$ is matrix of rank $n - 1$ (nullity $1$) which
satisfies $Q \vec{1} = \vec{0}^\T$. To obtain the solution, the matrix
$A = Q$ is passed to \textls{IDR}($s$)\textls{STAB}($\ell$) along with
the right vector $\vec{b} = 0$ and the initial guess $\vec{x}$. The
initial guess is chosen to satify $\vec{x} \vec{1}^\T = 1$.

We have identified the following three numerical problems in the
original version of the algorithm algorithm that lead to eventual
breakdown:
\begin{enumerate}
\item The quantity $\vec{x} \vec{1}^\T$ may increse indefinitely
  instead of staying constant. Thus, the normalization condition in
  \cref{eq:algorithms:idrstab:equation} is violated. Unconstrained
  increase results in the loss of precision and overflow of the
  elements of $\vec{x}$ such that the normalization cannot be restored
  by the division $\hat{\vec{x}} = \vec{x} / \vec{x} \vec{1}^\T$.
\item The vectors $\vec{V}[j, \cdot]$ that shall form the basis of the
  Krylov subspace $\mathcal{K}_s(A \Pi_j, \vec{r}[j] \Pi_j)$ may
  become linearly dependent, such that the Gram--Schmidt process in
  lines~\ref{nl:algorithms:idrstab:idr:gs-start}%
  --\ref{nl:algorithms:idrstab:idr:gs-end} of
  \cref{alg:algorithms:idrstab:idr} fail due to some $\vec{v}[j, q]$
  becoming the zero vector.
\item The matrix $\largesigma$ may become singular such that the
  linear equation $\vecarrow{\alpha} \largesigma = \vecarrow{m}$ has
  no solution in line~\ref{nl:algorithms:idrstab:solve-sigma} of
  \cref{alg:algorithms:idrstab:idr}. Since
  $\largesigma = U[j, \cdot] \widetilde{R}_0^\T$, this corresponds the
  projection of the vectors $\vec{U}[j, \cdot]$ into
  $\Span \widetilde{\vec{R}}_0$ becoming linearly dependent.

  In addition, a nearly singular $\largesigma$ results in the
  accumulation of errors that cause the norm of the residual to
  increase exponentially instead of converging to zero.
\end{enumerate}

\paragraph{Handling numerical problems}

Is this section, we present the modification made to
\textls{IDR}($s$)\textls{STAB}($\ell$) in
\cref{alg:algorithms:idrstab:idrstab,alg:algorithms:idrstab:idr,%
  alg:algorithms:idrstab:gmres} to improve its behavior with
\textls{CTMC} generator matrices. We restrict our attention to the
solution of equations of the form
\eqref{eq:algorithms:idrstab:equation}, that is, steady-state
normalized solution of Markovian models with zero right vector. Thus,
we will write $Q$ instead of $A$ for the linear equation matrix and
assume a right hand side $\vec{b} = \vec{0}$.

\begin{obs}
  \label{obs:algorithms:idrstab:normalized-update}
  \textls{IDR}($s$)\textls{STAB}($\ell$) only performs updates
  of the approximate solution of the form $\vec{x} \gets \vec{x} +
  \lambda \vec{t}$, where
  \begin{asparaenum}
  \item $\vec{t}$ is either an initial shadow
    residual $\tilde{\vec{r}}_0[j]$ for some $j$
  \item or there exists a vector $\vec{w}$ such that
    $\vec{t} = \vec{w} Q$.
  \end{asparaenum}
\end{obs}

In the second case of \cref{obs:algorithms:idrstab:normalized-update},
one may notice that
\begin{equation}
  \vec{t} \vec{1}^\T = \vec{w} Q \vec{1}^\T = \vec{w} \vec{0}^\T = 0
  \text,
\end{equation}
therefore $\vec{x} \vec{1}^\T$ can be forced to stay constant by
selecting initial shadow residual vectors
$\widetilde{R}_0 \vec{1}^\T =
\vec{0}^\T$.
Line~\ref{nl:algorithms:idrstab:init-normalize} of
\cref{alg:algorithms:idrstab:idrstab} contains this modification. The
initialization of the shadow residuals is otherwise identical to the
recommendation of \citet{sonneveld2010convergence}, which selects a
random orthonormalized set of $s$ vectors.

The handling of the second problem of the Gram--Schmidt process
failure is more compilcated.

\begin{obs}
  \label{obs:algorithms:idrstab:normalized-v}
  If $Q \in \RR^n$ is a rank $n - 1$ matrix such that
  $Q \vec{1}^\T = \vec{0}^\T$ and $\vec{b} = \vec{0}$, then
  $\vec{v}[i, q] \, \vec{1}^\T = 0$ for all $i, q$.
\end{obs}

\begin{proof}
  Because $\vec{v}[i, q] = \vec{v}[i - 1, q] \, Q$ for all $i > 0$, it
  suffices to show that $\vec{v}[0, q] \, \vec{1}^\T = 0$. The chain
  of projections in \vref{eq:algorithms:idrstab:projections} shows
  that
  \begin{equation}
    \vec{v}[0, q] = \vec{w} \Pi_0 = \vec{w} ( I - Q^{j}
    \widetilde{R}_0^\T \largesigma^{-1} U[0, \cdot] ) \text,
  \end{equation} where $\vec{w}$ is either $\vec{r}[0]$ or
  $\vec{v}[1, q - 1]$.

  Notice that $\vec{v}[1, q - 1] \, \vec{1}^\T = 0$ as we have shown
  before and $\vec{r}[0] \, \vec{1}^\T = -\vec{x} Q \vec{1}^\T =
  0$. Hence if $\vec{u}[0, k] \, \vec{1}^\T = \vec{0}$ for all $k$,
  \begin{align}
    \vec{v}[0, q] = \vec{w} \Pi_0 \vec{1}^\T &= \vec{w} \vec{1}^\T - \vec{w} Q^{j}
    \widetilde{R}_0^\T \largesigma^{-1} U[0, \cdot] \, \vec{1}^\T \\
    & =\vec{w} \vec{1}^\T - \vec{w} Q^{j}
    \widetilde{R}_0^\T \largesigma^{-1} \vec{0}^\T = 0 \text.
  \end{align}

  Now it all remains to be shown that
  $\vec{u}[0, k] \, \vec{1}^\T = 0$. We will use induction over the
  number of repetitions performed. If $\vec{v}[0, k] \, \vec{1}^\T$
  holds in the current repetition step,
  $\vec{u}[0, k] \, \vec{1}^\T = 0$ holds in the next, because the
  references to $\vec{U}$ and $\vec{V}$ are swapped every repetition
  step. This property is not distrurbed by the performed Gram--Schmidt
  orthonormalizations and the \textls{GMRES}($\ell$) steps.

  We also notice that \vref{alg:algorithms:idrstab:idrstab}
  initializes $\vec{u}[0, 0] = \vec{r}[0] = -\vec{x} Q$ and
  $\vec{u}[0, q] = \vec{u}[0, q - 1] \, Q$ for $q > 0$. This completes
  the induction.
\end{proof}

\begin{obs}
  \label{obs:algorithms:idrstab:vanishing-v}
  If the conditions from \cref{obs:algorithms:idrstab:normalized-v}
  and $\vec{v}[j, q] = \vec{0}$ hold, then $\vec{v}[i, q] = \vec{0}$
  for all $i < j$.
\end{obs}

\begin{proof}
  Suppose that there is some $\vec{v}[i, q] \ne \vec{0}$ such that $i
  < j$. Without loss of generality, we may assume that $i$ is the
  largest index with this property, i.e.~$\vec{v}[i', q] = \vec{0}$
  for all $i' > i$. Then $\vec{0} = \vec{v}[i + 1, q] = \vec{v}[i, q]
  \, Q$, therefore $\vec{v}[i, q]$ is the solution of the linear
  equation $\vec{x} Q = \vec{0}$.

  However, we know that the nullspace
  $\ker Q = \Span \{ \vec{\uppi} \}$, where $\vec{\uppi}$ is the
  stationary distribution of the \textls{CTMC} of which $Q$ is the
  infinitesimal generator, i.e~$\vec{\uppi} Q = \vec{0}$,
  $\vec{\uppi} \vec{1}^\T = 1$. Therefore,
  $\vec{v}[i, q] \in \Span \{ \vec{\uppi} \}$ such that
  $\vec{v}[i, q] \, \vec{1}^\T = 0$
  (\cref{obs:algorithms:idrstab:normalized-v}). This means
  $\vec{v}[i, q] = \frac{0}{1} \cdot \vec{\uppi} = \vec{0}$.
\end{proof}

\cref{obs:algorithms:idrstab:vanishing-v} means that if the
Gram--Schmidt process breaks down with $\vec{v}[j, q] = \vec{0}$, we
are unable to recover a solution vector by looking at some $\vec{v}[i,
q]$, $i < j$, because those vector are also zero.

\savebox\IdrStabProjectionsBox{\parbox{\linewidth}%
  {\small\begin{equation}
      \newcommand*{\rightproj}[1]{\overset{\smash{\clap{%
              \raisebox{3pt}{$\mathsurround=0pt\Pi_{#1}$}}}}\rightarrow}%
      \newcommand*{\neproj}[1]{{\nearrow}\mathrlap{\,\Pi_{#1}}}%
      \newcommand*{\mulA}{{\downarrow}\mathrlap{\,Q}}%
      \newcommand*{\noMulA}{{\color{black!50}\mulA}}%
      \newcommand*{\noMulMinA}{{\color{black!50}{\downarrow}\mathrlap{\,-Q}}}%
      \begin{array}{*{7}{c}@{\,\,}c@{\,\,}c}
        & & & & \vec{d}[0] & & \vec{d}[1] & \cdots & \vec{d}[s - 1] \text, \\
        & &  & \neproj{-1} & \noMulA & \neproj{-1} & \noMulA & & \noMulA \\    
        \vec{x}_{-} & \rightarrow & \vec{x} & & \vec{v}[0,0] & & \vec{v}[0,1] & \cdots & \vec{v}[0, s - 1] \text, \\
        & & \noMulMinA & \neproj{0} & \noMulA & \neproj{0} & \noMulA & & \noMulA \\
        \vec{r}_{-}[0] & \rightproj{1} & \vec{r}[0] & & \vec{v}[1,0] & & \vec{v}[1,1] & \cdots & \vec{v}[1, s - 1] \mathrlap{\text,} \\
        & & \noMulA & \neproj{1} & \noMulA & \neproj{1} & \noMulA & & \noMulA \\
        \vec{r}_{-}[1] & \rightproj{2} & \vec{r}[1] & & \vec{v}[2,0] & & \vec{v}[2,1] & \cdots & \vec{v}[2, s - 1] \mathrlap{\text,}, \\
        \vdots & & \vdots & & \vdots & & \vdots & & \vdots \\
        \vec{r}_{-}[j - 2] & \rightproj{j - 1} & \vec{r}[j - 2] & & \vec{v}[j - 1,0] & & \vec{v}[j - 1,1] & \cdots & \vec{v}[j - 1, s - 1] \mathrlap{\text,} \\
        & & \noMulA & \neproj{j - 1} & \noMulA & \neproj{j - 1} & \noMulA & & \noMulA \\
        \vec{r}_{-}[j - 1] & \rightproj{j} & \vec{r}[j - 1] & & \vec{v}[j,0] & & \vec{v}[j,1] & \cdots & \vec{v}[j, s - 1] \mathrlap{\text,} \\
        & & \mulA & \neproj{j} & \mulA & \neproj{j} & \mulA & & \mulA \\
        & & \boxed{\vec{r}[j]} & & \boxed{\vec{v}[j + 1,0]} & & \boxed{\vec{v}[j + 1,1]} & \cdots & \boxed{\vec{v}[j + 1, s - 1]}
      \end{array}\label{eq:algorithms:idrstab:extended-projections}
    \end{equation}}}%

\begin{figure}
  \centering\usebox\IdrStabProjectionsBox
  \caption{Extended repetition step of the \textls{IDR}($s$) part of \textls{IDR}($s$)\textls{STAB}($\ell$).}
  \label{fig:algorithms:idrstab:extended-projections}
\end{figure}

The projection scheme in \vref{eq:algorithms:idrstab:projections} is
extended with two additional arrays
$\vec{C}, \vec{D} \in \RR^{s \times n}$, which serve as the
``(-1)$th$'' rows of $\vec{U}$ and $\vec{V}$. This results in the
extended projections shown in
\cref{fig:algorithms:idrstab:extended-projections}, where
$\Pi_{-1} = I - A^{j - i} \widetilde{R}_0^\T \largesigma^{-1} D$.  In
order to avoid $\vec{d}[q] = \vec{0}$, $\vec{c}[0]$ is initialized to
$-\vec{x}$ so that $\vec{v}[0, 0] = \vec{r} = \vec{c}[0] Q$.

After a Gram--Schmidt breakdown, the solution vector $\vec{x}$ is
obtained as
\begin{equation}
  \vec{x} = \frac{1}{\vec{d}[q] \, \vec{1}^\T} \vec{d}[q] \text.
\end{equation}

Despite our attempts, we did not handle the third divergence problem
of singular or nearly singular $\largesigma$. A possible remedy is the
more careful choice of the stabilizing polynomial $\Omega_k$. Chosing
$\vecarrow{\gamma}$ through means other than the minimization of the
residual norm $\| \vec{r}[0] \|_2$ may result in better converge
behavior~\citep{rendel2011tuning,sleijpen1995maintaining}.

Empirical results on the convergence of
\textls{IDR}($s$)\textls{STAB}($\ell$) in the steady-state analysis of
Markovian models of various size are presented in%
~\vref{sec:evaulations:idrstab}.

\section{Transient analysis}

\subsection{Uniformization}
\label{ssec:algorithms:uniformization}

The \emph{uniformization} or \emph{randomization} method solves the
initial value problem
\begin{gather}
  \frac{\dd \vec{\uppi}(t)}{\dd t} = \vec{\uppi}(t) \, Q, \quad
  \vec{\uppi}(t) = \vec{\uppi}{0}
  \tag{\ref{eq:background:ctmc:diffeq}~revisited}\\
  \shortintertext{by computing}
  \vec{\uppi}(t) = \sum_{k = 0}^{\infty} \vec{\uppi}_0 P^k e^{-\alpha
    t} \frac{(\alpha t)^k}{k!}\text,
  \label{eq:algorithms:transient:uniformization}
\end{gather}
where $P = \alpha^{-1} Q + I$,
$\alpha \ge \max_{i} \lvert a[i, i] \rvert$ and
$e^{-\alpha t} \frac{(\alpha t)^k}{k!}$ is the value of the Poisson
probabilty function with rate $\alpha t$ at $k$.

Integrating both sides of
\vref{eq:algorithms:transient:uniformization} to compute $\vec{L}(t)$
yields~\citep{reibman1989markov}
\begin{align}
  \int_{0}^t \vec{\uppi}(u) \,\dd u = \vec{L}(t)
  &= \sum_{k = 0}^{\infty} \vec{\uppi}_0 P^k \int_{0}^{t} e^{-\alpha
    u} \frac{(\alpha u)^k}{k!} \,\dd u \\
  &= \sum_{k = 0}^{\infty} \vec{\uppi}_0 P^k \frac{1}{\alpha} \sum_{l
    = k + 1}^{\infty} e^{-\alpha t} \frac{(\alpha t)^l}{l!} \\
  &= \frac{1}{\alpha} \sum_{k = 0}^{\infty} \vec{\uppi}_0 P^k \mleft(
    1 - \sum_{l = 0}^{k} e^{-\alpha t} \frac{(\alpha t)^l}{l!}
    \mright) \text.
    \label{eq:algorithms:transient:aggregate-uniformization}
\end{align}

Both \cref{eq:algorithms:transient:uniformization,%
eq:algorithms:transient:aggregate-uniformization} can be realized
as
\begin{equation}
  \vec{x} = \frac{1}{W} \mleft(\; \sum_{k = 0}^{k_{\text{left}} - 1} w_{\text{left}}
  \vec{\uppi}_0 P^k + \sum_{k = k_{\text{left}}}^{k_{\text{right}}}
  w[k - k_{\text{left}}] \vec{\uppi}_0 P^k \!\mright) \text,
  \label{eq:algorithms:transient:uniformization-impl}
\end{equation}
where $\vec{x}$ is either $\vec{\uppi}(t)$ or $\vec{L}(t)$,
$k_{\text{left}}$ and $k_{\text{right}}$ are \emph{trimming constants}
selected based on the required precision, $\vec{w}$ is a vector of
(possibly accumulated) Poisson weights and $W$ is a scaling
factor. The weight before the left cutoff $w_{\text{left}}$ is $1$ if
the accumulated probability vector $\vec{L}(t)$ is calculated, $0$
otherwise.

\begin{algorithm}
  \KwIn{infinitesimal generator $Q \in \RR^{n \times n}$, initial
    probability vector $\vec{\uppi}_0 \in \RR^n$, truncation
    parameters $k_{\text{left}}, k_{\text{right}} \in \NN$, weights
    $w_{\text{left}} \in \RR$,
    $\vec{w} \in \RR^{k_{\text{right}} - k_{\text{left}}}$, scaling
    constant $W \in \RR$, tolerance $\tau > 0$}
  \KwOut{instantenous or accumulated probability vector $\vec{x} \in
    \RR^n$}
  \KwAllocate{$\vec{x}, \vec{p}, \vec{q} \in \RR^n$}\;
  $\alpha^{-1} \gets 1 / \max_{i} \lvert a[i,i] \rvert$
  \label{ln:algorithms:transient:uniformization:alpha}\;
  $\vec{p} \gets \vec{\uppi}_0$\;
  \leIf(\Operation*[f]{VectorScale})
  {$w_{\text{left}} = 0$}{$\vec{x} \gets \vec{0}$}
  {$\vec{x} \gets w_{\text{left}} \cdot \vec{p}$}
  \For{$k \gets 1$ \KwTo $k_{\text{right}}$}{
    $\vec{q} \gets \vec{p} Q$
    \Operation*{VectorMatrixMultiplyFromLeft}
    $\vec{q} \gets \alpha^{-1} \cdot \vec{q}$
    \Operation*{\InPlace VectorScale}
    $\vec{q} \gets \vec{q} + \vec{p}$
    \Operation*{\InPlace VectorAdd}
    \If{$\| \vec{q} - \vec{p} \| \le \tau$}{
      \label{ln:algorithms:transient:uniformization:steadystate-det}
      $\vec{x} \gets \vec{x} + \mleft( \sum_{l = k}^{k_\text{right}}
      w[l - k_{\text{left}}] \mright) \cdot \vec{q}$
      \Operation*{\InPlace VectorAdd}
      \KwSty{break}\;
    }
    \lIf(\Operation*[f]{\InPlace VectorAdd})
    {$k < k_{\text{left}} \land w_{\text{left}} \ne 0$}{%
      $\vec{x} \gets \vec{x} + w_{\text{left}} \cdot \vec{q}$
    }
    \lElseIf(\Operation*[f]{\InPlace VectorAdd})
    {$k \ge k_{\text{left}}$}{%
      $\vec{x} \gets \vec{x} + w[k - k_{\text{left}}] \cdot \vec{q}$
    }
    Swap the references to $\vec{p}$ and $\vec{q}$
  }
  $\vec{x} \gets W^{-1} \cdot \vec{x}$
  \Operation*{\InPlace VectorScale}
  \KwRet{$\vec{x}$}\;
  \caption{Uniformization.}
  \label{alg:algorithms:transient:uniformization}
\end{algorithm}

\Vref{eq:algorithms:transient:uniformization-impl} is implemented by
\vref{alg:algorithms:transient:uniformization}. The algorithm performs
\emph{steady-state} detection in line%
~\ref{ln:algorithms:transient:uniformization:steadystate-det} to avoid
unneccessary work once the iteration vector $\vec{p}$ reaches the
steady-state distribution $\vec{\uppi}(\infty)$,
i.e.~$\vec{p} \approx \vec{p} P$. If the initial distribution
$\vec{\uppi}_0$ is not further needed or can be generated efficiently
(as it is the case with a single initial state), the result vector
$\vec{x}$ may share the same storing, resulting in a memory overhead
of only two vectors $\vec{p}$ and $\vec{q}$.

\begin{algorithmpage}
  \KwIn{Poisson rate $\lambda = \alpha t$, tolerance $\tau >
    10^{-50}$}
  \KwOut{truncation parameters $k_{\text{left}}, k_{\text{right}} \in
    \NN$, weights $\vec{w} \in \RR^{k_{\text{right}} -
      k_{\text{left}}}$, scaling constant $W \in \RR$}
  \AlgSection{Calculate weights with high precision}
  $M_w \gets 30, M_a \gets 44, M_s \gets 21$
  \tcp*{Constants determine cutoff estimation accuracy}
  $m \gets \lfloor \lambda \rfloor, \textit{tSize} \gets \lfloor
  M_w \sqrt{\lambda} + M_a \rfloor, \textit{tStart} \gets \max
  \{m + M_s - \lfloor \textit{tSize} / 2 \rfloor, 0 \}$\;
  \KwAllocate{$\textbf{tWeights} \in \RR^{\textit{tSize}}$}\;
  $\textit{tWeights}[m - \textit{tStart}] \gets 2^{176}$\;
  \For{$j \gets m - \textit{tStart}$ \KwDownto $1$}{%
    $\textit{tWeights}[j - 1] = (j + \textit{tStart})\,
    \textit{tWeights}[j] / \lambda$
  }
  \For{$j \gets m - \textit{tStart} + 1$ \KwTo $\textit{tSize}$}{%
    $\textit{tWeights}[j + 1] = \lambda\, \textit{tWeights}[j] / (j +
    \textit{tStart})$
  }
  \AlgSection{Determine normalization constant and cutoff points}
  $W \gets 0$\;
  \For{$j \gets 0$ \KwTo $m - \textit{tStart} - 1$}{%
    $W \gets W + \textit{tWeights}[j]$
  }
  $\textit{sum1} \gets 0$
  \tcp*{Avoid adding small numbers to larger numbers}
  \For{$j \gets \textit{tSize} - 1$ \KwDownto $m -
    \textit{tStart}$}{%
    $\textit{sum1} \gets \textit{sum1} + \textit{tWeights}[j]$
  }
  $W \gets W + \textit{sum1}, \textit{threshold} \gets W \tau / 2,
  \textit{cdf} \gets 0, i \gets 0$\;
  \While{$\textit{cdf} < \textit{threshold}$}{%
    $\textit{cdf} \gets \textit{cdf} + \textit{tWeights}[i]$\;
    $i \gets i + 1$
  }
  $k_{\text{left}} \gets \textit{tStart} + i, \textit{cdf} \gets 0, i
  \gets \textit{tSize} - 1$\;
  \While{$\textit{cdf} < \textit{threshold}$}{%
    $\textit{cdf} \gets \textit{cdf} + \textit{tWeights}[i]$\;
    $i \gets i - 1$
  }
  $k_{\text{right}} \gets \textit{tStart} + i$\;
  \AlgSection{Copy weights between cutoff points}
  \KwAllocate{$\vec{w} \in \RR^{k_{\text{right}} -
      k_{\text{left}}}$}\;
  \For{$j \gets k_{\text{left}}$ \KwTo $k_{\text{right}}$}{%
    $w[j - k_{\text{left}}] \gets \textit{tWeights}[j - \textit{tStart}]$
  }
  \KwRet{$k_{\text{left}}, k_{\text{right}}, \vec{w}, W$}\;
  \caption{\citeauthor{DBLP:journals/corr/Burak14}'s algorithm for
    calculating the Poisson weights.}
  \label{alg:algorithms:transient:poisson}
\end{algorithmpage}

The weights and trimming constants may be calculated by the famous
algorithm of \citet{DBLP:journals/cacm/FoxG88}. However, their
algorithm is extremely complicated due to the limitations of
single-precision floating-point arithmetic%
~\citep{jansen2011understanding}. We implemented
\citeauthor{DBLP:journals/corr/Burak14}'s significantly simpler
algorithm~\citep{DBLP:journals/corr/Burak14} in double precision
instead (\vref{alg:algorithms:transient:poisson}), which avoids
underflow by a scaling factor $W \gg 1$.

\subsection{\textls*[35]{TR-BDF2}}
\label{ssec:algorithms:trbdf2}

A weakness of the uniformization algorithm is the poor tolerance of
\emph{stiff} Markov chains. The \CTMC\ is called stiff if the
$\lvert \lambda_{\min} \rvert \ll \lvert \lambda_{\max} \rvert$, where
$\lambda_{\min}$ and $\lambda_{\max}$ are the nonzero eigenvalues of
the infinitesimal generator matrix $Q$ of minimum and maximum absolute
value~\citep{DBLP:journals/cor/ReibmanT88}. In other words, stiff
Markov chains have behaviors on drastically different timescales, for
example, clients are served frequently while failures happen
infrequently.

Stiffness leads to very large values of $\alpha$ in
line~\ref{ln:algorithms:transient:uniformization:alpha} of
\vref{alg:algorithms:transient:uniformization}, thus a large right
cutoff $k_{\text{right}}$ is required for computing the transient
solution with sufficient accuracy. Moreover, the slow stabilization
results in taking many iterations before steady-state detection in
line~\ref{ln:algorithms:transient:uniformization:steadystate-det}.

Some methods that can handle stiff \textls{CTMCs} efficiently are
stochastic complementation~\citep{meyer1989stochastic}, which
decouples the slow and fast behaviors of the system, and adaptive
uniformization~\citep{van1994adaptive}, which varies the uniformization
rate $\alpha$. Alternatively, an $L$-stable differential equation
solver may be used to solve \vref{eq:background:ctmc:diffeq}, such as
\mbox{\textls{TR-BDF2}}~\citep{DBLP:journals/cor/ReibmanT88,%
DBLP:journals/tcad/BankCFGRS85}.

\textls{TR-BDF2} is an implicit integrator with alternating trapezoid
rule (\textls{TR}) steps
\begin{equation}
  \vec{\uppi}_{k + \gamma} (2I + \gamma h_k Q) = 2 \vec{\uppi}_{k} +
  \gamma h_k \vec{\uppi}_k Q
\end{equation}
and second order backward difference steps
\begin{equation}
  \vec{\uppi}_{k + 1} [(2 - \gamma) I - (1 - \gamma) h_k Q] =
  \frac{1}{\gamma} \vec{\uppi}_{k + \gamma} - \frac{(1 -
    \gamma)^2}{\gamma} \vec{\uppi}_k \text,
\end{equation}
which advance the time together by a step of size $h_k$. The constant
$0 < \gamma < 1$ sets the breakpoint between the two steps. We set it
to $\gamma = 2 - \sqrt{2} \approx 0.59$ following the recommendation
of \citet{DBLP:journals/tcad/BankCFGRS85}.

As a guess for the initial step size $h_0$, we chose the
uniformization rate of $Q$. The $k$th step size $h_k > 0$, including
the $0$th one, is selected such that the local error estimate
\begin{equation}
  \label{eq:algorithms:transient:trbdf-lte}
  \textit{\textls{LTE}}_{k + 1} = \mleft\| 2 \frac{-3\gamma^4 + 4\gamma - 2}{24 -
    12\gamma} h_k \mleft[ -\frac{1}{\gamma} \vec{\uppi}_k +
  \frac{1}{\gamma (1 - \gamma)} \vec{\uppi}_{k + \gamma} - \frac{1}{1
    - \gamma} \vec{\uppi}_{k + 1} \mright] \mright\|
\end{equation}
is bounded by the local error tolerance
\begin{equation}
  \textit{\textls{LTE}}_{k + 1} \le \mleft( \frac{ \tau - \sum_{i = 0}^k
    \textit{\textls{LTE}}_{i} }{ t - \sum_{i = 0}^k k_i } \mright) h_{k + 1}
  \text.
\end{equation}
This Local Error per Unit Step (\textls{LEPUS}) error control
\enquote{produces excellent results for many problems}, but is usually
costly~\citep{DBLP:journals/cor/ReibmanT88}. Moreover, the accumulated
error at the end of integration may be larger than the prescribed
tolerance $\tau$, since \vref{eq:algorithms:transient:trbdf-lte} is
only an approximation of the true error.

\begin{algorithmpage}
  \KwIn{infinitesimal generator $Q \in \RR^{n \times n}$, initial
    distribution $\vec{\uppi}_0$,\\mission time $t > 0$, tolerance
    $\tau > 0$}
  \KwOut{transient distribution $\vec{\uppi}(t)$}
  \KwAllocate{$\vec{\uppi}_k, \vec{\uppi}_{k + \gamma}, \vec{\uppi}_{k +
      1}, \vec{d}_k, \vec{d}_{k + 1}, \vec{y} \in \RR^n$}\;
  $\textit{maxIncrease} \gets 10, \textit{leastDecrease} \gets 0.9$\;
  $\textit{timeLeft} \gets t, h \gets 1 / \max_{i} \lvert q[i,i]
  \rvert, \gamma \gets 2 - \sqrt{2}, C \gets \mleft\lvert \frac{-3\gamma^4 +
    4\gamma - 2}{24 - 12\gamma} \mright\rvert, \textit{errorSum} \gets 0$\;
  $\vec{\uppi}_k \gets \vec{\pi}_0$,
  $\vec{d}_k \gets \vec{\uppi}_k Q$
  \Operation*{VectorMatrixMultiplyFromLeft}
  \While{$\textit{timeLeft} > 0$}{%
    $\textit{stepFailed} \gets \KwFalse, h \gets \min \{h,
    \textit{timeLeft}\}$\;
    \While{$\KwTrue$}{
      \AlgSection{\textls[30]{TR} step}
      $\vec{y} \gets 2 \cdot \vec{\uppi}_k$
      \Operation*{VectorScale}
      $\vec{y} \gets \vec{y} + \gamma h \cdot \vec{d}_k$
      \Operation*{\InPlace VectorAdd}
      Solve $\vec{\uppi}_{k + \gamma} (2I + -\gamma h Q) = \vec{y}$
      for $\vec{\uppi}_{k + \gamma}$ with initial guess
      $\vec{\uppi}_k$\label{ln:algorithms:transient:trbdf2:solve1}\;
      \AlgSection{\textls[30]{BDF2} step}
      $\vec{y} \gets -\frac{(1 - \gamma)^2}{\gamma} \cdot
      \vec{\uppi}_k$
      \Operation*{VectorScale}
      $\vec{y} \gets \frac{1}{\gamma} \cdot \vec{\uppi}_{k + \gamma}$
      \Operation*{\InPlace VectorScale}
      Solve $\vec{\uppi}_{k + 1} ( (2 - \gamma) I + (\gamma - 1) h Q ) =
      \vec{y}$ for $\vec{\uppi}_{k + 1}$ with initial guess
      $\vec{\uppi}_{k + \gamma}$\label{ln:algorithms:transient:trbdf2:solve2}\;
      \AlgSection{Error control and step size estimation}
      $\vec{y} \gets -\frac{1}{\gamma} \vec{d}_k$
      \Operation*{VectorScale}
      $\vec{y} \gets \vec{y} + \frac{1}{\gamma (1 - \gamma)}
      \vec{\uppi}_{k + \gamma} Q$
      \Operation*{VectorMatrixAccumulateMultiplyFromLeft}
      $\vec{d}_{k + 1} \gets \vec{\uppi}_{k + 1} Q$
      \Operation*{VectorMatrixMultiplyFromLeft}
      $\vec{y} \gets \vec{y} + \mleft(-\frac{1}{1 - \gamma}\mright)
      \vec{d}_{k + 1}$
      \Operation*{\InPlace VectorAdd}
      $\textit{\textls{LTE}} \gets 2 C h \| \vec{y} \|, \textit{localTol} \gets
      (\tau - \textit{errorSum}) / \textit{timeLeft} \cdot h$\;
      \If(\tcp*[f]{Successful step}){$\textit{\textls{LTE}} < \textit{localTol}$}{%
        $\textit{timeLeft} \gets \textit{timeLeft} - h,
        \textit{errorSum} \gets \textit{errorSum} +
        \textit{\textls{LTE}}$\;
        \tcp{Do not try to increase h after a failed step}
        \lIf{$\neg \textit{stepFailed}$}{%
          $h \gets h \cdot \min\{ \textit{maxIncrease},
          \sqrt[3]{\textit{localTol} / \textit{\textls{LTE}}} \}$
        }
        \KwSty{break}\;
      }
      $\textit{stepFailed} \gets \KwTrue, h \gets h \cdot \min\{
      \textit{leastDecrease}, \sqrt[3]{\textit{localTol} /
        \textit{\textls{LTE}}} \}$\;
    }
    Swap the references to $\vec{\uppi}_k, \vec{\uppi}_{k + 1}$ and
    $\vec{d}_k, \vec{d_{k + 1}}$\;
  }
  \KwRet{$\vec{\uppi}_k$}\;
  \caption{\textls{TR-BDF2} for transient analysis.}
  \label{alg:algorithms:transient:trbdf2}
\end{algorithmpage}

An implementation of \textls{TR-BDF2} based on the pseudocode of
\citet{DBLP:journals/cor/ReibmanT88} is shown in
\vref{alg:algorithms:transient:trbdf2}.

In lines~\ref{ln:algorithms:transient:trbdf2:solve1} and~%
\ref{ln:algorithms:transient:trbdf2:solve2} any linear equation solver
from \vref{sec:algorithms:solvers} may be used except power iteration,
since the matrices, in general, do not have strictly negative
diagonals. Due to the way the matrices, which are linear combinations
of $I$ and $Q$, are passed to the inner solvers, our \textls{TR-BDF2}
integrator is currently limited to $Q$ matrices which are not in block
form.

The vectors $\vec{\uppi}_0, \vec{\uppi}_k$ and
$\vec{\uppi}_{k + \gamma}, \vec{d}_{k + 1}$ may share storage,
respectively, therefore only 4 state-space sized vectors are required
in addition to the initial distribution $\vec{\uppi}_0$.

The most computationally intensive part is the solution of two linear
equation per every attempted step, which may make \textls{TR-BDF2}
extremely slow. However, its performance does \emph{not} depend on the
stiffness of the Markov chain, which may make it better suited to stiff
\textls{CTMCs} than uniformization%
~\citep{DBLP:journals/cor/ReibmanT88}.

\section{Mean time to first failure}

In \textls{MTFF} calculation (\vref{ssec:background:mtff}), quantities
of the forms
\begin{equation}
  \MTFF = - \underbrace{\vec{\uppi}_U Q_{UU}^{-1}}_{\vec{\upgamma}} \vec{1}^\T, \quad
  \Pr(X(\TFF_{+ 0}) = y) = - \underbrace{\vec{\uppi}_U Q_{UU}^{-1}}_{\vec{\upgamma}} \vec{q}_{UD'}^\T
  \tag{\ref{eq:background:mtff}, \ref{eq:background:failure-mode}
    revisited}
\end{equation}
are computed, where $U, D, D'$ are the set of operations states,
failure states and a specific failure mode $D' \subsetneq D$,
respectively.

The vector $\vec{\upgamma} \in \RR^{\lvert U \rvert}$ is the solution
of the linear equation
\begin{equation}
  \label{eq:algorithms:mtff:gamma}
  \vec{\upgamma} Q_{UU} = \vec{\uppi}_{U}
\end{equation}
and may be obtained by any linear equation solver.

The sets $U, D = D_1 \cup D_2 \cup \cdots$ are constructed by the
evaluation of \CTL\ expressions. If the failure mode $D_i$ is
described by $\varphi_i$, then the sets $D$ and $U$ are described by
\CTL\ formulas
$\varphi_D = \neg \ctlAX \KwTrue \lor \varphi_1 \lor \varphi_2 \lor
\cdots$
and $\varphi_U = \neg \varphi_D$, where the deadlock condition
$\neg \ctlAX \KwTrue$ is added to make \eqref{eq:algorithms:mtff:gamma}
irreducible.

After the set $U$ is generated symbolically, the matrix $Q_{UU}$ may be
decomposed in the same way as the whole state space $S$. Thus, the
vector-matrix operations required for solving
\eqref{eq:algorithms:mtff:gamma} can be executed as in steady-state
analysis.
