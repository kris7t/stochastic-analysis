\chapter{Evaluation}
\label{chap:evaluation}

\section{Testing}

When developing an algorithm library for formal analysis of safety critical systems it is vital to verify the correctness of the implementation. Since the complexity of the code base makes formal verification difficult we confined ourselves to rigorously testing the functionalities provided by the library.

\subsection{Combinatorial testing}

As described in \cref{chap:algorithms} algorithms use the common vector and matrix interfaces to perform various operations. This makes the used storage techniques transparent which in turn makes the code base more concise, reusable and less prone to errors. 

The most important requirement against the datastructure operations is mathematical correctness regardless of the storage technique used. Considering the number of implementations for a given interface and the previous requirement we used a simple unit testing design pattern (also known as interface testing pattern) as the core building block for the datastructure testing \citep{myers2011art}. 

The basic idea behind this pattern is to write unit tests for interface operations without any knowledge about the concrete implementation. Hiding implemetation details can be achieved in a number of ways. Some unit testing frameworks (like \emph{NUnit}, \citep{NUnit}) support the usage of generic test classes and running them for multiple concrete types.

Since most of the time multiple instances of different types of interface implementations are needed in a single unit test we choose a more flexible approach for hiding implementation details. This approach is based on class inheritance and abstract factory methods. Whenever we need an instance for a given interface we delegate the instantiation to an abstract factory method in the test class. 

A drawback of this approach is that the test class itself becomes abstract so we can't run the tests inside it directly. However we can easily inherit from the base test class and implement the abstract factory methods in any way we'd like. But the most important advantage of this approach manifests itself when we apply the virtual modifier to one or more unit tests in the base class. This way we can completely override tests in the derived classes if needed based on the types of the interface implementations. So the first step in testing the datastructure library was to implement these abstract unit tests that operate on an interface level.

\subsubsection{Abstract tests}

In order to make sure we cover the most possible usage scenarios of the datastructure we followed some common testing techniques. As a first step we used equivalence partitioning to identify the valid and invalid ranges of the parameters of the operations. Next we implemented the parameter value checks in interface code contract classes using Microsoft's Code Contract library \citep{CodeContracts}. This enabled us to implement the parameter check logic in one place for an operation making the code more maintainable. Moreover every class implementing a datastructure interface and it's operations will automatically contain these logics if code contracts are enabled. Code contracts can be disabled if needed resulting in a perfomance boost for the datastructure library since the parameter checks are skipped.

Writing unit tests for valid parameter values was straighforward since it's possible to cover multiple valid parameter ranges with a single unit test. However testing for invalid parameter values requires some care. We must ensure that there is only one invalid parameter per unit test so one error doesn't obscure the other. This significantly increases the number of unit tests and the possibility that we forget to test an invalid parameter range. Therefore we aimed to gather every possible invalid parameter range automatically. 

For this purpose we used Microsoft's IntelliTest tool \citep{IntelliTest} (formerly known as Pex, \citep{tillmann2008pex}) which assists in automating white-box and unit-testing. IntelliTest automatically generates unit tests using constraint satisfaction problem solving based on the source code of the method under test. Using IntelliTest on our interface code contract classes provided us with many invalid parameter values which we could use in our abstract unit tests.

\subsubsection{Concrete tests}

Once the abstract unit tests were implemented the next step was to create the derived classes for every storage combination and implement the abstract factory methods. Since the number of possible combinations were too many to implement manually we used Microsoft's Text Template Transformation Toolkit (T4, \citep{T4}) to generate the derived classes. The created template files provide ways to modify the behavior of abstract tests (through simple regular expression based configuration files) and to decrease the number of generated test by using pairwise testing instead of full combinatorial testing of implementation combinations. To generate the combinations for pairwise testing we used the ACTS tool \citep{borazjany2012combinatorial}. 

As a result of this testing process more than $78\;000$ unit tests were generated using full combinatorial testing (more than $18\;000$ with pairwise testing) which together with the behavior configuration files serve as a quasi-formal specification for the expected behavior of future and modified implementations (e.g.\ perfomance optimization). Breaking changes in implementation should either be rejected or the test suite and configuration files should be revised as specification change. Every unit test was executed sucessfully for both sequential and parallel operation implementations.

\subsection{Software redundancy based testing}

Apart from testing the datastructure operation implementations it is vital to test the correctness of higher level algorithms used in the analysis workflow, e.g.\ the linear equation solver algorithms. Testing every implemented algorithm one by one with unit tests would be tremendous work and it can't be easily automated (or maintained in case of manual testing). Moreover every algorithm is used as part of a bigger workflow which raises the question of compatibility of algorithms during an analysis. 

As described in \cref{chap:overview:sec:our-workflow} for almost every step of the workflow numerous algorithms are available.

\begin{obs}
  \label{obs:evaluation:reward-results}
  The result of a performance analysis (e.g.\ reward calculation) is mathematically independent of the used analysis workflow. It only depends on the possible behaviors of the system and the definition of the required performance measure. Two results calculated by using two different analysis methods can only differ from eachother due to the numercial precision properties of the used algorithms.
\end{obs}  

Combining our fully configurable workflow with \cref{obs:evaluation:reward-results} presents a new approach for testing the algorithm implementations in a maintainable and almost automatic manner. We can take advantage of the concept of software redundancy commonly used in safety critical applications. The main idea behind software redundancy is to perform a calculation multiple times with usually fundamentally different algorithms (often developed by independent teams) thus minimizing the possibility of common mode failures. After the calculations a voting component examines whether every algorithm calculated the same result. If that's not the case then one or more of the algorithms are incorrect.

The building block for this testing phase consists of running our analysis workflow for a given configuration and saving the calculated results (reward and sensitivity values). We generated $588$ mathematically consistent configurations in total, executed them for our running example (\cref{fig:background:spn:sharedresource}), multiple benchmark models and case studies. Finally we examined the maximum absolute difference of the calculated results as an error indicator for each performance measure in each model as presented in the next sections.

Beside verifying the correctness of the developed algorithms, our main goal with software redundance based testing is to gather a knowledge base about the effectiveness of different analysis approaches for models with varying properties. The gathered observations are summarized in \cref{sec:evaluation:results}.

\section{Measurements}

In this section we introduce the models used throughout the testing and benchmarking phase then we present preliminary results about the performance of solver algorithms using the implemented block Kronecker decomposition matrix form.

\subsection{Shared resource}

One of the bechmark models was the modified version of stochastic \emph{SharedResource} (SR) system (presented in \cref{fig:background:spn:sharedresource}). We added three more nodes to the system and modified some of its parameters along two dimensions. On one hand we increased the number of reachable states by adding more resources and local processes to the model. On the other hand we changed the rates of transitions in the model resulting in changes in it's stochastic  behaviour. We created symmetric, slightly asymmetric and significantly asymmetric versions of the model. In the third case there are orders of magnitude of difference between the transitions rates of the model.  

\subsection{Kanban}

We used the \SPN\ version of the \emph{kanban} (KB) system \citep{ciardo2003logical} as the other benchmark model. The model was scaled by modifying the available resources at each stage of the model resulting in an increase in the size of the state space.

\subsection{Cloud performability}

One of the models we used for analyis represents a cloud architecture \citep{ghosh2012scalable} with physichal and virtual machines serving incoming jobs using warm and cold spare resources in case of increasing load. We modified some aspects of the model in \citep{ghosh2012scalable} since our library currently doesn't support the GPSN formalism.

\section{Baselines}

In the next paragraphs we present some of the state of the art modeling and analysis tools and their features without being exhaustive. These will serve as a baseline for evaluating the performance of the best algorithm combinations of our library against the algorithms available in presented analysis tools.

\subsection{GreatSPN}

\emph{GreatSPN} (also refered to as \emph{GSPN}) \citep{baarir2009greatspn} is a freely available tool for the modeling, verification and performance evaluation of distributed stochastic systems. The tool support numerous extenstions of stochastic Petri nets and provides a wide range of analysis options the various types of Petri nets. To our best knowledge GSPN is available only for a few Unix based systems with single core 32-bit CPUs which poses a restriction on the amount of resources it can use during an analysis.

\subsection{Möbius}

Möbiues is a commercial tool with with many supported formalisms and analysis types \citep{courtney2009mobius}. Furthermore they use runtime codegeneration to greatly improve the tool's performance. They provide a white-box analysis library making it possible for third party developer to integrate their own tools with Möbius.

\section{Results}
\label{sec:evaluation:results}

\begin{table}
  \centering{\small
  \begin{tabular}{@{}lrllrr@{}}
    \toprule
    \multicolumn{1}{@{}c}{Model} & \multicolumn{1}{c}{States} &
    \multicolumn{1}{c}{Generator} & \multicolumn{1}{c}{Algorithm} &
    \multicolumn{1}{c}{Memory} & \multicolumn{1}{c@{}}{Time} \\
    \midrule
    SR-Sym-7 & $10\,775\,710$
    & Sparse & Uniformization & $3\,120$~MiB & $279$~s \\
    & & & \textls{BiCGSTAB} & $3\,450$~MiB & $236$~s \\
    & & \textls{BK} & Uniformization & $650$~MiB & $222$~s \\
    & & & \textls{BiCGSTAB} & $815$~MiB & $162$~s \\

    SR-Asym-7 & $10\,775\,710$
    & Sparse & Uniformization & $3\,116$~MiB & $316$~s \\
    & & & \textls{BiCGSTAB} & $3\,450$~MiB & $236$~s \\
    & & \textls{BK} & \textls{BiCGSTAB} & $812$~MiB & $373$~s \\

    SR-Degen-7 & $10\,775\,710$
    & Sparse & \textls{BiCGSTAB} & \multicolumn{2}{c@{}}{Breakdown} \\
    & & BK & Group \textls{GS} / Jacobi & \multicolumn{2}{c@{}}{No convergence} \\
    
    SR-Sym-9 & $81\,466\,099$
    & Sparse & \textls{BiCGSTAB} & $25\,564$~MiB & $2\,542$~s \\

    SR-Asym-9 & $81\,466\,099$
    & Sparse & \textls{BiCGSTAB} & \multicolumn{2}{c@{}}{Oscillation} \\
    & & \textls{BK} & Group \textls{GS} / Jacobi & $2\,388$~MiB & $9\,402$~s \\
    
    Cloud-3-2 & $20\,047\,500$
    & Sparse & \textls{BiCGSTAB} & \multicolumn{2}{c@{}}{Out of memory} \\
    & & BK & \textls{BiCGSTAB} & \multicolumn{2}{c@{}}{Breakdown} \\
    & & & Group \textls{GS} / Jacobi & $684$~MiB & $3\,379$~s \\
    
    KanBan-5 & $2\,546\,432$
    & Sparse & Uniformization & $833$~MiB & $54$~s \\
    & & & \textls{BiCGSTAB} & $911$~MiB & $92$~s \\
    & & BK & Uniformization & $360$~MiB & $70$~s \\
    & & & \textls{BiCGSTAB} & $392$~MiB & $124$~s \\

    KanBan-7 & $41\,644\,800$
    & Sparse & Uniformization & $12\,471$~MiB & $909$~s \\
    & & BK & Uniformization & $6\,253$~MiB & $1\,135$~s \\
    \bottomrule
  \end{tabular}}
  \caption{Preliminary benchmark results.}
  \label{tab:evaluation:results}
\end{table}


\begin{obs}
	Based on the preliminary measurements and combinatorial testing we can note some interesting observations:

	\begin{enumerate}
		\item As expected the storage requirement of the block Kronecker form is almost an order of magnitude lower than that of the sparse form. 

		\item For models with moderate state space sizes (approximately a few millions) the sparse form outperforms the block Kronecker form.

		\item However for models with considerably bigger state space sizes (almost a hundred million) the block Kronecker form outperforms the sparse form not just in memory usage but in analysis time as well. This is probably because of the inefficient cache usage of the sparse structure.

		\item For models consisting of similar transition rates the BiCGSTAB algorithm is the most effective solution method for both sparse and block Kronecker matrix representations. 

		\item Slower, but more memory efficient solvers (Gauss-Seidel iteration, Jacobi iteration) often diverged using sparse matrix form while converged using the block Kronecker form. This is due to the possibility of different state ordering in the block Kronecker form.  

		\item For irregular models (i.e. containing very different transition rates) it's possible that the BiCGSTAB algorithm won't converge while other algorithms can handle the model.
	\end{enumerate}
\end{obs}

Every data used for testing and measurement is publicly available at
\url{https://github.com/kris7t/stochastic-analysis}.