\chapter{Background}
\label{chap:background}

\section{Petri nets}

Petri nets are a widely used graphical and mathematical modeling tool
for systems which are concurrent, asynchronous, distributed, parallel
or nondeterministic.

\begin{dfn}
  A \emph{Petri net} is a 5-tuple $\PN = (P, T, F, W, M_0)$, where
  \begin{asparaitem}
  \item $P = \{p_1, p_2, \ldots, p_n\}$ is a finite set of places;
  \item $T = \{t_1, t_2, \ldots, t_m\}$ is a finite set of transitions;
  \item $F \subseteq (P \times T) \cup (P \times T)$ is a set of
    arcs, also called the flow relation;
  \item $W: F \to \NNpos$ is an arc weight function;
  \item $M_0: P \to \NN$ is the initial marking;
  \item $P \cap T = \emptyset$ and $P \cup T \neq \emptyset$.
  \end{asparaitem}
  A \emph{Petri net structure} without an initial marking is a 4-tuple
  $N = (P, T, F, W)$, while a Petri net with a given initial marking
  is denoted by $(N, M_0)$~\citep{murata1989petri}.
\end{dfn}

Arcs from $P$ to $T$ are called \emph{input arcs}. The input places of
a transition $t$ are denoted by $\inarc{t} = \{ p : (p, t) \in F
\}$.
In contrast, arcs of the form $(t, p)$ are called \emph{output arcs}
and the output places of $t$ are denoted by
$\outarc{t} = \{p : (t, p) \in F \}$.

A \emph{marking} $M: P \to \NN$ assigns a number of \emph{tokens} to each
place. The transition $t$ is \emph{enabled} in the marking $M_1$
\paren{written as $M_1\,\tranto{t}\,$} when $M(p) \ge W(p, t)$ for all $p \in
\inarc{t}$.

Petri nets are graphically represented as edge weighted directed
bipartite graphs. Places are drawn as circles, while transitions are
drawn as rules or rectangles. Edge weights of $1$ are ususally omitted
from presentation. Dots on places correspond to tokens in the current
marking.

If $M_1\,\tranto{t}$ the transition $t$ can be \emph{fired} to get a
new marking $M_2$ \paren{written as $M_1 \tranto{t} M_2$} by
decreasing the token counts for each place $p \in \inarc{t}$ by
$W(p, t)$ and increasing the token counts for each place
$p \in \outarc{t}$ by $W(t, p)$. Note that in general, $\inarc{t}$ and
$\outarc{t}$ need not be disjoint. Thus, the firing rule can be
written as
\begin{equation}
  \label{eq:background:petri:fire}
  M_2(p) = M_1(p) - W(p, t) + W(t, p) \text,
\end{equation}
where we take $W(x, y) = 0$ if $(x, y) \notin F$ for brevity.

A marking $M'$ is \emph{reachable} from the marking $M$ (written as
$M \reachto M'$) if there exists a sequence of markings and transitions
for some finite $k$ such that
\begin{equation}
  M_1 \tranto{t_{i_1}} M_2 \tranto{t_{i_2}} M_3 \tranto{t_{i_3}}
  \cdots \tranto{t_{i_{k - 1}}} M_{k - 1} \tranto{t_{i_k}} M_k \text,
\end{equation}
where $M_1 = M$ and $M_k = M'$. A marking $M$ is in the reachable
\emph{state space} of the net if $M_0 \reachto M$.

\begin{figure}
  \centering
  \newcommand*{\examplenet}[3]{
    \begin{tikzpicture}
      \matrix [column sep=1cm,ampersand replacement=\&] {
        \node [petri net place] (h2) {#1}; \& \& \\
        \& \node [petri net transition] (t) {} ; \& \node [petri net place] (h2o) {#3}; \\
        \node [petri net place] (o2) {#2}; \& \& \\
      };
      \draw [-{Latex}] (h2) edge node [above] {$2$} (t) (o2) edge (t)
      (t) edge node [above] {$2$} (h2o);
      \node [below=0cm of h2] {$p_{\text{H$_2$}}$};
      \node [below=0cm of o2] {$p_{\text{O$_2$}}$};
      \node [below=0cm of h2o] {$p_{\text{H$_2$O}}$};
      \node [below=0cm of t] {$t$};
    \end{tikzpicture}}

  $\vcenter{\hbox{\examplenet{\token\token}{\token\token}{}}}
  \quad\vcenter{\hbox{$\tranto{t}$}}\quad
  \vcenter{\hbox{\examplenet{}{\token}{\token\token}}}$
  \caption{A Petri net model of the reaction of hydrogen and oxygen.}
  \label{fig:background:petri:h2o}
\end{figure}

\begin{example}
  The Petri net in~\cref{fig:background:petri:h2o} models the chemical
  reaction
  \begin{equation}
    2 \, \mathrm{H}_2 + \mathrm{O}_2 \rightarrow 2 \,
    \mathrm{H}_2\mathrm{O} \text.
  \end{equation}
  In the initial marking \paren{left} there are two hydrogen and two
  oxygen molecules, represented by token on the places $p_{\text{H$_{2}$}}$
  and $p_{\text{O$_{2}$}}$, therefore the transition $t$ is
  enabled. Firing $t$ yields the marking on the right where the two
  tokens on $p_{\text{H$_{2}$O}}$ are the reaction products. Now $t$ is no
  longer enabled.
\end{example}

\begin{figure}
  \centering
  \begin{tikzpicture}
    \matrix [column sep=0.65cm,row sep=0.5cm,label distance=0cm] {
      \node [petri net place,label={\strut$p_{W_1}$}] (W1) {};
      & \node [petri net transition,label={\strut$t_{a_1}$}] (a1) {};
      & \node [petri net place,label={\strut$p_{S_1}$}] (S1) {};
      & \node [petri net transition,label={\strut$t_{d_1}$}] (d1) {};
      & \node [petri net place,label={\strut$p_{C_1}$}] (C1) {\token};
      & \node [petri net transition,label={\strut$t_{r_1}$}] (r1) {}; \\
      & & \node [petri net place] (S) {\token}; & & & \\
      \node [petri net place,label={-90:\strut$p_{W_2}$}] (W2) {};
      & \node [petri net transition,label={-90:\strut$t_{a_2}$}] (a2) {};
      & \node [petri net place,label={-90:\strut$p_{S_2}$}] (S2) {};
      & \node [petri net transition,label={-90:\strut$t_{d_2}$}] (d2) {};
      & \node [petri net place,label={-90:\strut$p_{C_2}$}] (C2) {\token};
      & \node [petri net transition,label={-90:\strut$t_{r_2}$}] (r2) {}; \\
    };
    \node [right=0cm of S] {$p_S$};
    \draw [-{Latex}] (W1) edge (a1) (S) edge (a1) (a1) edge (S1)
    (S1) edge (d1) (d1) edge (S) (d1) edge (C1) (C1) edge (r1)
    (W2) edge (a2) (S) edge (a2) (a2) edge (S2)
    (S2) edge (d2) (d2) edge (S) (d2) edge (C2) (C2) edge (r2);
    \draw [-{Latex},rounded corners] (r1.east) -| ++(0.65cm,1.2cm) -|
    ($(W1.west)+(-0.65cm,0)$) -- (W1);
    \draw [-{Latex},rounded corners] (r2.east) -| ++(0.65cm,-1.2cm) -|
    ($(W2.west)+(-0.65cm,0)$) -- (W2);
  \end{tikzpicture}
  \caption{The \emph{SharedResource} Petri net model.}
  \label{fig:background:petri:sharedresource}
\end{figure}

\begin{example}
  In~\cref{fig:background:petri:sharedresource} we introduce the
  \emph{SharedResource} model which will serve as a running example
  throughout this report.

  The model consists of a single shared resource $S$ and two
  consumers. Each consumer can be in one of the
  $C_i$ \paren{calculating locally}, $W_i$ \paren{waiting for
    resource} and $S_i$ \paren{shared working} states. The transitions
  $r_i$ \paren{request resource}, $a_i$ \paren{acquire resource} and
  $d_i$ \paren{done} correspond to behaviours of the consumers.

  The Petri net model allows the verification of safety properties,
  e.g.~we can show that there is mutual exclusion
  --~$M(S_1) + M(S_2) \le 1$ for all reachable markings~-- or that
  deadlocks cannot occur. In contrast, we cannot compute dependability
  or performability measures \paren{e.g.~the utilization of the shared
    resource or number of calculations completed per unit time}
  because the model does not describe the temporal behaviour of the
  system.
\end{example}

\subsection{Petri nets extended with inhibitor arcs}

One of the most frequently used extensions of Petri nets is the
addition of inhibitor arcs, which modifies the rule for transition
enablement. This modification gives Petri nets expressive power
equivalent to Turing
machines~\citep{DBLP:conf/apn/Chrzastowski-Wachtel99}.

\begin{dfn}
  A \emph{Petri net with inhibitor arcs} is a 3-tuple $\PNI = (\PN,
  I, W_I)$, where
  \begin{asparaitem}
  \item $\PN = (P, T, F, W, M_0)$ is a Petri net;
  \item $I \subseteq P \times T$ is the set of inhibitor arcs;
  \item $W_I: I \to \NNpos$ is the inhibitor arc weight function.
  \end{asparaitem}
\end{dfn}

Let $\inharc{t} = \{ p : (p, t) \in I\}$ denote the set of inhibitor
places of the transition $t$. The enablement rule for Petri nets with
inhibitor arcs can be formalized as
\begin{equation}
  \label{eq:background:petri:inh-fire}
  M\,\tranto{t} \Longleftrightarrow \text{\strut$M(p) \ge W(p, t)$ for all
    $p \in \inarc{t}$ and $M(p) < W_I(p, t)$ for all $p \in
    \inharc{t}$.}
\end{equation}
The firing rule \eqref{eq:background:petri:fire} remains unchanged.

\section{Continous-time Markov chains}



\section{Stochastic Petri nets}
