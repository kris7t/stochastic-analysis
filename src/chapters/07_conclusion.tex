\chapter{Conclusion and future work}
\label{chap:conclusion}

We have developed and presented our numeric backend to the
\emph{configurable stochastic analysis framework} for the
dependability, reliability and performability analysis of complex
asynchronous systems.  Our presented approach is able to combine the
strength and advantages of the different algorithms and data
structures into one framework. Various optimization techniques were
used during the development and many of the algorithms are
parallelized to exploit the advantages of modern mulitcore processor
architectures.

From the theoretical side, we have obtain preliminary results in the
adaptation of the state-of-the-art
\textls{IDR}($s$)\textls{STAB}($\ell$) algorithm to steady-state
stochastic analysis tasks for integration into the framework.

In addition we have investigated the composability of the various data
storage, numerical solution and infinitesimal generator matrix
representation techniques and combined them together to provide
configurable stochastic analysis in our framework.

Extensive investigation was executed in the field to be able to
develop 3 generator matrix decomposition and representation
techniques, 7 steady-state solvers, 2 transient analysis algorithms
for the computation of engineering measures.

Our long term goal is to provide these analysis techniques also for a
wider community, we have integrated our library into the
\textsc{PetriDotNet} framework. Our algorithms are used also in the
education for illustration purposes of the various stochastic analysis
techniques. In addition, our tool was also used in an industrial
project: one of our case-studies is based on that project. More than
$70\,000$ generated test cases serve to ensure correctness as much as
possible. In addition, software redundancy based testing was applied
to further improve the quality of our library.

\needspace{5ex}

Despite our attempts to be as comprehensive as possible, many
promising directions for future research and development are
\begin{itemize}
\item more extensive benchmarking of algorithms to extend the
  knowledge base about the effectiveness and behavior of stochastic
  analysis approaches toward and adaptive framework for stochastic
  analysis;
\item completion of the work started on
  \textls{IDR}($s$)\textls{STAB}($\ell$) by improving the
  stabilization part
  \citep{sleijpen1995maintaining,sonneveld2010convergence} of the
  algorithm to attain convergence on a wide range of stochastic models
  and parameter settings;
\item the implementation and development of further numerical
  algorithms, including those that can take advantage of the various
  decompositions of stochastic models~%
  \citep{buchholz1999structured,buchholz2000multilevel,%
    dayar2012analyzing};
\item reduction of the size of Markov chains through the exploitation
  of model symmetries~\citep{buchholz1994exact,haddad1995evaluation};
\item the development of preconditioners for the available iterative
  numerical solution methods~%
  \citep{DBLP:journals/informs/LangvilleS04};
\item distributed implementations of the existing algorithms~%
  \citep{DBLP:conf/imcsit/BylinaB08};
\item support for fully symbolic storage and solution of Markov
  chains~\citep{DBLP:journals/sigmetrics/CiardoM05,%
  DBLP:conf/qest/ZhaoC12,DBLP:conf/qest/2012};
\item the use of tensor decompositions instead of vectors to store
  state distributions and intermediate results to greatly reduce
  memory requirements of solution algorithms~%
  \citep{grasedyck2013literature,ballani2013projection,%
  dolgov2013tt}.
\end{itemize}

\paragraph{Acknowledgment} We would like to thank Prof.~Peter
Buchholz for his helpful comments on Krylov subspace solution
algorithms.

We would like to thank IncQueryLabs Ltd.~for their support during the
summer internship.
