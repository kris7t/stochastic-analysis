\chapter{Conclusion and future work}
\label{chap:conclusion}

We have developed and presented our \emph{configurable stochastic
  analysis framework} for the dependability, reliability and
performability analysis of complex asynchronous systems.  Our
presented approach is able to combine the strength and advantages of
the different algorithms into one framework.  We have not only
implemented a stochastic analysis library, but we integrated the
various state space traversal, generator matrix representation and
numerical analysis algorithms together. Various optimization
techniques were used during the development and many of the algorithms
are paralellized to exploit the advantages of modern mulitcore
processor architectures.

From the theoretical side, we have developed an algorithm which can
efficiently compile the symbolic state space representation into the
complex data structure representation of the stochastic process. We
have formalised our algorithm and proved its correctness. 
This new algorithm helps us to exploit the efficient state space
representation of symbolic algorithms in stochastic analysis.

In addition
we have investigated the composability of the various data storage,
numerical solution and state space representation techniques and
combined them together to provide configurable stochastic analysis
in our framework.

Extensive investigation was executed in the field to be able to
develop more than 2 state space exploration algorithms, 3 state space
representation algorithms, 3 generator matrix decomposition and
representation algorithms, 7 steady-state solvers, 2 transient
analysis algorithms and 4 different computation algorithms for
engineering measures.  Our long term goal is to provide these analysis
techniques also for a wider community, we have integrated our library
into the \textsc{PetriDotNet} framework. Our algorithms are used also
in the education for illustration purposes of the various stochastic
analysis techniques. In addition, our tool was also used in an
industrial project: one of our case-studies is based on that project.
The stochastic analysis library is built from more than $50\,000$
lines of code. More than $70\,000$ generated test cases serve to
ensure correctness as much as possible. In addition, software
redundancy based testing was applied to further improve the quality of
our library.

Despite our attempts to be as comprehensive as possible, many
promising directions for future research and development are
\begin{itemize}
\item more extensive benchmarking of algorithms to extend the
  knowledge base about the effectiveness and behavior of stochastic
  analysis approaches toward and adaptive framework for stochastic
  analysis;
\item support for extended formalisms for stochastic models, such as
  Generalized Stochastic Petri Nets (\textls{GSPN})~%
  \citep{DBLP:journals/tse/TeruelFP03} and Stochastic Automata Networks
  (\textls{SAN})~\citep{DBLP:conf/pnpm/1985}, as well as models with
  more general stochastic transition behaviors~%
  \citep{Longo:2015:TSR:2767455.2767457};
\item the implementation and development of further numerical
  algorithms, including those that can take advantage of the various
  decompositions of stochastic models~%
  \citep{buchholz1999structured,buchholz2000multilevel,%
    dayar2012analyzing};
\item reduction of the size of Markov chains through the exploitation
  of model symmetries~\citep{buchholz1994exact,haddad1995evaluation};
\item the development of preconditioners for the available interative
  numerical solution methods~%
  \citep{DBLP:journals/informs/LangvilleS04};
\item distributed implementations of the existing algorithms~%
  \citep{DBLP:conf/imcsit/BylinaB08};
\item support for fully symbolic storage and solution of Markov
  chains~\citep{DBLP:journals/sigmetrics/CiardoM05,%
  DBLP:conf/qest/ZhaoC12,DBLP:conf/qest/2012};
\item the use of tensor decompositions instead of vectors to store
  state distributions and intermediate results to greatly reduce
  memory requirements of solution algorithms~%
  \citep{grasedyck2013literature,ballani2013projection,%
  dolgov2013tt}.
\end{itemize}

\paragraph{Acknowledgement} We would like to thank IncQueryLabs
Ltd. for their support during the summer internship.
