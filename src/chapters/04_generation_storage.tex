\chapter{Efficient generation and storage of continuous-time Markov chains}
\chaptermark{Efficient generation and storage of CTMCs}
\label{chap:genstor}

\section{Explicit methods}

\subsection{Explicit state space generation}

\subsection{Explicit generator matrices}

\subsection{Block Kronecker generator matrices}

\subsubsection{Kronecker generator matrices}

The Kronecker decomposition for a superposed \textls{SPN} with $J$
components expresses the infinitesimal generator matrix of the
associated \CTMC\ in the form
\begin{equation}
  \label{eq:genstor:explicit:kronecker}
  Q = Q_O + Q_D, \quad Q_O = \bigkrplus_{j = 0}^{J - 1} \loc{Q_L}{j}
  + \sum_{t \in T_S} \Lambda(t) \bigkrtimes_{j = 0}^{J - 1}
  \loc{Q_{t}}{j}, \quad Q_D = -\diag \{ Q_O \vec{1}^\T \} \text,
\end{equation}
where $Q_O$ and $Q_D$ are the off-diagonal and diagonal parts of
$Q$. The matrix
\begin{equation}
  \loc{Q_L}{j} = \sum_{t \in \loc{T_L}{j}} \Lambda(t) \, \loc{Q_L}{j}
\end{equation}
is the \emph{local} transition matrix of the component $j$, while
the matrix
\begin{equation}
  \loc{Q_t}{j} \in \RR^{n_j \times n_j}, \quad
  \loc{q_t}{j}[\loc{x}{j},\loc{y}{j}] = \begin{cases}
    1 & \text{if $\loc{x}{j} \tranto{t} \loc{y}{j}$,} \\
    0 & \text{otherwise}
  \end{cases}
\end{equation}
describes the effects of the transition $t$ on
$\LN{j}$. $\loc{Q_t}{j}$ has a nonzero element for every local state
transition caused by $t$. If $j \notin \supp t$, $\loc{Q_t}{j}$ is
an $n_j \times n_j$ identity matrix.

It can be seen that
\begin{align}
  q_O[\vec{x}, \vec{y}]
  &= \sum_{\vphantom{\loc{T_L}{j}}j = 0}^{J - 1} \,\, \sum_{t \in
    \loc{T_L}{j}} \Lambda(t) \,
    \loc{q_t}{j}[\loc{x}{j}, \loc{y}{j}] \, + \sum_{t \in T_S} \Lambda(t)
    \prod_{\vphantom{T_S}j = 0}^{J - 1}
    \loc{q_t}{j}[\loc{x}{j}, \loc{y}{j}] \\
  &= \sum_{\vphantom{\loc{T_L}{j}}j = 0}^{J - 1} \,\,
    \sum_{\substack{\vphantom{\loc{T_L}{j}}t \in \loc{T_L}{j} \\
  \mathclap{\loc{x}{j} \tranto{t} \loc{y}{j}}}} \Lambda(t) \, +
  \sum_{\mathclap{\in T_S, \vec{x} \tranto{t} \vec{y}}} \Lambda(t)
  = \sum_{\mathclap{\vphantom{T_S}t \in T, \vec{x} \tranto{t}
  \vec{y}}} \Lambda(t) \text,
  \label{eq:genstor:explicit:block:qblock}
\end{align}
which is the same as \vref{eq:background:spn:q-d}. Indeed,
\cref{eq:genstor:explicit:kronecker} is a representation of the
infinitesimal generator matrix.

The matrices $\loc{Q_L}{j}$ and $\loc{Q_t}{j}$ and the vector
$-Q_O \vec{1}^\T$ together are usually much smaller than the full
generator matric $Q$ even when stored in a sparse matrix form. Hence
Kronecker decomposition may save a significant amount of storage at
the expense of some computation time.

Unfortunately, the Kronecker generator $Q$ is a
$n_0 n_1 \cdots n_{J - 1} \times n_0 n_1 \cdots n_{J - 1}$ matrix,
i.e.~in encodes the state transitions in the potential state space
$\PS$ instead of the reachable state space $\RS$.

\emph{Potential Kronecker methods}%
~\citep{DBLP:journals/informs/BuchholzCDK00} perform computations with
the $\lvert \PS \rvert \times \lvert \PS \rvert$ $Q$ matrix and
vectors of length $\lvert \PS \rvert$. In addition to increasing
storage requirements, this may lead to problems in some numerical
solution algorithms, because the \CTMC\ over $\PS$ is not neccessarily
irreducible even if it is irreducible over $\RS$.

In contrast, \emph{actual Kronecker methods}~%
\citep{DBLP:journals/tse/Kemper96,%
  DBLP:journals/informs/BuchholzCDK00,%
  DBLP:journals/fgcs/BenoitPS06} work with vectors of length
$\lvert \RS \rvert$. However, additional conversions must be performed
between the actual dense indexing of the vectors and the potential
sparse indexing of the $Q$ matrix, which leads to implementation
complexities and computational overhead.

A third approach, which we discuss in the next subsection, imposes a
hierarchical structure on $\RS$~%
\citep{DBLP:conf/cpe/BauseBK98,%
  DBLP:journals/sigmetrics/BuchholzK98,%
  DBLP:journals/tse/Buchholz99}.

\subsubsection{Macro state construction}

The hierarchical structuring of the reachable state space expresses
$\RS$ as
\begin{equation}
  \RS = \bigcup_{\macro{\vec{x}} \in \Macro{\RS}}
  \prod_{\vphantom{\Macro{\RS}}j = 0}^{J - 1}
  \, \loc{\RS}{j}_{\loc{\macro{x}}{j}}, \quad
  \loc{\RS}{j} = \bigcup_{\mathclap{\loc{\macro{x}}{j} \in \loc{\Macro{\RS}}{j}}}
  \,\, \loc{\RS}{j}_{\loc{\macro{x}}{j}} \text,
\end{equation}
where
$\Macro{\RS} = \{\macro{0}, \macro{1}_1, \ldots, \Macro{\macro{n} - 1}
\}$
a set of \emph{global macro states},
$\loc{\Macro{\RS}}{j} = \{\loc{\macro{0}}{j}, \loc{\macro{1}}{j},
\ldots, \loc{\Macro{\macro{n}_j - 1}}{j} \}$ is the set of \emph{local
macro states} of $\LN{j}$, and $\loc{\RS}{j}_{x} = \{\loc{0}{j}_x,
\loc{1}{j}_x, \ldots, \loc{(n_{j, x} - 1)}{j}_x \}$ are the
\emph{local micro states} in the local macro state
$\loc{\macro{x}}{j}$. The product symbol denotes the composition of
local markings, as in \vref{eq:background:sspn:concat}.

The local micro states form a partition
$\loc{\RS}{j} = \bigcup_{x \in \loc{\Macro{RS}}{j}} \loc{\RS}{j}_{x}$
of the state space of the $j$th \textls{SSPN} component.

Construction of macro states is performed as follows%
~\citep{DBLP:journals/tse/Buchholz99}:
\begin{enumerate}
\item The equivalence relation $\loc{\sim}{j}$ is defined over
  $\loc{\RS}{j}$ as
  \begin{equation}
    \label{eq:genstor:explicit:block:locsim}
    \loc{x}{j} \mathbin{\loc{\sim}{j}} \loc{y}{j}
    \quad\Longleftrightarrow\quad \{ \loc{\hat{\vec{z}}}{j} : \vec{x}
    \in \RS, \loc{z}{j} =
    \loc{x}{j} \} = \{ \loc{\hat{\vec{z}}}{j} : \vec{x} \in \RS, \loc{z}{j} = \loc{y}{j} \}
    \text,
  \end{equation}
  where
  $\loc{\hat{\vec{z}}}{j} = (\loc{z}{0}, \ldots, \loc{z}{j - 1},
  \loc{z}{j + 1}, \ldots, \loc{z}{J - 1})$,
  i.e.~two local states are equivalent if they are reachable in the
  same combinations of ambient local markings. Therefore, the relation
  \begin{equation}
    \vec{x} \sim \vec{y} \quad\Longleftrightarrow\quad \loc{x}{j}
    \mathbin{\loc{\sim}{j}} \loc{y}{j} \text{ for all $j = 0, 1,
      \ldots, J - 1$,}
  \end{equation}
  defined over $\PS$, has the property that whether
  $\vec{x} \sim \vec{y}$, either both $\vec{x}$ and $\vec{y}$ are
  reachable (global) markings, or neither are.
\item Reachable local macro states are the partitions of
  $\loc{\RS}{j}$ generated by $\loc{\sim}{j}$. A bijection
  $\loc{\Macro{\RS}}{j} \leftrightarrow \loc{\RS}{j} / {\loc{\sim}{j}}$
  is formed between the integers
  $0, 1, \ldots, \loc{\macro{n}}{j} - 1$ and the local state
  partitions for each component $\LN{j}$.
\item The set of potential macro states is
  \begin{equation}
    \Macro{\PS} = \prod_{j = 0}^{J - 1} \loc{\Macro{\RS}}{j}
    \supseteq \Macro{\RS}
  \end{equation}
  the Descares product of the local macro states. If macro state
  $\macro{x} \in \Macro{\PS}$ contains a reachable state, all
  associated (micro) states are reachable, because $\Macro{\PS}$ is
  the partition $\RS / {\sim}$ of $\RS$ generated by the relation
  $\sim$. Thus, $\Macro{\RS}$ is constructed by enumerating the
  reachable macro states in $\Macro{\PS}$. A bijection is formed
  between the reachable subset of $\Macro{\PS}$ and the integers
  $\macro{0}, \macro{1}, \ldots, \Macro{\macro{n} - 1}$.
\end{enumerate}

\begin{algorithm}
  \KwIn{Reachable state space $\RS$, reachable local state spaces
    $\loc{\RS}{j}$}
  \KwOut{Macro state space $\Macro{RS}$, local macro state spaces
    $\loc{\Macro{\RS}}{j}$}
  \KwAllocate{bit vector $\vec{b} \in \{0, 1\}^{n_0 n_1 \cdots n_{J -
        1}}$ initialized with zeroes}\;
  \ForEach{$\vec{x} \in \RS$}{
    \tcp{Fill $\vec{b}$ with ones corresponding to reachable states}
    $b[n_{J - 1} n_{J - 2} \cdots n_1 \loc{x}{0} + n_{J - 1} n_{J - 2}
    \cdots n_2 \loc{x}{1} + \cdots + n_{J - 1} \loc{x}{J - 2} +
    \loc{x}{J - 1}] \gets 1$\;
  }
  \For{$j \gets 0$ \KwTo $J - 1$}{
    Reshape $\vec{b}$ into matrix $B$ with $n_j$ columns\;
    Partition the columns of $B$ by componentwise equality
    \label{ln:genstor:explicit:block:macro:partition}\;
    \ForEach{subset $S$ of the partition $B/{=}$}{
      Create a new local macro state $\loc{\macro{y}}{j}$ in
      $\loc{\Macro{\RS}}{j}$\;
      Assign all local micro states $z \in S$ to $\loc{\macro{y}}{j}$\;
      Drop all columns of $B$ corresponding to $S$
      but a single representant of $\loc{\macro{y}}{j}$\;
    }
  }
  \ForEach{$\macro{\vec{x}} \in \loc{\RS}{0} \times \loc{\RS}{1}
    \times \cdots \times \loc{\RS}{J - 1}$}{
    \lIf{$b[\macro{\vec{x}}] = 1$}{
      Add $\macro{\vec{x}}$ to $\Macro{\RS}$ as a global macro state
    }
  }
  \KwRet{$\Macro{\RS}, \bigl\{\loc{\Macro{\RS}}{j}\bigr\}_{j = 0}^{J -
      1}$}\;
  \caption{Hiearchical decomposition of the reachable state space into
    macro states by \citet{DBLP:journals/tse/Buchholz99}.}
  \label{alg:genstor:explicit:block:macro}
\end{algorithm}

The pseudocode for this process is shown in
\cref{alg:genstor:explicit:block:macro}. The decomposition is
extremely memory intensive due to the allocation of the bit vector
$\vec{b}$ of length $\lvert \PS \rvert$.

In \citep{DBLP:journals/tse/Buchholz99}, sorting the columns of $B$
lexicographically was recommended to calculate the partition $B / {=}$
In our implementation, we insert the rows of $B$ into a bitwise trie
and detect duplicates instead, so that no mapping between the original
order and sorted ordering of columns needs to be maintained.

\begin{runningExample}
  The macro states of the \emph{RunningExample} \textls{SSPN} model
  (\vref{fig:background:sspn:sharedresource}) are obtained from its
  component state space
  (\vref{tab:background:sspn:sharedresource-states}) as follows:
  \begin{enumerate}
  \item The bit vector $\vec{b}$ is filled according to the reachable
    states $\RS$,
    \begin{equation}
      \begin{blockarray}{r@{\mkern15mu}*{18}{>{$\bgroup\small$}c<{$\egroup$}}@{\mkern12mu}l}
        & 0 & 0 & 0 & 0 & 0 & 0 & 1 & 1 & 1 & 1 & 1 & 1 & 2 & 2 & 2 &
        2 & 2 & 2 & \\[-0.7ex]
        & 0 & 0 & 1 & 1 & 2 & 2 & 0 & 0 & 1 & 1 & 2 & 2 & 0 & 0 & 1 &
        1 & 2 & 2 & \\[-0.7ex]
        & 0 & 1 & 0 & 1 & 0 & 1 & 0 & 1 & 0 & 1 & 0 & 1 & 0 & 1 & 0 &
        1 & 0 & 1 & \\[-0.5ex]
        \begin{block}{r@{\mkern15mu}(*{18}{c})@{\mkern12mu}l}
          \vec{b} = & 1 & 0 & 1 & 0 & 0 & 1 & 1 & 0 & 1 & 0 & 0 & 1 &
          0 & 1 & 0 & 1 & 0 & 0 & \text, \\
        \end{block}
      \end{blockarray}
    \end{equation}
    where the mixed indices in small type refer to the states of the
    local nets $\LN{0}$, $\LN{1}$ and $\LN{2}$.
  \item We reshape $\vec{b}$ into a matrix $B$ so that each column
    corresponds to a local state of the component $\LN{0}$,
    \begin{equation}
      \begin{blockarray}{r*{2}{@{\mkern5mu}>{$\bgroup\small$}c<{$\egroup$}}*{3}{>{$\bgroup\small$}c<{$\egroup$}}}
        & & & 0 & 1 & 2 \\[-0.5ex]
        \begin{block}{r*{2}{@{\mkern5mu}>{$\bgroup\small$}c<{$\egroup$}}(*{3}{c})}
          \multirow{6}{*}{$B =\,$} & 0 & 0 & 1 & 1 & 0 \\
          & 0 & 1 & 0 & 0 & 1 \\
          & 1 & 0 & 1 & 1 & 0 \\
          & 1 & 1 & 0 & 0 & 1 \\
          & 2 & 0 & 0 & 0 & 0 \\
          & 2 & 1 & 1 & 1 & 0 \\
      \end{blockarray}
    \end{equation}
    in order to conclude that
    \begin{align}
      \loc{\Macro{\RS}}{0}_0
      &= \{ \loc{0}{0}_0 = \loc{M}{0}_0, \loc{1}{0}_0 = \loc{M}{0}_1 \},
      &\loc{\Macro{\RS}}{0}_1
      &= \{ \loc{0}{0}_1 = \loc{M}{0}_2 \}.
    \end{align}
  \item After removing all local states of $\LN{0}$ except
    representants of $\loc{\Macro{\RS}}{0}$, $\vec{b}$ is reshaped
    again
    \begin{equation}
      \begin{blockarray}{r*{2}{@{\mkern5mu}>{$\bgroup\small$}c<{$\egroup$}}*{3}{>{$\bgroup\small$}c<{$\egroup$}}}
        & & & 0 & 1 & 2 \\[-0.5ex]
        \begin{block}{r*{2}{@{\mkern5mu}>{$\bgroup\small$}c<{$\egroup$}}(*{3}{c})}
          \multirow{4}{*}{$B =\,$} & \macro{0} & 0 & 1 & 1 & 0 \\
          & \macro{0} & 1 & 0 & 0 & 1 \\
          & \macro{1} & 0 & 0 & 0 & 0 \\
          & \macro{1} & 1 & 1 & 1 & 0 \\
      \end{blockarray}
    \end{equation}
    to find that
    \begin{align}
      \loc{\Macro{\RS}}{1}_0
      &= \{ \loc{0}{1}_0 = \loc{M}{1}_0, \loc{1}{1}_0 = \loc{M}{1}_1 \},
      &\loc{\Macro{\RS}}{0}_1
      &= \{ \loc{0}{1}_1 = \loc{M}{1}_2 \}.
    \end{align}
  \item Finally, we reshape
    \begin{equation}
      \begin{blockarray}{r*{2}{@{\mkern5mu}>{$\bgroup\small$}c<{$\egroup$}}*{2}{>{$\bgroup\small$}c<{$\egroup$}}}
        & & & 0 & 1\\[-0.5ex]
        \begin{block}{r*{2}{@{\mkern5mu}>{$\bgroup\small$}c<{$\egroup$}}(*{2}{c})}
          \multirow{4}{*}{$B =\,$} & \macro{0} & \macro{0} & 1 & 0 \\
          & \macro{0} & \macro{1} & 0 & 1 \\
          & \macro{1} & \macro{0} & 0 & 1 \\
          & \macro{1} & \macro{1} & 0 & 0 \\
      \end{blockarray}
    \end{equation}
    and conclude
    \begin{align}
      \loc{\Macro{\RS}}{2}_0
      &= \{ \loc{0}{2}_0 = \loc{M}{2}_0\},
      &\loc{\Macro{\RS}}{2}_1
      &= \{ \loc{0}{2}_1 = \loc{M}{2}_1 \}.
    \end{align}
  \item Unfolding the matrix $B$
    \begin{equation}
      \begin{blockarray}{r@{\mkern15mu}*{8}{>{$\bgroup\small$}c<{$\egroup$}}}
        & \macro{0} & \macro{0} & \macro{0} & \macro{0} & \macro{1} &
        \macro{1} & \macro{1} & \macro{1} \\[-0.5ex]
        & \macro{0} & \macro{0} & \macro{1} & \macro{1} & \macro{0} &
        \macro{0} & \macro{1} & \macro{1} \\[-0.5ex]
        & \macro{0} & \macro{1} & \macro{0} & \macro{1} & \macro{0} &
        \macro{1} & \macro{0} & \macro{1} \\[-0.5ex]
        \begin{block}{r@{\mkern15mu}(*{8}{c})}
          \vec{b} = & 1 & 0 & 0 & 1 & 0 & 1 & 0 & 0 \\
        \end{block}
      \end{blockarray}
    \end{equation}
    shows that the reachable global macro states are
    \begin{equation}
      \Macro{\RS} = \{ \macro{0} = ( \loc{\macro{0}}{0},
      \loc{\macro{0}}{1}, \loc{\macro{0}}{2} ),
      \macro{1} = ( \loc{\macro{0}}{0},
      \loc{\macro{1}}{1}, \loc{\macro{1}}{2} ),
      \macro{2} = ( \loc{\macro{1}}{0},
      \loc{\macro{0}}{1}, \loc{\macro{1}}{2} ) \} \text,
    \end{equation}
    where $\macro{0}$ corresponds to the free state of the resource,
    while in $\macro{1}$ and $\macro{2}$, the clients $\LN{1}$ and
    $\LN{0}$ are using the resource, respectively.
  \end{enumerate}
\end{runningExample}

\subsubsection{Block kronecker matrix composition}

The \emph{hierarchical} or \emph{block} Kronecker form of $Q$
expresses the infinitesimal generator of the \CTMC\ over the reachable
state space by the means of macro state decomposition.

The matrices $\loc{Q_t}{j}[\loc{\macro{x}}{j}, \loc{\macro{x}}{j}]$
and $\loc{Q_L}{j}[\loc{\macro{x}}{j}, \loc{\macro{x}}{j}] \in
\RR^{n_{j,x} \times n_{n, y}}$ describe
the effects of a single transition $t \in T$ and the aggregate effects of
local transitions on $\LN{j}$ as its state changes from the local
macro state $\loc{\macro{x}}{j}$ to $\loc{\macro{y}}{j}$,
respectively. Formally,
\begin{gather}
  \loc{q_t}{j}[\loc{\macro{x}}{j},
  \loc{\macro{y}}{j}][\loc{a_x}{j}, \loc{b_y}{j}] = \begin{cases}
    1 & \text{if $\loc{a_x}{j} \tranto{t} \loc{b_y}{j}$,} \\
    0 & \text{otherwise,}
  \end{cases} \label{eq:genstor:explicit:block:tran-matrix}\\
  \loc{Q_L}{j}[\loc{\macro{x}}{j}, \loc{\macro{y}}{j}] = \sum_{t \in
    \loc{T_L}{j}} \Lambda(t) \, \loc{Q_t}{j}[\loc{\macro{x}}{j},
  \loc{\macro{y}}{j}] \label{eq:genstor:explicit:block:local-matrix} \text.
\end{gather}
In the case $j \notin \supp t$, we define $\loc{Q_t}{j}[\loc{\macro{x}}{j},
\loc{\macro{y}}{j}]$ as an identity matrix if $\loc{\macro{x}}{j} =
\loc{\macro{y}}{j}$ and a zero matrix otherwise.

Let us call macro state pairs $(\macro{\vec{x}}, \macro{\vec{y}})$
\emph{single local macro state transitions} (slmst.) at $h$ if $\macro{\vec{x}}$
and $\macro{\vec{y}}$ differ only in a single index $h$
($\loc{\macro{x}}{h} \ne \loc{\macro{y}}{j}$).

The off-diagonal part $Q_D$ of $Q$ is written as a block matrix with
$\macro{n} \times \macro{n}$ blocks. A single block is expressed as
\begin{equation}
  Q_O[\macro{\vec{x}}, \macro{\vec{y}}] = \begin{cases}
    \begin{multlined}[c][7cm]
      \bigkrplus_{j = 0}^{J - 1}
      \loc{Q_L}{j}[\loc{\macro{x}}{j}, \loc{\macro{x}}{j}]\\[-4.5ex]
      + \sum_{t \in T_S} \Lambda(t) \bigkrtimes_{j = 0}^{J - 1}
      \loc{Q_t}{j}[\loc{\macro{x}}{j}, \loc{\macro{x}}{j}]
    \end{multlined}
    & \text{if $\macro{\vec{x}} = \macro{\vec{y}}$,} \\[6.5ex]
    \begin{multlined}[c][7cm]
      I_{N_1 \times N_1} \krtimes
      \loc{Q_L}{h}[\loc{\macro{x}}{h}, \loc{\macro{x}}{h}] \krtimes
      I_{N_12\times N_2} \\[-1.5ex]
      + \sum_{t \in T_S} \Lambda(t) \bigkrtimes_{j = 0}^{J - 1}
      \loc{Q_t}{j}[\loc{\macro{x}}{j}, \loc{\macro{x}}{j}]
    \end{multlined}
    & \text{if
      $(\macro{\vec{x}}, \macro{\vec{y}})$ is slmst.~at $h$,} \\[5ex]
    \displaystyle \sum_{t \in T_S} \Lambda(t) \bigkrtimes_{j = 0}^{J - 1}
      \loc{Q_t}{j}[\loc{\macro{x}}{j}, \loc{\macro{x}}{j}] &
      \text{otherwise,}
  \end{cases}
\end{equation}
where $I_1 = \prod_{f = 0}^{h - 1} n_{h, \loc{x}{h}}$, $I_2 = \prod_{f
  = h + 1}^{J - 1} n_{h, \loc{x}{h}}$. If $\vec{x} = \vec{y}$, the
matrix block describes transition which leave the global macro state
unchanged, therefore any local transition may fire. If
$(\macro{\vec{x}}, \macro{\vec{y}})$ is slmst.~at $h$, only local
transitions on the component $h$ may cause the global state
transition, since no other local transition may affect $\LN{h}$. In
every other case, only synchronizing transitions may occur.

This expansion of block matrices is equivalent to
\vref{eq:genstor:explicit:kronecker} except the considerations to the
hierarchical structure of the state space.

The full $Q$ matrix is written as
\begin{equation}
  Q = Q_O + Q_D, \quad Q_D = -\diag\{ Q_O \vec{1}^\T \}
\end{equation}
as usual.

\begin{algorithm}
  \KwIn{State spaces $\loc{\Macro{\RS}}{j} \loc{\RS_x}{j}$,
    transitions $T$, transition rates $\Lambda$}
  \KwOut{Transition matrices $\loc{Q_t}{j}, \loc{Q_L}{j}$}
  \For{$j \gets 0$ \KwTo $J - 1$}{
    \ForEach{$(\loc{\macro{x}}{j}, \loc{\macro{y}}{j}) \in
      \loc{\Macro{\RS}}{j} \times \loc{\Macro{\RS}}{j}$}{%
      \uIf{$j \in \supp t$}{
        \KwAllocate{$\loc{Q_t}{j}[\loc{\macro{x}}{j},
          \loc{\macro{y}}{j}] \in \RR^{n_{j,x} \times n_{n,y}}$}\;
        Fill in $\loc{Q_t}{j}[\loc{\macro{x}}{j},
          \loc{\macro{y}}{j}]$ according to
          \vref{eq:genstor:explicit:block:tran-matrix}
      }
      \lElseIf{$\loc{\macro{x}}{j} = \loc{\macro{y}}{j}$}{
        $\loc{Q_t}{j}[\loc{\macro{x}}{j}, \loc{\macro{y}}{j}] \gets
        I_{n_{j,x} \times n_{j,y}}$
      }
      \lElse{$\loc{Q_t}{j}[\loc{\macro{x}}{j}, \loc{\macro{y}}{j}]
        \gets 0_{n_{j,x} \times n_{j,y}}$}
    }
    \KwAllocate{$\loc{Q_L}{j}[\loc{\macro{x}}{j},
      \loc{\macro{y}}{j}] \in \RR^{n_{j,x} \times n_{n,y}}$}\;
    \ForEach{$t \in \loc{T_L}{j}$}{
      $\loc{Q_L}{j}[\loc{\macro{x}}{j}, \loc{\macro{y}}{j}] \gets
      \loc{Q_L}{j}[\loc{\macro{x}}{j}, \loc{\macro{y}}{j}] +
      \Lambda(t) \, \loc{Q_t}{j}[\loc{\macro{x}}{j},
      \loc{\macro{y}}{j}]$
    }
  }
  \caption{Transition matrix construction for block Kronecker
    matrices}
  \label{alg:genstor:explicit:block:tranmatrix}
\end{algorithm}

\begin{algorithm}
  \KwIn{State spaces $\Macro{\RS},
    \loc{\Macro{\RS}}{j} \loc{\RS_x}{j}$, transitions $T$, transition
    rates $\Lambda$,\\matrices $\loc{Q_t}{j}, \loc{Q_L}{j}$}
  \KwOut{Infinitesimal generator $Q$}
  \KwAllocate{block matrix $Q$ with $\macro{n} \times \macro{n}$
    blocks}\;
  \ForEach{$(\macro{\vec{x}}, \macro{\vec{y}}) \in \Macro{\RS} \times
    \Macro{\RS}$}{
    Initialize $Q[\macro{\vec{x}}, \macro{\vec{y}}]$ as a linear
    combination of matrices\;
    \uIf{$\macro{\vec{x}} = \macro{\vec{y}}$}{
      \For{$j \gets 0$ \KwTo $J - 1$}{
        \If{$\loc{Q_L}{j}[\loc{\macro{x}}{j}, \loc{\macro{y}}{j}] \ne 0$}{
          $I_1 \gets I_{\prod_{f = 0}^{j - 1} n_{f, \loc{x}{f}} \times
            \prod_{h = 0}^{j - 1} n_{f, \loc{x}{f}}}, \quad
          I_2 \gets\text I_{\prod_{g = j + 1}^{J - 1} n_{f, \loc{x}{f}} \times
            \prod_{f = j + 1}^{J - 1} n_{f, \loc{x}{f}}}$\;
          $Q[\macro{\vec{x}}, \macro{\vec{x}}] \gets Q[\macro{\vec{x}},
          \macro{\vec{x}}] + I_1 \krtimes
          \loc{Q_L}{j}[\loc{\macro{x}}{j}, \loc{\macro{x}}{j}]
          \krtimes I_2$\;
        }
      }
    }
    \ElseIf{$(\macro{\vec{x}}, \macro{\vec{y}})$ is a slmst.~at $h$}{
      $I_1 \gets I_{\prod_{f = 0}^{h - 1} n_{f, \loc{x}{f}} \times
        \prod_{h = 0}^{h - 1} n_{f, \loc{x}{f}}}, \quad
      I_2 \gets\text I_{\prod_{f = f + 1}^{J - 1} n_{f, \loc{x}{f}} \times
        \prod_{f = h + 1}^{J - 1} n_{f, \loc{x}{f}}}$\;
      $Q[\macro{\vec{x}}, \macro{\vec{x}}] \gets Q[\macro{\vec{x}},
      \macro{\vec{x}}] + I_1 \krtimes
      \loc{Q_L}{h}[\loc{\macro{x}}{h}, \loc{\macro{x}}{h}]
      \krtimes I_2$\;
    }
    \ForEach{$t \in T_S$}{
      Initialize $B$ as an empty Kronecker product\;
      $\textit{zeroProduct} \gets \KwFalse$\;
      \For{$j \gets 0$ \KwTo $J - 1$}{
        \uIf{$\loc{Q}{j}[\vec{x}, \vec{y}] = 0$}{
          $\textit{zeroProduct} \gets \KwTrue$\;
          \KwSty{break}\;
        }
        \uElseIf{$\loc{Q}{j}[\vec{x}, \vec{y}]$ is an identity
          matrix}{
          \uIf{the last term of $B$ is an indentity matrix $I_{N, N}$}{
            Enlarge the last term of $B$ to $I_{N n_{j,x} \times N
              n_{j,y}}$
            \label{ln:genstor:explicit:block:construction:extend}\;
          }
          \lElse{
            $B \gets B \krtimes \loc{Q}{j}[\vec{x}, \vec{y}]$
          }
        }
      }
      \lIf{$\neg \textit{zeroProduct}$}{
        $Q[\vec{x}, \vec{y}] \gets Q[\vec{x}, \vec{y}] + \Lambda(t)\,
        B$ \label{ln:genstor:explicit:block:construction:discard}
      }
    }
  }
  \KwAllocate{block vector $\vec{d}$ with $\macro{n}$ blocks}\;
  $\vec{d} \gets -Q \vec{1}^\T$\;
  \lForEach{$\macro{x} \in \Macro{\RS}$}{
    $Q[\macro{x}, \macro{x}] \gets Q[\macro{x}, \macro{x}] + \diag
    \{ \vec{d}[\macro{x}] \}$
    \label{ln:genstor:explicit:block:construction:diagonal}
  }
  \KwRet{$Q$}
  \caption{Block Kronecker matrix construction.}
  \label{alg:genstor:explicit:block:construction}
\end{algorithm}

\Vref{alg:genstor:explicit:block:tranmatrix} shows the construction of
the local transition matrices according to
\cref{eq:genstor:explicit:block:tran-matrix,%
eq:genstor:explicit:block:local-matrix}.

The construction of the block matrix $Q$ is shown in
\vref{alg:genstor:explicit:block:construction}. The formulation from
\cref{eq:genstor:explicit:block:qblock} is optimized in several ways:
\begin{itemize}
\item If a Kronecker product contains a $0$ matrix term, it is itself
  zero, therefore, such products are discarded in line%
  ~\ref{ln:genstor:explicit:block:construction:discard}.
\item For identity matrices $I_{N \times N} \krtimes I_{n \times n}$
  holds. This is exploited in line%
  ~\ref{ln:genstor:explicit:block:construction:extend} to reduce the
  number of terms in the Kronecker producs.
\item Instead of constructing $Q_O$ and $Q_D$ separately, the diagonal
  elements are added to the blocks of $Q$ along its diagonal in line%
  ~\ref{ln:genstor:explicit:block:construction:diagonal}.
\end{itemize}

\section{Symbolic methods}

\subsection{Multivalued decision diagrams}

Multivalued decision diagrams (\textls{MDDs}) \textbf{TODO Cite}
provide a compact, graph-based representation for functions of the
form $\NN^J \to \{0, 1\}$.

\begin{dfn}
  A quasi-reduced ordered \emph{multivalued decision diagram}
  (\textls{MDD}) encoding the function
  $f(\loc{x}{0}, \loc{x}{1}, \ldots, \loc{x}{J - 1}) \in \{0, 1\}$
  (where the domain of each variable $\loc{x}{j}$ is
  $\loc{D}{j} = \{0, 1, \ldots, n_j - 1\}$) is a tuple
  $\MDD = (V, \ddnode{r}, \ddnode{0}, \ddnode{1}, \mddLevel,
  \mddChildren)$, where
  \begin{asparaitem}
  \item $V = \bigcup_{i = 0}^{J} V_i$ is a finite set of \emph{nodes},
    where $V_0 = \{ \ddnode{0}, \ddnode{1} \}$ are the \emph{terminal
      nodes}, the rest of the nodes $V_N = V \setminus V_0$ are
    \emph{nonterminal nodes};
  \item $\mddLevel : V \to \{0, 1, \ldots, J\}$ assigns nonnegative
    \emph{level numbers} to each node
    ($V_i = \{\ddnode{v} \in V : \mddLevel(\ddnode{v}) = i \}$);
  \item $\ddnode{v} \in V_J$ is the \emph{root node};
  \item $\ddnode{0}, \ddnode{1} \in V_0$ are the \emph{zero} and
    \emph{one terminal nodes};
  \item $\mddChildren: \bigl( \bigcup_{i = 1}^J V_i \times \loc{D}{i
      - 1} \bigr) \to V$
    is a function defining edges between nodes labeled by the items of
    the domains, such that either
    $\mddChildren(\ddnode{v}, x) = \ddnode{0}$ or
    $\mddLevel(\mddChildren(\ddnode{v}, x)) = \mddLevel(\ddnode{v}) - 1$ for
    all $\ddnode{v} \in V$, $x \in \loc{D}{\mddLevel(\ddnode{v}) - 1}$,
  \item if $\ddnode{n}, \ddnode{m} \in V_j, j > 0$ then the subgraphs
    formed by the nodes reachable from $\ddnode{n}$ and $\ddnode{m}$
    are either non-isomorphic, or $\ddnode{n} = \ddnode{m}$.
  \end{asparaitem}
\end{dfn}

We remark that due to the presence of the terminal level $V_0$ the
indexing of the levels and the domains is shifted, i.e.~the level
$V_i$ corresponds to the domain $\loc{D}{i - 1}$.

According to the semantics of \textls{MDDs}, $f(\vec{x}) = 1$ if the
node $\ddnode{1}$ is reachable from $\ddnode{r}$ through the edges
labeled with $\loc{x}{0}, \loc{x}{1}, \ldots, \loc{x}{J - 1}$,
\begin{multline}
  f(\loc{x}{0}, \loc{x}{1}, \ldots, \loc{x}{J - 1}) = 1
  \quad\Longleftrightarrow\\
  \mddChildren(\mddChildren(\ldots \mddChildren(\ddnode{r}, \loc{x}{J
    - 1}) \ldots, \loc{x}{1}), \loc{x}{0}) = \ddnode{1} \text.
\end{multline}

\begin{dfn}
  A quasi-reduced ordered \emph{edge-valued multivalued decision
    diagram} (\textls{EDD}) \textbf{TODO Cite} encoding the function
  $g(\loc{x}{0},
  \loc{x}{1}, \ldots, \loc{x}{J - 1}) \in \NN$ is a tuple $\EDD = (V,
  \ddnode{r}, \ddnode{0}, \ddnode{1}, \mddLevel, \mddChildren,
  \eddLabel)$, where
  \begin{asparaitem}
  \item $\MDD = (V, \ddnode{r}, \ddnode{0}, \ddnode{1}, \mddLevel,
    \mddChildren)$ is a quasi-reduced ordered \textls{MDD},
  \item $\eddLabel: \bigl( \bigcup_{i = 1}^J V_i \times \loc{D}{i
      - 1} \bigr) \to \NN$ is an edge label function.
  \end{asparaitem}
\end{dfn}

According to the semantics of \textls{EDDs}, the function $g$ is
evaluated as
\begin{equation}
  g(\vec{x}) = \begin{cases}
    \text{undefined}
    & \text{if $f(\vec{x}) = 0$,} \\
    \sum_{j = 0}^{J - 1} \eddLabel(\loc{\ddnode{n}}{j}, \loc{x}{j})
    & \text{if $f(\vec{x}) = 1$,}
  \end{cases}
\end{equation}
where $f$ is the function associated with the underlying \textls{MDD}
and $\loc{\ddnode{n}}{j}$ are the nodes along the path to
$\ddnode{1}$, i.e.~
\begin{equation}
  \loc{\ddnode{n}}{J - 1} = \ddnode{r}, \quad \loc{\ddnode{n}}{j} =
  \mddChildren(\loc{\ddnode{n}}{j + 1}, \loc{x}{j + 1}) \text.
\end{equation}

\subsection{Symbolic state spaces}

Symbolic techniques involving \textls{MDDs} can efficiently store
large reachable state spaces of superposed Petri nets. Reachable
states $x \in \RS$ are associated with state codings $\vec{x} =
(\loc{x}{0}, \loc{x}{1}, \ldots, \loc{x}{J - 1})$. The function $f:
\PS \to \{0, 1\}$ can be stored as an \textls{MDD} where $f(\vec{x}) =
1$ if and only if $x \in \RS$. The domains of the \textls{MDD} are the
local state spaces $\loc{D}{j} = \loc{\RS}{j}$.

Similarly, \textls{EDDs} can efficiently store the mapping between
symbolic state encodings $\vec{x}$ and reachable state indices $x \in
\RS = \{0, 1, \ldots, n - 1\}$ as the function $g(\vec{x}) = x$. This
mapping is used to refer to elements of state probability vectors
$\vec{\uppi}$ and the sparse generator matrix $Q$ when these objects
are created and accessed.

\begin{figure}
  \centering
  \begin{tikzpicture}[
    ddnode/.style={
      draw,tdk highlight,
      rectangle split,
      rectangle split parts=#1,
      rectangle split horizontal,
      inner sep=5pt
    },
    terminal/.style={
      draw,circle,tdk highlight
    }
    ]

    \matrix [row sep=1.2cm,column sep=1cm,node distance=1cm] {
      \node {$V_3:$};
      & \node [ddnode=2] (x00) {0\nodepart{two}1}; \\
      \node {$V_2:$};
      & \node [ddnode=3] (x10) {0\nodepart{two}1\nodepart{three}2};
      & \node [ddnode=3] (x11) {0\nodepart{two}1\nodepart{three}2}; \\
      \node {$V_1:$};
      & \node [ddnode=3] (x20) {0\nodepart{two}1\nodepart{three}2};
      & \node [ddnode=3] (x21) {0\nodepart{two}1\nodepart{three}2}; \\
      \node {$V_0:$};
      & \node [terminal] (t0) {$\ddnode{0}$};
      & \node [terminal] (t1) {$\ddnode{1}$}; \\
    };

    \begin{scope}[
      every edge/.append style={-{Latex}},
      every node/.append style={above}
      ]
      \draw (x00.one south) edge node [right] {0} (x10);
      \draw (x00.two south) edge node [above] {4} (x11);
      \draw (x10.one south) edge node [left] {0} (x20);
      \draw (x10.two south) edge node [right] {2} (x20);
      \draw (x11.one south) edge node [left,yshift=-5pt,xshift=2pt] {0} (x21);
      \draw (x11.two south) edge node [right] {1} (x21);
      \draw (x11.three south) edge node [left,yshift=5pt] {2} (x20);
      \draw (x20.one south) edge node [below,xshift=-5pt] {0} (t1);
      \draw (x20.two south) edge node [xshift=5pt] {1} (t1);
      \draw (x21.three south) edge node [right] {0} (t1);
    \end{scope}
  \end{tikzpicture}
  \caption{\textls{EDD} state space mapping for the
    \emph{SharedResource} \textls{SSPN}.}
  \label{fig:genstor:symbolic:edd-example}
\end{figure}

\begin{runningExample}
  \Cref{fig:genstor:symbolic:edd-example} shows the state space of the
  \emph{SharedResource} model encoded as an \textls{EDD}. The edge
  labels express the lexicographic mapping of symbolic state codes
  $\vec{x}$ to state indices $\vec{x}$. Edges to the terminal zero
  node $\ddnode{0}$ were omitted for the sake of clarity.
\end{runningExample}

Some algorithms for \textls{MDD} state space exploration are
\emph{breath-first search} \textbf{TODO cite} and \emph{saturation}%
~\citep{Ciardo:2006}. We use the implementation of saturation from the
\textls{Petridotnet} framework~\citep{TDK2010_Darvas,Petridotnet}.

\begin{algorithm}
  \KwIn{node for target state $\ddnode{t}$, node for
    source state $\ddnode{s}$, target state offset $y$, source state
    offset $y$, transition $t$, transition rate $\lambda$,
    matrix $Q_O$}
  \eIf{$\mddLevel(\ddnode{t}) = 0$}{
    \lIf{$\ddnode{t} = \ddnode{1} \land \ddnode{s} = \ddnode{1}$}{%
      $q_O[x, y] \gets q_O[x, y] + \lambda$}
  }{
    $j \gets \mddLevel(\ddnode{t}) - 1$\;
    \ForEach{$\loc{y}{j} \in \loc{RS}{j}$}{
      \lIf{$\mddChildren(\ddnode{t}, \loc{y}{j}) = \ddnode{0}$}{%
        \KwSty{return}
      }
      Find $\loc{x}{j}$ such that $\loc{x}{j} \tranto{t} \loc{y}{j}$\;
      \lIf{$\mddChildren(\ddnode{s}, \loc{x}{j}) = \ddnode{0}$}{%
        \KwSty{return}
      }
      $\textsc{FillIn}(\mddChildren(\ddnode{t}, \loc{y}{j}),
      \mddChildren(\ddnode{s}, \loc{x}{j}),$\\
      $\hphantom{\textsc{FillIn}(}y + \eddLabel(\ddnode{t},
      \loc{y}{j}), x + \eddLabel(\ddnode{t}, \loc{x}{j}), t, \lambda,
      Q_O)$\;
    }
  }
  \caption{\textsc{FillIn} procedure for matrix construction from
    \textls{EDD} state space.}
  \label{alg:genstor:symbolic:fillin}
\end{algorithm}

\begin{algorithm}
  \KwIn{state space \textls{MDD} root $\ddnode{r}$, state space size
    $n$, transitions $T$, transition rate function $\Lambda$}
  \KwOut{generator matrix $Q \in \RR^{n \times n}$}
  \KwAllocate{$Q_O \in \RR^{n \times n}$}\;
  \ForEach{$t \in T$}{
    $\textsc{FillIn}(\ddnode{r}, \ddnode{r}, 0, 0, t, \Lambda(t), Q_O)$\;
  }
  $\vec{d} \gets -Q_O \vec{1}^\T$\;
  \KwRet{$Q_O + \diag\{ \vec{d} \}$}\;
  \caption{Sparse matric construction from \textls{EDD} state space.}
  \label{alg:genstor:symbolic:sparse}
\end{algorithm}

\Vref{alg:genstor:symbolic:fillin,alg:genstor:symbolic:sparse}
illustrate the construction of a generator matrix based on the state
space encoded as \textls{EDDs}. The procedure \textls{FillIn} descends
the \textls{EDD} following a path for the target and the source
state simulatenously. The edge labels, represeting state indices, are
summed on both paths. If a transition $\vec{x} \tranto{t} \vec{y}$ is
found, the matrix element $q_O[x, y]$ corresponding to the summed
indices in increated by the transition rate
$\Lambda(t)$. \Cref{alg:genstor:symbolic:sparse} repeats
\textsc{FillIn} for all transitions.

\subsection{Symbolic hierarchical state space decomposition}

The memory requirements and runtime of
\vref{alg:genstor:explicit:block:macro} may be significantly improved
by the use of symbolic state space storage instead of a bit vector.

To symbolically partition the local states $\loc{\RS}{j}$ into macro
states $\loc{\Macro{\RS}}{j}$, we will use the following notations of above and
below substates from \citet{ciardo2001saturation}:

\begin{dfn}
  The set of \emph{above} substates coded by the node $\ddnode{n}$ is
  \begin{gather}
    \mddAbove(\ddnode{n}) \subseteq \{ (\loc{x}{j + 1}, \loc{x}{j +
      2}, \ldots, \loc{x}{J - 1}) \in \loc{\RS}{j + 1} \times
    \loc{\RS}{j +
      2} \times \cdots \times \loc{\RS}{J - 1} \} \text, \\
    \intertext{such that}
    \vec{x} \in \mddAbove(\ddnode{n})
    \quad\Longleftrightarrow\quad \mddChildren(\mddChildren(\ldots
    \mddChildren(\ddnode{r}, \loc{x}{J - 1}) \ldots ,\loc{x}{j + 2}),
    \loc{x}{j + 1}) = \ddnode{n}
  \end{gather}
  and $j = \mddLevel(\ddnode{n}) - 1$, i.e.~$\mddAbove(\ddnode{n})$ is
  the set of all paths in the \textls{MDD} leading from $\ddnode{r}$
  to $\ddnode{n}$.
\end{dfn}

\begin{dfn}
  The set of \emph{below} substates coded by the node $\ddnode{n}$ is
  \begin{gather}
    \mddBelow(\ddnode{n}) \subseteq \{ (\loc{x}{0}, \loc{x}{1},
    \ldots, \loc{x}{j}) \in \loc{\RS}{0} \times
    \loc{\RS}{1} \times \cdots \times \loc{\RS}{j} \} \text, \\
    \intertext{such that}
    \vec{x} \in \mddBelow(\ddnode{n})
    \quad\Longleftrightarrow\quad \mddChildren(\mddChildren(\ldots
    \mddChildren(\ddnode{n}, \loc{x}{j}) \ldots ,\loc{x}{1}),
    \loc{x}{0}) = \ddnode{1}
  \end{gather}
  and $j = \mddLevel(\ddnode{n}) - 1$, i.e.~$\mddBelow(\ddnode{n})$ is
  the set of all paths in the \textls{MDD} leading from $\ddnode{n}$
  to $\ddnode{1}$.
\end{dfn}

The relation $\loc{\sim}{j}$ over $\loc{\RS}{j}$ can be expressed
with $\mddAbove(\ddnode{n})$ and $\mddBelow(\ddnode{n})$ is a way that
can be handled easily with symbolic techniques.

\begin{obs}
  \label{obs:genstor:symbolic:block:paths}
  The set of states which contain some local state $\loc{x}{j}$ is
  \begin{multline}
    \{\loc{\hat{\vec{z}}}{j} : \vec{z} \in \RS, \loc{z}{j} = \loc{x}{j}\} =\\ \{
    (\vec{b},\vec{a}) : \ddnode{n} \in V_{j + 1},
    \mddChildren(\ddnode{n}, \loc{x}{j}) \ne \ddnode{n}, \vec{b} \in
    \mddBelow(\mddChildren(\ddnode{n}, \loc{x}{j})), \vec{a} \in
    \mddAbove(\ddnode{n}) \} \text.
  \end{multline}
\end{obs}

\begin{proof}
  Any reachable state $\vec{z} \in \RS$ that has
  $\loc{z}{j} = \loc{x}{j}$ is represented by a path in the
  \textls{MDD} that passes through a pair of nodes
  $\ddnode{n} \in V_{j + 1}$ and
  $\mddChildren(\ddnode{n}, \loc{x}{j}) \ne \ddnode{0}$. Therefore,
  some path $\vec{a} \in \mddAbove(\ddnode{n})$ must be followed from
  $\ddnode{r}$ to reach $\ddnode{n}$, then some path
  $\vec{b} \in \mddBelow(\mddChildren(\ddnode{n}, \loc{x}{j}))$ must
  be followed from $\mddChildren(\ddnode{n}, \loc{x}{j})$ to
  $\ddnode{1}$.

  This means all paths from $\ddnode{r}$ to $\ddnode{1}$ containing
  $\loc{x}{j}$ are of the form $(\vec{b}, \loc{x}{j}, \vec{a})$ and
  the converse also holds.
\end{proof}

\begin{figure}
  \centering
  \begin{tikzpicture}[
    ddnode/.style={
      draw,tdk highlight,
      rectangle split,
      rectangle split parts=#1,
      rectangle split horizontal,
      inner sep=5pt
    },
    terminal/.style={
      draw,circle,tdk highlight
    }
    ]

    \matrix [row sep=1.2cm,column sep=1cm,node distance=1cm] {
      \node {$V_3:$};
      & \node [ddnode=2] (x00) {0\nodepart{two}1}; \\
      \node {$V_2:$};
      & \node [ddnode=3] (x10) {0\nodepart{two}1\nodepart{three}2};
      & \node [ddnode=3] (x11) {0\nodepart{two}1\nodepart{three}2}; \\
      \node {$V_1:$};
      & \node [ddnode=3] (x20) {0\nodepart{two}1\nodepart{three}2};
      & \node [ddnode=3] (x21) {0\nodepart{two}1\nodepart{three}2}; \\
      \node {$V_0:$};
      & \node [terminal] (t0) {$\ddnode{0}$};
      & \node [terminal] (t1) {$\ddnode{1}$}; \\
    };

    \begin{scope}[
      every edge/.append style={-{Latex}},
      every node/.append style={above}
      ]
      \draw (x00.one south) edge node [right] {0} (x10);
      \draw [very thick, dashed] (x00.two south) edge node [above] {4} (x11);
      \draw (x10.one south) edge node [left] {0} (x20);
      \draw (x10.two south) edge node [right] {2} (x20);
      \draw (x11.one south) edge node [left,yshift=-5pt,xshift=2pt] {0} (x21);
      \draw (x11.two south) edge node [right] {1} (x21);
      \draw [very thick] (x11.three south) edge node [left,yshift=5pt] {2} (x20);
      \draw [very thick, dotted] (x20.one south) edge node [below,xshift=-5pt] {0} (t1);
      \draw [very thick, dotted] (x20.two south) edge node [xshift=5pt] {1} (t1);
      \draw (x21.three south) edge node [right] {0} (t1);
    \end{scope}
  \end{tikzpicture}
  \caption{The set of all paths having $\loc{x}{1} = \loc{2}{1}$ in
    the \emph{SharedResource} \textls{EDD}.}
  \label{fig:genstor:symbolic:edd-bind-variable}
\end{figure}

\begin{runningExample}
  \Cref{fig:genstor:symbolic:edd-bind-variable} shows all path in the
  \emph{SharedResource} \textls{MDD} with $\loc{x}{1} = \loc{2}{1}$.

  The single path in the set $\mddAbove(\ddnode{n})$ is dashed, while
  paths in the set $\mddBelow(\mddChildren(\ddnode{n}, \loc{2}{1}))$ are
  drawn as dotted edges.
\end{runningExample}

\begin{obs}
  
  If $\ddnode{n}$ and $\ddnode{m}$ are distinct nonterminal nodes of a
  quasi-reduced ordered \textls{MDD},
  $\mddAbove(\ddnode{n}) \cup \mddAbove(\ddnode{m}) = \emptyset$ and
  $\mddBelow(\ddnode{n}) \ne \mddBelow(\ddnode{m})$.
\end{obs}

\begin{proof}
  We prove the statements indirectly. Let
  $\vec{a} \in \mddAbove(\ddnode{n}) \cup \mddAbove(\ddnode{m})$. If we
  follow the path $\vec{a}$ for $\ddnode{r}$, we arrive at
  $\ddnode{n}$, because $\vec{a} \in \mddAbove(\ddnode{n})$. However, we
  also arrive at $\ddnode{m}$, because
  $\vec{a} \in Above(\ddnode{m})$. This is a contradition, since
  $\ddnode{n} \ne \ddnode{m}$, $\mddAbove(\ddnode{n})$ and
  $\mddAbove(\ddnode{m}$ must be disjoint.

  Now suppose that there are $\ddnode{n}, \ddnode{m} \in V_N$ such
  that $\mddBelow(\ddnode{n}) = \mddBelow(\ddnode{m})$. Because the paths
  $\mddBelow(\ddnode{n})$ describe the subgraph reachable from
  $\ddnode{n}$ completely, this means the subgraphs reachable from
  $\ddnode{n}$ and $\ddnode{m}$ are isomorphic. This is impossible,
  because then the \textls{MDD} cannot be reduced, thus
  $\mddBelow(\ddnode{n})$ and $\mddBelow(\ddnode{m})$ must be distinct.
\end{proof}

\begin{obs}
  The relation $\loc{x}{j} \mathbin{\loc{\sim}{j}} \loc{y}{j}$ can be
  expressed as
  \begin{multline}
    \loc{x}{j} \mathbin{\loc{\sim}{j}} \loc{y}{j}
    \quad\Longleftrightarrow\\
    \{ (\ddnode{n}, \mddChildren(\ddnode{n}, \loc{x}{j})) : \ddnode{n}
    \in V_{j + 1} \} = \{ (\ddnode{n}, \mddChildren(\ddnode{n},
    \loc{y}{j})) : \ddnode{n} \in V_{j + 1} \} \text.
  \end{multline}
\end{obs}

\begin{proof}
  Let
  \begin{align}
    X &= \{ \loc{\hat{\vec{z}}}{j} : \vec{z} \in \RS, \loc{z}{j} =
        \loc{x}{j} \},
    & Y &= \{ \loc{\hat{\vec{z}}}{j} : \vec{z} \in \RS, \loc{z}{j} =
        \loc{y}{j} \}.
  \end{align}
  According to \vref{eq:genstor:explicit:block:locsim}, $\loc{x}{j}
  \mathbin{\loc{\sim}{j}} \loc{y}{j}$ if and only if $X = Y$.

  Define
  \begin{align}
    X(\ddnode{n}) &= \{ \vec{b} : (\vec{b}, \vec{a}) \in X, \vec{b} \in
    \mddAbove(\ddnode{n}) \},
    & Y(\ddnode{n}) &= \{ \vec{b} : (\vec{b}, \vec{a}) \in Y, \vec{b} \in
    \mddAbove(\ddnode{n}) \} \text.
  \end{align}
  $X = Y$ holds precisely when
  $X(\ddnode{n}) = Y(\ddnode{n})$ for all $\ddnode{n} \in V_{j + 1}$.
  We may notice that $X(\ddnode{n})$ and $Y(\ddnode{n})$ partition $X$
  and $Y$, respectively, because the $\mddAbove$-sets are disjoint for
  each node.

  According to \cref{obs:genstor:symbolic:block:paths},
  \begin{align}
    X(\ddnode{n}) &= \mddBelow(\mddChildren(\ddnode{n}, \loc{x}{j})),
    & Y(\ddnode{n}) &= \mddBelow(\mddChildren(\ddnode{n}, \loc{y}{j}))
                      \text.
  \end{align}
  Thus, $X(\ddnode{n}) = Y(\ddnode{n})$ if and only if
  $\mddChildren(\ddnode{n}, \loc{x}{j}) = \mddChildren(\ddnode{n},
  \loc{y}{j})$,
  because the $\mddBelow$-sets are distinct for each node.
\end{proof}

\textbf{TODO algoritmus, ref to Buchholz and Kemper}

\section{Matrix storage}

Ref to linear algebra libraries, why not used

\subsection{Simple matrix formats}

Dense, column sparse

Native implementation

Ref to other sparse formats

\subsection{Compound matrices as expression trees}

\begin{figure}
  \centering
  \begin{tikzpicture}
    \matrix [every node/.append style={
      text width=2.1cm,minimum height=1.2cm,align=center,
      draw,tdk highlight
    },column sep=1cm,row sep=0.3cm] {
      \node (block) {Block\\Matrix}; & \node (lin) {Linear\\Combination};
      & \node (kr) {Kronecker\\Matrix}; & \node (sparse) {Sparse\\Matrix}; \\
      & & & \node (id) {Identity\\Matrix}; \\
      & & \node (diag) {Diagonal Matrix}; & \node (vec) {Vector}; \\
    };
    \draw [{Diamond[length=10pt]}-,
    every node/.append style={at end,above,anchor=south east}]
    (block) edge node {$\ast$} (lin) (lin) edge node {$\ast$} (kr)
    (kr) edge node {$\ast$} (sparse)
    (diag) edge node [below,anchor=north east,yshift=-1] {1} (vec);
    \draw [every node/.append style={at end,below,anchor=north east,yshift=-1pt}]
    ($(lin.east)+(0.5cm,0)$) |- node {0..1} (diag)
    ($(kr.east)+(0.5cm,0)$) |- node {$\ast$} (id);
  \end{tikzpicture}
  \caption{Data structure for block Kronecker matrices.}
  \label{fig:genstor:kronecker:datastructure}
\end{figure}
