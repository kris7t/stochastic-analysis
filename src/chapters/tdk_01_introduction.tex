\chapter{Introduction}
\label{chap:introduction}

The growing need for ensuring the correctness of critical systems --~such as
safety-critical, distributed and cloud applications~-- requires the
rigorous analysis of the functional and extra-functional properties.
A large class of typical quantitative questions regarding
dependability and performability are usually addressed by stochastic
analysis.

Recent critical systems are often distributed/asynchronous, leading to
the well-known phenomenon of \emph{state space explosion}. The size
and complexity of such systems often prevents the success of the
analysis due to the high sensitivity to the number of possible
behaviors. In addition, temporal characteristics of the components can
easily lead to huge computational overhead or prevent algorithms from convergence.

Calculation of dependability and performability measures can be
reduced to steady-state and transient solutions of Markovian
models. Various approaches are known in the literature for these
problems differing in the representation of the stochastic behavior of
the models or in the applied numerical algorithms. The efficiency of
these approaches are influenced by various characteristics of the
models, therefore no single best approach is known.

In this paper our goal is to propose a solution for the various
problems occuring in stochastic analysis of complex systems.

The first step in Markovian analysis is the exploration of the state
space, i.e.~the possible behaviors of the system. 
Various algorithms exist for state space exploration. We addressed the
state space traversal problem with the
development of both explicit state algorithms and symbolic
approaches. Explicit state traversal is fast in general and handles
even systems with complex transition functions, while symbolic state
space traversal can handle even huge state spaces. While symbolic approaches
provide an efficient state space exploration and storage technique,
their application to support the vector operations and index
manipulations extensively used by stochastic algorithms is cumbersome.
In this paper we propose a fully symbolic algorithm to bridge the gap between
symbolic state space representation and the data structures
intensively used by our stochastic analysis algorithms.
The new algorithm is introduced to transform the
symbolic state space representation into a decomposed linear algebraic
representation. This approach allows leveraging existing symbolic
techniques, such as the specification of properties with
\emph{Computational Tree Logic}~\paren{\textls{CTL}} expressions.

We introduce the concept of configurable stochastic analysis. We developed a
framework to support the combination of
\begin{itemize}
\item various state space exploration techniques with
\item decomposition algorithms and representation,
  techniques for the stochastic behaviour of the systems 
\item various numerical algorithms to solve the steady-state and
  transient analysis problem,
\item computation of high level measures such as various reward,
  sensitivity and mean time to failure values.
\end{itemize}

Various problems were solved during our work: an approach is
introduced to combine the different matrix representations with
numerical solution algorithms. A diverse set of algorithms are
implemented for steady-state reward and sensitivity analysis,
transient reward analysis and mean-time-to-first-failure analysis of
stochastic models in the \emph{Stochastic Petri
  Net}~\paren{\textls{SPN}} Markov reward model formalism. Several
optimizations and improvements were applied to provide efficient
algorithms. Most of the developed algorithms are parallelized to
exploit the modern multicore architectures. Benchmarks and industrial
case studies are used to evaluate the applicability of our approach.

The analysis framework is integrated into the \textsc{PetriDotNet}
modeling application. More than $78\,000$ unit tests are generated
with a combinatorial interface testing approach to ensure the
correctness of the data structure. To validate the stochastic analysis
pipeline and the implemented algorithms through software redundancy,
$588$ mathematically consistent configurations of the pipeline are
executed and evaluated for several models.

The remainder of this work is structured as follows:
\Cref{chap:background} reviews some preliminaries of the stochastic
analysis of stochastic Petri nets. \Cref{chap:overview} presents the
configurable stochastic analysis pipeline. \Cref{chap:genstor}
describes the available state space exploration and algorithms and
decompositions of stochastic behaviors, including the hierarchical
decomposition algorithm for symbolic state spaces in
\cref{ssec:genstor:symbolic:hierarchical}. \Cref{chap:algorithms}
presents numerical steady-state and transient analysis algorithms and
their implementations in our framework, with special attention to the
homogenous linear equation systems arising from steady-state
analysis. After describing the testing and validation methodologies
applied to our framework in \cref{chap:evaluation} as well as the
benchmark results, we conclude our paper in \cref{chap:conclusion}.

