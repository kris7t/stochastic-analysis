
%
% General
%

\let\Pr\relax
\DeclareMathOperator{\Pr}{\mathbb{P}}
\DeclareMathOperator{\Ex}{\mathbb{E}}
\DeclareMathOperator{\diag}{diag}
\DeclareMathOperator{\Span}{span}
\DeclareMathOperator{\grade}{grade}
\DeclareMathOperator*{\argmin}{arg\,min}
\DeclareMathOperator*{\argmax}{arg\,max}


\newcommand*{\dd}{\mathrm{d}}
\newcommand*{\T}{\mathrm{T}}
\newcommand*{\vecarrow}{}\let\vecarrow\vec
\renewcommand*{\vec}[1]{\bm{\mathrm{#1}}}
\newcommand*{\krtimes}{}\let\krtimes\otimes
\newcommand*{\bigkrtimes}{}\let\bigkrtimes\bigotimes
\newcommand*{\krplus}{}\let\krplus\oplus
\newcommand*{\bigkrplus}{}\let\bigkrplus\bigoplus
\newcommand*{\eltimes}{}\let\eltimes\odot
\newcommand*{\largesigma}{\text{\larger[1]{$\mathsurround=0pt\sigma$}}}

\newcommand*{\mathabbrev}[1]{\textit{\textls{#1}}}

%
% Figures
%

\tikzset{
  tdk highlight/.style={
    fill=white,
    drop shadow={color=black,opacity=0.1,shadow xshift=2pt, shadow yshift=-2pt}
  }
}

%
% Numbers
%

\newcommand*{\RR}{\mathbb{R}} % Reals
\newcommand*{\RRpos}{\RR^{+}} % Positiver reals
\newcommand*{\QQ}{\mathbb{Q}} % Rationals
\newcommand*{\ZZ}{\mathbb{Z}} % Whole numbers
\newcommand*{\NN}{\mathbb{N}} % Naturals
\newcommand*{\NNpos}{\NN^{+}} % Positive whole numbers

%
% Operations
%

\newcommand*{\partialVec}[1]{\partialElement{\vec{#1}}}
\newcommand*{\partialElement}[4]{#1[#2{:}#3{:}#4]}
\newcommand*{\partialMat}[7]{#1[#2{:}#3{:}#4, #5{:}#6{:}#7]}

\newcounter{operation}
\newcommand*{\TheOperationPrefix}{}
\newcommand*{\OperationPrefix}[1]{%
  \global\def\TheOperationPrefix{#1}%
  #1 & & & & & \\}
\newcommand*{\OperationDesc}[1]{\OperationDescNamed{#1}{#1}}
\newcommand*{\OperationSugarDesc}[1]{\OperationDescNamed{#1}{#1*}}
\newcommand*{\OperationDescNamed}[7]{%
  \refstepcounter{operation}%
  \label{operation:\TheOperationPrefix#1}%
  \hspace{1em}#2 & #3 & #4 & #5 & #6 & #7\\}

\newcommand*{\operationname}[2][]{\hyperref[operation:#1#2]{#2}}

\newcommand*{\InPlace}{In-place}
\SetKwComment{OperationComment}{$\triangleright$~}{}
\makeatletter
\def\OperationFormat@Quark{\OperationFormat@Quark}
\newcommand*{\OperationFormat}[1]{%
  \futurelet\@InPlace\OperationFormat@#1\OperationFormat@Quark}
\def\OperationFormat@{%
  \ifx\@InPlace\InPlace\relax\expandafter\@firstoftwo\else
  \expandafter\@secondoftwo\fi
  \OperationFormat@InPlace\OperationFormat@@}
\def\OperationFormat@InPlace\InPlace{\relax\InPlace~\OperationFormat@@}
\def\OperationFormat@@#1\OperationFormat@Quark{\operationname{#1}}
\newcommand*{\Operation}{\tdk@comment\OperationComment\OperationFormat}
\makeatother

%
% UML
%

\tikzset{
  uml class/.style={draw, tdk highlight, inner xsep=5pt,font=\strut\ttfamily},
  uml not a class/.style={font=\strut},
  uml lollipop/.style={draw,circle,tdk highlight,inner sep=3pt},
  uml inheritance/.style={draw,{Triangle[open,fill=white,angle=90:8pt]}-{}}
}

\newcommand{\umlLollipop}[3][4em,2em]{
  \path (#2) edge ++(#1);
  \node at ($(#2)+(#1)$) [uml lollipop,label=0:{#3}] {};
}

%
% Markov chains
%

\newcommand*{\CTMC}{\textls{CTMC}}
\newcommand*{\TFF}{\mathabbrev{TFF}}
\newcommand*{\MTFF}{\mathabbrev{MTFF}}

%
% Petri nets
%

\newcommand*{\PN}{\mathabbrev{PN}}
\newcommand*{\SPN}{\mathabbrev{SPN}}
\newcommand*{\PNI}{\PN_I}
\newcommand*{\RS}{\mathabbrev{RS}}
\newcommand*{\inarc}{\prescript{\bullet}{}}
\newcommand*{\outarc}[1]{#1^\bullet}
\newcommand*{\inharc}{\prescript{\circ}{}}
\newcommand*{\outharc}[1]{#1^\circ}
\newcommand*{\tranto}[1]{\mathbin{[#1\rangle}}
\newcommand*{\reachto}{}\let\reachto\leadsto
\newcommand*{\token}{\ensuremath{\bullet}}

\tikzset{
  petri net place/.style={
    inner sep=0pt,minimum size=0.65cm,draw,circle,tdk highlight
  },
  petri net transition/.style={
    inner sep=0pt,minimum width=0.2cm,minimum height=0.65cm,draw,tdk highlight
  }
}

%
% Stochastic Automata Networks
%

\newcommand*{\SAN}{\mathabbrev{SAN}}

%
% Stochastic Reward Nets
%

\newcommand*{\SRN}{\mathabbrev{SRN}}
\newcommand*{\rateReward}{\mathit{rr}}
\newcommand*{\impulseReward}{\mathit{ir}}

%
% Superposed Stochastic Petri Nets
%

\newcommand*{\SSPN}{\mathabbrev{SSPN}}
\newcommand*{\partitions}{\mathscr{P}}
\newcommand*{\loc}[2]{#1^{(#2)}}
\newcommand*{\LN}{\loc{\mathabbrev{LN}}}
\newcommand*{\LNI}{\loc{\mathabbrev{LN}_I}}
\newcommand*{\PS}{\mathabbrev{PS}}
\DeclareMathOperator{\supp}{supp}
\newcommand*{\macro}{}\let\macro\tilde
\newcommand*{\Macro}{}\let\Macro\widetilde

%
% CTL
%

\newcommand*{\CTL}{\textls{CTL}}
\newcommand*{\KwTrue}{\textrm{true}}
\newcommand*{\KwFalse}{\textrm{false}}
\newcommand*{\ctlAX}{\textbf{\textsf{\textls[30]{AX}}}\,}

%
% MDD
%

\newcommand*{\MDD}{\mathabbrev{MDD}}
\newcommand*{\mddLevel}{\textit{level}}
\newcommand*{\mddChildren}{\textit{children}}
\newcommand*{\ddnode}{}\let\ddnode\underline
\newcommand*{\EDD}{\mathabbrev{EDD}}
\newcommand*{\eddLabel}{\textit{label}}
\newcommand*{\mddAbove}{\mathscr{A}}
\newcommand*{\mddBelow}{\mathscr{B}}

%
% Running example
%

\newcommand*{\runningExamplePetriNet}{%
\matrix [column sep=0.65cm,row sep=0.5cm,label distance=0cm,ampersand replacement=\&] {
\node [petri net place,label={\strut$p_{W_1}$}] (W1) {};
\& \node [petri net transition,label={\strut$t_{a_1}$}] (a1) {};
\& \node [petri net place,label={\strut$p_{S_1}$}] (S1) {};
\& \node [petri net transition,label={\strut$t_{d_1}$}] (d1) {};
\& \node [petri net place,label={\strut$p_{C_1}$}] (C1) {\token};
\& \node [petri net transition,label={\strut$t_{r_1}$}] (r1) {}; \\
\& \& \node [petri net place] (S) {\token}; \& \& \& \\
\node [petri net place,label={-90:\strut$p_{W_2}$}] (W2) {};
\& \node [petri net transition,label={-90:\strut$t_{a_2}$}] (a2) {};
\& \node [petri net place,label={-90:\strut$p_{S_2}$}] (S2) {};
\& \node [petri net transition,label={-90:\strut$t_{d_2}$}] (d2) {};
\& \node [petri net place,label={-90:\strut$p_{C_2}$}] (C2) {\token};
\& \node [petri net transition,label={-90:\strut$t_{r_2}$}] (r2) {}; \\
};
\node [left=0cm of S] {$p_S$};
\draw [-{Latex}] (W1) edge (a1) (S) edge (a1) (a1) edge (S1)
(S1) edge (d1) (d1) edge (S) (d1) edge (C1) (C1) edge (r1)
(W2) edge (a2) (S) edge (a2) (a2) edge (S2)
(S2) edge (d2) (d2) edge (S) (d2) edge (C2) (C2) edge (r2);
\draw [-{Latex},rounded corners] (r1.east) -| ++(0.65cm,1.2cm) -|
($(W1.west)+(-0.65cm,0)$) -- (W1);
\draw [-{Latex},rounded corners] (r2.east) -| ++(0.65cm,-1.2cm) -|
($(W2.west)+(-0.65cm,0)$) -- (W2);
}

\newcommand*{\runningExampleSPN}{%
\runningExamplePetriNet
\node [below=2pt of a1] {$1.0$};
\node [below=2pt of d1] {$0.5$};
\node [below=2pt of r1] {$\frac{1}{\theta[0]} = 1.6$};
\node [above=2pt of a2] {$1.0$};
\node [above=2pt of d2] {$1.1$};
\node [above=2pt of r2] {$\frac{1}{\theta[1]} = 0.8$};
}