\documentclass[a4paper,10pt,twoside,openright]{memoir}
\semiisopage
\checkandfixthelayout

\usepackage[T1]{fontenc}
\usepackage[utf8]{inputenc}
\usepackage{lmodern}

\usepackage{array,booktabs}
\renewcommand*{\arraystretch}{1.2}

\usepackage{amsmath,amssymb,amsthm,mathtools,relsize,stackengine}

\newcommand*{\PN}{\textit{PN}}
\newcommand*{\LN}{\textit{LN}}
\newcommand*{\HN}{\textit{HN}}
\newcommand*{\PI}{\textit{PI}}
\newcommand*{\xto}{}\let\xto\xrightarrow
\newcommand*{\vecarrow}{}\let\vecarrow\vec\let\vec\mathbf
\newcommand*{\EffC}[1]{\mathop{\textnormal{EC}\mathnormal{(#1)}}}
\DeclareMathOperator{\EME}{EME}
\DeclareMathOperator{\SME}{SME}
\DeclareMathOperator{\PriME}{\Pi ME}
\DeclareMathOperator{\HME}{HME}
\DeclareMathOperator{\MME}{MME}
\newcommand*{\MInv}[1]{\tau\vec{#1}}
\DeclareMathOperator{\StrC}{SC}
\DeclareMathOperator{\CasC}{CC}
\DeclareMathOperator{\SCC}{SCC}
\newcommand*{\CCS}{\textit{CCS}}
\newcommand*{\ECS}{\textit{ECS}}

\usepackage{xparse}

\usepackage{microtype}

\usepackage{tikz}
\usetikzlibrary{arrows,calc,positioning,matrix}
\usepackage[compactlambda]{petriTikz}

\pgfdeclarelayer{bg}
\pgfdeclarelayer{connections}
\pgfsetlayers{bg,connections,main}

\tikzset{>=latex'}

\newsubfloat{figure}

\usepackage{hyperref}

\begin{document}

\mainmatter

\chapter{Example Petri Net}

We will now consider the Generalized Stochastic Petri Net $\PN$ in
Figure~\ref{fig:example:pn}~(a). The net contain seven eight places,
two immediate transitions $t_1$ and $t_2$ and five timed transitions.

\begin{figure}
  \hspace*{\fill}
  \parbox[c]{0.55\linewidth}{\centering \begin{tikzpicture}
  \petriP(p1)(0,0){$p_1$}<1>
  \petriT(t1)(2,-0.8){$t_1$}<1>{2}
  \petriT(t2)(-2,-0.8){$t_2$}<1>{1}
  \petriP(p2)(3,-1.5){$p_2$}
  \petriP(p4)(1,-2.4){$p_4$}
  \petriT(t3)(0,-2.4){$t_3$}{1}
  \petriP(p5)(-1,-1.5){$p_5$}<1>
  \petriP(p3)(-3,-1.5){$p_3$}
  \petriT(t4)(2,-2.4){$t_4$}{0.1}
  \petriT(t5)(-2,-2.4){$t_5$}{0.1}
  \petriP(p6)(1,-4.3){$p_6$}
  \petriT(t6)(0,-4.3){$t_6$}{1}
  \petriP(p7)(-1,-4.3){$p_7$}
  \petriT(t7)(-2,-4.3){$t_7$}{2}

  \begin{pgfonlayer}{connections}
    \draw [->]
    (p1) edge (t1)
    (p5) edge (t1)
    (t1) edge (p2)
    (p1) edge (t2)
    (p5) edge (t2)
    (t2) edge (p3)
    (p4) edge (t3)
    (t3) edge (p5)
    (p2) edge (t4)
    (t4) edge (p4)
    (t4) edge (p6)
    (p3) edge (t5)
    (t5) edge (p5)
    (t5) edge (p7)
    (p6) edge (t6)
    (t6) edge (p7)
    (p7) edge (t7)
    (t7) to ++(-1.5,0) |- (p1);
  \end{pgfonlayer}

  \begin{pgfonlayer}{bg}
    \path [rounded corners=5pt,fill=black!10,draw,dashed]
    (-3.8,0.9) rectangle (3.5,-3.2);
    \path [rounded corners=5pt,fill=black!10,draw,dashed]
    (-2.5,-3.4) rectangle (2.5,-5.1);
    \node [anchor=north east] at (3.5,0.9) {$\LN_1$};
    \node [anchor=north east] at (2.5,-3.4) {$\LN_2$};
  \end{pgfonlayer}
\end{tikzpicture}}
  \hfill
  \parbox[c]{0.4\linewidth}{\centering \begin{tikzpicture}
  \petriP(pi1)(0,0){$\PI_1$}<1>
  \petriT(t4)(1.5,-1){$t_4$}
  \petriT(t5)(-1.5,-1){$t_5$}
  \petriP(pi2)(0,-2){$\PI_2$}
  \petriT(t7)(-1.5,-3){$t_7$}

  \begin{pgfonlayer}{connections}
    \draw [->]
    (pi1) edge (t4)
    (t4) edge (pi2)
    (pi1) edge (t5)
    (t5) edge (pi2)
    (pi2) edge (t7)
    (t7) to ++(-0.5,0) |- (pi1);
  \end{pgfonlayer}
\end{tikzpicture}}
  \hspace*{\fill}
  \par\vspace{0.5\onelineskip}
  \hspace*{\fill}
  \parbox[t]{0.55\linewidth}{\subcaption{%
      Partition consisting of regions $\LN_1$ and $\LN_2$}}
  \hfill
  \parbox[t]{0.4\linewidth}{\subcaption{High level structure}}
  \hspace*{\fill}
  \caption{Exaple Generalized Stochastic Petri Net $\PN$ with its
    initial marking $M_0$}
  \label{fig:example:pn}
\end{figure}

\section{Decomposition into High Level and Low Level Nets}

\subsection{Partitioning Into Minimal Regions}

To facilitate Kronecker analysis we partition the net into
\emph{minimal regions}. A set of transitions $T_r \subset T$ defines a
region if $({}^{\bullet, \circ}T_r)^{\bullet, \circ} = T_r$, where
${}^{\bullet,\circ}T_r = {}^{\bullet}T_r \cup {}^{\circ}T_r$ is set
set of places connected to transitions in $T_r$ by input or inhibitor
edges, and $P_r^{\bullet, \circ} = P_r^{\bullet} \cup P_r^{\circ}$ is
the set of transitions connected to places in $P_r$ by input or
inhibitor edges. In other words, the minimum regions partition the set
of transitions $T$ such that no place is shared by two transitions for
different regions as input on inhibitor.

We first partition $T$ into singleton sets $(\{p_i\})_{i = 1}^7$, then
iterate over the places. In the $i$th iteration, all the sets which
share $p_i$ as input (or inhibitor) are merged. We only need to merge
regions in iterations when $p_i$ has at least two adjacent input (or
inhibitor) edges.


\begin{table}
  {\centering
    \begin{tabular}{@{}lll@{}}
      \toprule
      & Action & Regions \\
      \midrule
      & Initialize & $\{t_1\}, \{t_2\}, \{t_3\}, \{t_4\}, \{t_5\},
                     \{t_6\}, \{t_7\}$ \\[1.5ex]
      $p_1$ & Merge $t_1 - t_2$ & $\{t_1, t_2\}, \{t_3\}, \{t_4\}, \{t_5\},
                     \{t_6\}, \{t_7\}$ \\
      $p_2$ & & $\{t_1, t_2\}, \{t_3\}, \{t_4\}, \{t_5\},
                     \{t_6\}, \{t_7\}$ \\
      $p_3$ & & $\{t_1, t_2\}, \{t_3\}, \{t_4\}, \{t_5\},
                     \{t_6\}, \{t_7\}$ \\
      $p_4$ & & $\{t_1, t_2\}, \{t_3\}, \{t_4\}, \{t_5\},
                     \{t_6\}, \{t_7\}$ \\
      $p_5$ & Merge $t_3 - t_5$ & $\{t_1, t_2\}, \{t_3, t_5\}, \{t_4\},
                     \{t_6\}, \{t_7\}$ \\
      $p_6$ & & $\{t_1, t_2\}, \{t_3, t_5\}, \{t_4\},
                     \{t_6\}, \{t_7\}$ \\
      $p_7$ & & $\{t_1, t_2\}, \{t_3, t_5\}, \{t_4\},
                     \{t_6\}, \{t_7\}$ \\[1.5ex]
      $t_1$ & Merge $t_1 - t_4$ & $\{t_1, t_2, t_4\}, \{t_3, t_5\},
                     \{t_6\}, \{t_7\}$ \\
      $t_2$ & Merge $t_2 - t_5$ & $\{t_1, t_2, t_3, t_4, t_5\},
                     \{t_6\}, \{t_7\}$ \\[1.5ex]
      $t_6$ & Merge $t_6 - t_7$ & $\{t_1, t_2, t_3, t_4, t_5\},
                     \{t_6, t_7\}$ \\
      \bottomrule
    \end{tabular}
    \par}
  \caption{Paritioning the set of transitions into regions.}
  \label{tbl:example:partition}
\end{table}

\end{document}