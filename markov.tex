
\documentclass[a4paper,11pt,twoside,openright]{memoir}

\renewcommand*{\baselinestretch}{1.15}
\setbinding{2mm}
\isopage
%% Page layout by Andreas Mathias from memman.pdf
% \setlength{\trimtop}{0pt}
% \setlength{\trimedge}{\stockwidth}
% \addtolength{\trimedge}{-\paperwidth}
% \settypeblocksize{634pt}{448.13pt}{*}
% \setulmargins{4cm}{*}{*}
% \setlrmargins{*}{*}{1.5}
% \setmarginnotes{17pt}{51pt}{\onelineskip}
% \setheadfoot{\onelineskip}{2\onelineskip}
% \setheaderspaces{*}{2\onelineskip}{*}
\checkandfixthelayout

%% Avoid widows and orphans
%\clubpenalty=10000
%\widowpenalty=10000
%\raggedbottom

\usepackage[usenames,dvipsnames]{xcolor}

\usepackage[T1]{fontenc}
\usepackage[utf8]{inputenc}
\usepackage{XCharter}
\usepackage[scaled]{helvet}
%\usepackage{berasans}
\usepackage{beramono}
\usepackage[expert,bitstream-charter,
greekuppercase=italicized,greeklowercase=italicized]{mathdesign}

%% https://groups.google.com/d/msg/comp.text.tex/XULGbUPl5hs/ffvt5EkzpyoJ
\DeclareSymbolFont{xyz}{U}{txmia}{m}{it}
\SetSymbolFont{xyz}{bold}{U}{txmia}{bx}{it}
\DeclareSymbolFontAlphabet{\varmathbb}{xyz}
\makeatletter
\DeclareMathSymbol{\m@thbbch@rA}{\mathord}{xyz}{129}
\DeclareMathSymbol{\m@thbbch@rB}{\mathord}{xyz}{130}
\DeclareMathSymbol{\m@thbbch@rC}{\mathord}{xyz}{131}
\DeclareMathSymbol{\m@thbbch@rD}{\mathord}{xyz}{132}
\DeclareMathSymbol{\m@thbbch@rE}{\mathord}{xyz}{133}
\DeclareMathSymbol{\m@thbbch@rF}{\mathord}{xyz}{134}
\DeclareMathSymbol{\m@thbbch@rG}{\mathord}{xyz}{135}
\DeclareMathSymbol{\m@thbbch@rH}{\mathord}{xyz}{136}
\DeclareMathSymbol{\m@thbbch@rI}{\mathord}{xyz}{137}
\DeclareMathSymbol{\m@thbbch@rJ}{\mathord}{xyz}{138}
\DeclareMathSymbol{\m@thbbch@rK}{\mathord}{xyz}{139}
\DeclareMathSymbol{\m@thbbch@rL}{\mathord}{xyz}{140}
\DeclareMathSymbol{\m@thbbch@rM}{\mathord}{xyz}{141}
\DeclareMathSymbol{\m@thbbch@rN}{\mathord}{xyz}{142}
\DeclareMathSymbol{\m@thbbch@rO}{\mathord}{xyz}{143}
\DeclareMathSymbol{\m@thbbch@rP}{\mathord}{xyz}{144}
\DeclareMathSymbol{\m@thbbch@rQ}{\mathord}{xyz}{145}
\DeclareMathSymbol{\m@thbbch@rR}{\mathord}{xyz}{146}
\DeclareMathSymbol{\m@thbbch@rS}{\mathord}{xyz}{147}
\DeclareMathSymbol{\m@thbbch@rT}{\mathord}{xyz}{148}
\DeclareMathSymbol{\m@thbbch@rU}{\mathord}{xyz}{149}
\DeclareMathSymbol{\m@thbbch@rV}{\mathord}{xyz}{150}
\DeclareMathSymbol{\m@thbbch@rW}{\mathord}{xyz}{151}
\DeclareMathSymbol{\m@thbbch@rX}{\mathord}{xyz}{152}
\DeclareMathSymbol{\m@thbbch@rY}{\mathord}{xyz}{153}
\DeclareMathSymbol{\m@thbbch@rZ}{\mathord}{xyz}{154}
\DeclareMathSymbol{\varBbbk}{\mathord}{xyz}{171}
\long\def\DoLongFutureLet #1#2#3#4{%
  \def\@FutureLetDecide{#1#2\@FutureLetToken
    \def\@FutureLetNext{#3}\else
    \def\@FutureLetNext{#4}\fi\@FutureLetNext}
  \futurelet\@FutureLetToken\@FutureLetDecide}
\def\DoFutureLet #1#2#3#4{\DoLongFutureLet{#1}{#2}{#3}{#4}}
\def\@EachCharacter{\DoFutureLet{\ifx}{\@EndEachCharacter}%
  {\@EachCharacterDone}{\@PickUpTheCharacter}}
\def\m@keCharacter#1{\csname\F@ntPrefix#1\endcsname}
\def\@PickUpTheCharacter#1{\m@keCharacter{#1}\@EachCharacter}
\def\@EachCharacterDone \@EndEachCharacter{}
\DeclareRobustCommand*{\varmathbb}[1]{\gdef\F@ntPrefix{m@thbbch@r}%
  \@EachCharacter #1\@EndEachCharacter}
\makeatother
\let\mathbb\varmathbb

\setsecheadstyle{\Large\sffamily\bfseries}
\setsubsecheadstyle{\large\sffamily\bfseries}
\setsubsubsecheadstyle{\normalsize\sffamily\bfseries}
\setparaheadstyle{\normalsize\sffamily\bfseries}
\setsubparaheadstyle{\normalsize\sffamily\bfseries}

\copypagestyle{my}{ruled}
\makeevenhead{my}{\thepage}{\itshape\textls{\MakeUppercase{\leftmark}}}{}
\makeoddhead{my}{}{\itshape\rightmark}{\thepage}
\makeevenfoot{my}{}{}{}
\makeoddfoot{my}{}{}{}
\makeatletter
\makepsmarks{my}{%
  \def\chaptermark##1{%
    \markboth{##1}{}}%
  \def\sectionmark##1{%
    \markright{%
      \ifnum\c@secnumdepth > \z@
      \thesection. \ %
      \fi
      ##1}}}
\makeatother
\pagestyle{my}

\renewcommand*{\chapnamefont}{\normalfont\huge\textit}
\renewcommand*{\chapnumfont}{\normalfont\huge\textit}
\renewcommand*{\chaptitlefont}{\normalfont\Huge\sffamily\bfseries}

\usepackage{array,booktabs,mdwlist}
\renewcommand*{\arraystretch}{1.2}
\usepackage{xspace}

\usepackage{amsmath,relsize,stackengine,mleftright}
\usepackage[ntheorem]{empheq}
\usepackage{upgreek}

%% http://tex.stackexchange.com/a/100151/8744
\makeatletter
\def\tagform@#1{\maketag@@@{\footnotesize(#1)\@@italiccorr}}
\makeatother

\DeclareMathOperator{\E}{\mathbb{E}}
\newcommand*{\PN}{\textit{PN}}
\newcommand*{\SPN}{\textit{SPN}}
\newcommand*{\GSPN}{\textit{GSPN}}
\newcommand*{\LN}{\textit{LN}}
\newcommand*{\HN}{\textit{HN}}
\newcommand*{\EN}{\textit{EN}}
\newcommand*{\PI}{\textit{PI}}
\newcommand*{\xto}{}\let\xto\xrightarrow
\newcommand*{\vecarrow}{}\let\vecarrow\vec
\renewcommand*{\vec}[1]{\boldsymbol{\mathrm{#1}}}
\newcommand*{\EffC}[1]{\mathop{\textnormal{EC}\mathnormal{(#1)}}}
\DeclareMathOperator{\EME}{EME}
\DeclareMathOperator{\SME}{SME}
\DeclareMathOperator{\PriME}{\Pi ME}
\DeclareMathOperator{\HME}{HME}
\DeclareMathOperator{\MME}{MME}
\newcommand*{\MInv}[1]{\tau\vec{#1}}
\DeclareMathOperator{\StrC}{SC}
\DeclareMathOperator{\CasC}{CC}
\DeclareMathOperator{\SCC}{SCC}
\newcommand*{\CCS}{\textit{CCS}}
\newcommand*{\ECS}{\textit{ECS}}
\newcommand*{\RS}{\textit{RS}}
\newcommand*{\TRS}{\textit{TRS}}
\newcommand*{\PS}{\textit{PS}}
\newcommand*{\lS}[1]{S^{(#1)}}
\newcommand*{\macroS}{}\let\macroS\tilde
\newcommand*{\macroStates}{}\let\macroStates\widetilde
\newcommand*{\macroTRS}{\macroStates{\TRS}}
\DeclareMathOperator{\diag}{diag}
\DeclareMathOperator{\level}{level}
\DeclareMathOperator{\Env}{Env}
\DeclareMathOperator{\Above}{Above}
\DeclareMathOperator{\Bag}{Bag}
\newcommand*{\To}{\textbf{to}\xspace}
\DeclareMathOperator{\dirplus}{\mathbin{\overline{\oplus}}}
\newcommand*{\leftTok}[1]{{}^\bullet #1}
\newcommand*{\leftInh}[1]{{}^\circ #1}
\newcommand*{\rightTok}[1]{#1{}^\bullet}
\newcommand*{\rightInhk}[1]{#1{}^\circ}

%% http://tex.stackexchange.com/a/13035/8744
\newcommand*{\textabbrev}[1]{\textls[50]{\textsc{#1}}}
\newcommand*{\Ctmc}{\textabbrev{Ctmc}}
\newcommand*{\ctmc}{\textabbrev{ctmc}}
\newcommand*{\Gspn}{\textabbrev{Gspn}}
\newcommand*{\gspn}{\textabbrev{gspn}}
\newcommand*{\Spn}{\textabbrev{Spn}}
\newcommand*{\spn}{\textabbrev{spn}}

\usepackage{xparse}

\usepackage{microtype}

\usepackage{tikz}
\usetikzlibrary{arrows,calc,positioning,shapes,matrix}
\usepackage[compactlambda]{petriTikz}

\pgfdeclarelayer{bg}
\pgfdeclarelayer{connections}
\pgfsetlayers{bg,connections,main}

\tikzset{>=latex'}

\newsubfloat{figure}

\usepackage[
  hyperfootnotes=false,
  breaklinks=true,
  hypertexnames=true,
  plainpages=false,
  pdfpagelabels=true,
  hidelinks
]{hyperref}

\newcommand{\paren}[1]{\textup(#1\textup)}

\let\openbox\relax
\usepackage[amsmath,thmmarks,thref,hyperref]{ntheorem}
\theoremheaderfont{\normalfont\sffamily\bfseries}
\makeatletter
% Lifted from ntheorem.sty
\newcounter{proof}
\newcounter{currproofctr}
\newcounter{endproofctr}
\newcommand*{\openbox}{\leavevmode
  \hbox to.77778em{%
    \hfil\vrule
  \vbox to.675em{\hrule width.6em\vfil\hrule}%
  \vrule\hfil}}
\gdef\proofSymbol{\openbox}
\newcommand{\proofname}{Proof}
\newenvironment{proof}[1][\proofname]{%
  \th@nonumberplain
  \def\theorem@headerfont{\itshape}%
  \normalfont
  \theoremsymbol{\ensuremath{_\blacksquare}}%
  \@thm{proof}{proof}{#1}%
}{\@endtheorem}
\newtheoremstyle{my}%
{\item[\hskip\labelsep \theorem@headerfont ##1~##2\theorem@separator]}%
{\item[\hskip\labelsep {\theorem@headerfont ##1~##2} \paren{##3}\theorem@separator]}
\makeatother
\theoremstyle{my}
\theoremseparator{.}
\newtheorem{thm}{Theorem}[chapter]
\newtheorem{prop}[thm]{Proposition}
\newtheorem{dfn}[thm]{Definition}

\usepackage{algorithm,algpseudocode}

\usepackage[
  style=authoryear-comp,
  maxnames=2,
  maxbibnames=99,
  uniquename=init,
  uniquelist=false,
  backref=false,
  doi=false,
  isbn=false,
  eprint=false,
  natbib=true,
  hyperref=true,
  autolang=hyphen,
  backend=biber
]{biblatex}
\bibliography{markov}
\AtEveryBibitem{% Clean up the bibtex rather than editing it
  \clearlist{address}
  \clearfield{date}
  \clearfield{issn}
  \clearlist{location}
  \clearfield{month}
  \clearfield{series}
  \ifentrytype{book}{}{% Remove publisher and editor except for books
    \ifentrytype{inbook}{}{
      \clearlist{publisher}
      \clearname{editor}
    }
  }
  \ifentrytype{article}{
    \clearfield{url}}{}
  \ifentrytype{inproceedings}{
    \clearfield{url}}{}
}
\DefineBibliographyStrings{english}{%
  bibliography = {References}
}
\renewcommand*{\bibopenparen}{\textup(}
\renewcommand*{\bibcloseparen}{\textup)}

\begin{document}

\mainmatter

\chapter{Background}

\section{Continous-time Markov chains}

\begin{dfn}
  A \emph{continous-time Markov chain} \paren{\emph{\ctmc}}
  $\{X(t) \in S : t \ge 0\}$ over the finite or countably infinite
  state space $S = \{s_1, s_2, \ldots\}$ is a stochastic process which
  satisfies the Markovian property
  \begin{equation}
    \Pr \{ X(t') = s_{j} \mid X(t) = s_{i}, \forall u \in U : X(u) \}
    = \Pr \{ X(t') = s_{j} \mid X(t) = s_{i} \}
  \end{equation}
  for all $t' > t$ and $U \subseteq [0, t)$.
\end{dfn}

\begin{dfn}
  A continous-time Markov chain is \emph{(time-)homogenous} if its
  transition probabilities are invariant with respect to time, i.e.
  \begin{equation}
    \Pr \{ X(t') = s_{j} \mid X(t) = s_{i} \} = p(t' - t)[i, j] \text.
  \end{equation}
\end{dfn}

We will assume that the \ctmc s we consider are homogenous and have a
finite state space, unless otherwise indicated.

Let us introduce to shorthand $\vec{\uppi}(t)$ to denote the vector of
probabilities
\begin{equation}
  \pi(t)[i] = \Pr\{X(t) = s_i\} \text,
\end{equation}
which will be called the \emph{probablity vector} at time $t$.

The row-stochastic matrix
\begin{equation}
  P(\tau) = \bigl(p(\tau)[i, j]\bigr)_{i, j = 1}^{\lvert S \rvert}
  \label{eq:intro:p_tau}
\end{equation}
describes the state transitions of the \ctmc\ over a time period
$\tau$. More concretely,
\begin{equation}
  \vec{\uppi}(t') = \vec{\uppi}(t) P(t' - t) \text.
\end{equation}

If $X(t)$ is a homogenous \ctmc\ over a finite state space, there exists
an \emph{infinitesimal generator matrix}
$Q \in \mathbb{R}^{\lvert S \rvert \times \lvert S \rvert}$ such that
\begin{equation}
  P(\tau) = \exp(\tau Q) = \sum_{k = 1}^\infty \frac{\tau^k Q^k}{k!} =
  I + \tau Q + o(\tau) \text,
\end{equation}
where $\exp(\cdot)$ denotes the matrix exponential function. This
matrix satisfies
\begin{enumerate}
\item If $i \ne j$, $q[i, j] \ge 0$;
\item
  $q[i, i] = -\bigl( q[i, 1] + q[i, 2] + \cdots + q[i, i - 1] + q[i, i + 1] +
  \cdots + q[i, \lvert S \rvert] \bigr)$
  for all $0 \ge i \ge \lvert S \rvert$, that is,
  $Q \vec{e}^T = \vec{0}^T$.
\end{enumerate}
Matricies having these two properties are called Q-matrices.

The \emph{sojourn time} from state $s_j$
\begin{equation}
  T(i) = \inf_{t > 0} \{ X(t) \ne s_i \mid X(0) = s_i \}
\end{equation}
has the memoryless property
\begin{equation}
  \Pr\{T(i) > t + \epsilon \mid T(i) > t\} = \Pr\{T(i) > \epsilon\} \text.
\end{equation}
and is exponentially distributed with rate $1 / \E[T(i)] = -q(i,
i)$.

Moreover, if a state change occurs at time $t$, that is,
$X(t-) \ne X(t+)$, then $X(t+) = s_j$ given $X(t-) = s_i$ with
probability $-q[i, j] / q[i, i]$. Thus we can use the matrix $Q$ to
visualize the behaviour of the \ctmc\ in state $s_i$ as a sojourn in
$s_i$ with exponentially distributed duration followed by a random
jump to some state $s_j \ne s_i$.

The configuration above is equivalent to a ``race'' between $n - 1$
exponentially distributed delays with rates
$q[i, 1], q[i, 2], \ldots, q[i, i - 1], q[i, i + 1], \ldots, q[i,
\lvert S \rvert]$, respectively, where the
first delay to complete causes the \ctmc\ to shift to the destination
state associated with the delay. The latter characterization turns out
to be the key intution for modelling asynchronous distributed systems
with continous-time Markov chains.

The time evolution of the probability vector satisfies the
differential equation
\begin{equation}
  \frac{d \vec{\uppi}(t)}{dt} = \vec{\uppi}(t) Q \text.
  \label{eq:intro:diffeq}
\end{equation}
This gives us a third way of understanding the structure of the
infintesimal generator matrix: the value $q[i, j]$ corresponds to a
``flow amount'' of probability mass from the state $s_i$ to
$s_j$. Off-diagonal entries $q[i, j]$ of $Q$ have positive flow amount
if the \ctmc\ can eventually leave $s_i$ towards $s_j$ and zero
otherwise, while negative entries $q[i, i]$ along a diagonal
correspond to ``leaking'' of probability mass from $s_i$ to other
states. A zero diagonal entry represents a final state which cannot be
left.

\begin{dfn}
  The state $s_j \in S$ of the \ctmc\ $X(t)$ is \emph{accessible} from
  $s_i \in S$ \paren{written as $s_i \to s_j$} if $p(\tau)[i, j] > 0$ for
  some $\tau > 0$.
\end{dfn}

\begin{dfn}
  The states $s_i, s_j \in S$ of the \ctmc\ $X(t)$ \emph{communicate}
  \paren{written as $s_i \leftrightarrow s_j$} if $s_i \to s_j$ and
  $s_j \to s_i$.
\end{dfn}

\begin{dfn}
  The \ctmc\ $X(t)$ is \emph{irreducible} if all of its states pairwise
  communicate.
\end{dfn}

If $X(t)$ is irreducible, then its infinitesimal generator matrix $Q$
has rank $\lvert S \rvert - 1$.

\subsection{Analysis tasks}

The two tasks most frequently associated with continous-time Markov
chains are \emph{transient} and \emph{steady-state} analysis.

In transient problems, the probability distribution vector
$\vec{\uppi}(t)$ at time $t$ is sought given the initial probability
distribution vector $\vec{\uppi}(0) = \vec{\uppi}_0$. In practice,
instead of directly calculating the matrix exponential
\eqref{eq:intro:p_tau}, either the initial value problem based on
\eqref{eq:intro:diffeq}
\begin{equation}
  \frac{d \vec{\uppi}(t)}{dt} = \vec{\uppi}(t) Q \text{ subject
    to } \vec{\uppi}(0) = \vec{\uppi}_0
\end{equation}
is solved, or the \emph{uniformization algorithm}
\begin{equation}
  \vec{\uppi}(t) = \sum_{k = 1}^{\infty} \vec{\uppi}_0 \mleft( I +
  \frac{1}{\gamma} Q \mright)^k \frac{(\gamma t)^k}{k!} e^{-\gamma t}
\end{equation}
is used with some appropriately chosen $\gamma > 0$.

In steady-state analysis, the limit
\begin{equation}
  \vec{\uppi}_{\infty} = \lim_{t \to \infty} \vec{\uppi}(t)
\end{equation}
is computed. If the \ctmc\ is irreducible, this limit is well-defined
and independent of the choice of the initial probability distribution
$\vec{\uppi}(0)$.

The equlibirium condition
\begin{equation}
  \frac{d \vec{\uppi}}{dt} = 0
\end{equation}
together with \eqref{eq:intro:p_tau} yields the homogenous system of
linear equations
\begin{equation}
  \label{eq:into:steadystate}
  \vec{\uppi}_\infty Q = \vec{0} \text{ subject to } \sum_{i =
    0}^{\lvert S \rvert} \pi_\infty[i] = 1 \text{ and } \forall 1 \le
  i \le \lvert S \rvert :  \pi_\infty[i] > 0
\end{equation}
for finding the steady-state probability distribution vector. In other
words, $\vec{\uppi}_\infty$ is the (unique) vector in the null space
of $Q$ which is a valid probability distribution vector, that is,
whose components are all nonnegative and sum to $1$.

Transient analysis is further discussed in Chapter~TODO, while
steady-state analysis is further discussed in Chapter~TODO.

While these two task primitives are often useful on their own, more
complicated inquiries, such as model checking for stochastic temporal
logical formulae also use them as subroutines. Such analysis methods
are discussed in Chapter~TODO.

\section{Petri nets}

\begin{dfn}
  A \emph{Petri net} is a tuple
  \begin{equation}
    \PN = (P, T, \Pi, W^{-}, W^{+}, W^{H}, M_0) \text,
  \end{equation}
  where
  \begin{itemize}
  \item $P$ and $T$ are finite set of \emph{places} and
    \emph{transitions}, respectively;
  \item $\Pi: P \to \mathbb{N}$ is the \emph{priority function}, which
    map places to nonnegative integer priorities;
  \item $W^{-}, W^{+}, W^{H} : T \to \Bag(P)$ are the \emph{input,
      output} and \emph{inhibitor arc functions}, which map
    transitions to ``bags'' \paren{multisets} of places;
  \item $M_0 \in \Bag(P)$ is the \emph{initial marking}.
  \end{itemize}
\end{dfn}

For convenience, we will write $w^-[j, i]$, $w^+[j, i]$ and
$w^H[j, i]$ for the multiplicity of the place $p_i \in P$ in
$W^-(t_j)$, $W^+(t_j)$ and $W^H(t_j)$, respectively, where $t_j \in T$
is a transition. A \emph{marking} is a multiset of places
$M \in \Bag(P)$. We write $m(i)$ for the multiplicity of $p_i$ in the
marking $M$. The set of \emph{input places} $\leftTok{t_j}$,
\emph{output places} $\rightTok{t_j}$ and \emph{inhibitor places}
$\leftInh{t_j}$ of a transition $t_j$ are defined as
\begin{equation}
  \begin{gathered}
    \begin{aligned}
      \leftTok{t_j} &= \{ p_i \in P : w^-[j, i] > 0 \}\text, &
      \rightTok{t_j} &= \{ p_i \in P : w^+[j, i] > 0 \}\text,
    \end{aligned} \\
    \leftInh{t_j} = \{ p_i \in P : w^H[j, i] > 0 \}\text,
  \end{gathered}
\end{equation}
respectively.

\subsection{Token game}

\begin{dfn}
  The transition $t_j$ has \emph{concession} in the marking $M$ if
  \begin{equation}
    \forall p_i \in \leftTok{t_j} : m(i) \ge w^-[j, i] \text{ and }
    \forall p_i \in \leftInh{t_j} : m(i) < w^H[j, i] \text.
  \end{equation}
\end{dfn}

\begin{dfn}
  The transition $t_j$ is \emph{enabled} in the marking $M$ if it has
  concession in $M$ and no transition $t_{k}$ has concession in $M$
  such that $\Pi(t_{k}) > \Pi(t_j)$.
\end{dfn}

If the transition $t_j$ is enabled in $M$, it can \emph{fire} to yield
another marking
\begin{equation}
  M' = M - W^-(T_j) + W^+(T_j)
\end{equation}
which is written as $M \xrightarrow{t_j} M'$. We can visualize the
marking $M$ as a bag of ``tokens'' located on some of the places
$P$. The firing of $t_j$ removes $W^-(t_j)$ tokens from
$\leftTok{t_j}$ and places $W^{+}(t_j)$ tokens to $\rightTok{t_j}$,
but only if there are less than $W^H(t_j)$ tokens on $\leftInh{t_j}$
and no higher priority transition has concession.

\begin{dfn}
  The marking $M'$ is \emph{reachable} from the marking $M$ if there
  exists a \paren{possibly empty} sequence of transitions
  $t_{i_1}, t_{i_2}, \ldots, t_{i_n}$ such that
  \begin{equation}
    M \xrightarrow{t_{i_1}} M_1 \xrightarrow{t_{i_2}} M_2
    \xrightarrow{t_{i_3}} \cdots \xrightarrow{t_{i_{n - 1}}} M_{n - 1}
    \xrightarrow{t_{i_n}} M_n = M'\text.
  \end{equation}
\end{dfn}

\begin{dfn}
  The \emph{reachability set} $\RS$ of a Petri net is the set of all
  markings reachable from the initial marking $M_0$.
\end{dfn}

The rechability set may contain an enormous amount of markings even if
the Petri net is of modest size. This phenomenon is called \emph{state
  space explosition} \textbf{TODO citation?}, which is a significant
obstacle in the analysis of Petri nets. The maximum size of the models
which can be analyzed using a given pool of resources may be incresead
by chosing an efficient representation of the reachability set instead
of explicit storage of all reachable markings. \textbf{TODO citation!}

\section{Stochastic Petri nets}

\begin{dfn}
  A \emph{Generalized Stochastic Petri Net} \paren{\emph{\gspn}} is a
  tuple
  \begin{equation}
    \GSPN = (P, T, \Pi, \Lambda, W^-, W^+, W^H, M_0) \text,
  \end{equation}
  where $(P, T, \Pi, W^-, W^+, W^H, M_0)$ is a Petri net and $\Lambda:
  T \to \mathbb{R}^{+}$ is called the \emph{rate} or \emph{weight}
  function.
\end{dfn}

Transitions with $\Pi = 0$ are called \emph{timed} transitions, while
transitions with $\Pi \ge 1$ are called \emph{immediate}
transitions. The value $\Lambda(t_j)$, abbreviated as $\lambda_j$, is
referred to as the \emph{rate} associated with $t_j$ if $t_j$ is
timed or the \emph{weight} of $t_j$ if $t_j$ is immediate.

\begin{dfn}
  A marking $M$ is \emph{vanishing} if there exists an immediate
  transition with concession. Non-vanishing markings are called
  \emph{tangible}.
\end{dfn}

The \emph{tangible reachability set} $\TRS \subseteq \RS$ is the set
of all reachable tangible states. The dynamics of a \gspn\ are
described with by continous-time Markov chain $X(t)$ whose set of
states $S$ is the tangible reachability set $\TRS$ and the initial
state $X(0)$ is the initial marking $M_0$. The states
$S = \{s_0, s_1, s_2, \ldots\}$ and tangible reachable markings
$\TRS = \{M_0, M_1, M_2, \ldots\}$ will be used interchangeably in the
rest of this report, specifically, we will assume that the state $s_0$
represents the initial marking $M_0$ in the \ctmc.

In tangible states, timed transitions that are enabled fire with
delays exponentially distributed according to their $\lambda$
rates. If the resulting marking is vanishing, the enabled immediate
transitions fire until a tangible marking is reached, which will
become the next state of the \ctmc. If multiple immediate transitions
are enabled in a vanishing marking, one is selected and fired randomly
with probability proportional to its $\lambda$ weight. Thus, state
changes of the \ctmc\ corresponding to the \gspn\ consist of the
firing of a timed transition followed by a -- possibly empty, if the
output marking of the timed transition is tangible -- random sequence
of immiditate transition firings. \textbf{TODO Example trace of a
  timed and immediate firing.}

Let $p_{\text{immediate}}(M_i, M_j)$ denote the probabilty that the
\gspn\ reaches the tangible marking $M_j$ starting from the tangible
or transient marking $M_i$ by a possible empty sequence of immediate
transition firings. Observe that when $M_i$ is tangible,
$p_{\text{immediate}}(M_i, M_j) = 1$ if $i = j$ and $0$ otherwise.

The behaviour of the \gspn\ is well-defined if
$p_{\text{immediate}}(M, M')$ is well-defined for all $M \in \RS$,
$M' \in \TRS$ \citep{DBLP:journals/tse/TeruelFP03} and the net is free
of \emph{confusion}. We also require the length of immediate firing
sequences from vanishing to tangible markings be bounded, i.e.~no
limits are needed to be evaluated to calculate
$p_{\text{immediate}}$. \textbf{TODO Example of a non-well-defined
  net.}

The infinitesimal generator of a \gspn\ can be written as
\begin{gather}
  q_{O}[i, j] = \sideset{}{'}\sum_{M_i \xrightarrow{t_k} M'} \lambda_k
  p_{\text{immedidate}}(M', M_j)
  \text, \\
  Q = Q_{O} + Q_{D} = Q_{O} - \diag Q_o \vec{e}^T \text,
\end{gather}
where the summation is performed over all transitions $t_k$ enabled in
the state $M_i$, while $Q_O$ and $Q_D$ stand for the off-diagonal and
diagonal parts of $Q$, respectively.

The storage of $Q$ for \gspn\ is particularly challenging due to state
space explosition, because the number of entries, therefore the amount
of memory required for the explicit storage of the inifinitesimal
generator matrix, grows quadratically in the number of tangible
reachable markings. This behaviour is opposed to the amount of memory
required for explicit storage of $\TRS$, which grows merely
linearly. Therefore, efficient representation of $Q$ becomes a problem
even in analysis task where explicit storate of $\TRS$ (or $\RS$) is
possible.

\subsection{Conflict and confusion}

A pair of immediate transitions $t_1$, $t_2$ cannot fire
``simultaneously'' if only one of traces
$M \xrightarrow{t_1} M' \xrightarrow{t_2} M'''$ and
$M \xrightarrow{t_2} M'' \xrightarrow{t_1} M'''$ is possible. In this
case, we must evaluate the effects of $t_1$ and $t_2$ separately. This
notion is formalized as the effective conflict relation, which is
noncommutative, unreflexive and intransitive due to the presence on
priority levels and inhibitor arcs.

\begin{dfn}[\cite{DBLP:journals/tse/ChiolaMBC93}]
  The transitions $t_1$ and $t_2$ are in \emph{effective conflict} in
  the marking $M$ \paren{written as $t_1 \EffC{M} t_2$} if and only if
  $t_2$ has concession in $M$, $t_1$ is enabled in $M$, $M
  \xrightarrow{t_1} M'$ and $t_2$ has no concession in $M'$.
\end{dfn}

If $t_1$ and $t_2$ are not in conflict, but the final outcome of the
sequence of immediate firings depends on the order in which $t_1$ and
$t_2$ are fired, the \gspn\ is in \emph{confusion}. \textbf{TODO Petri
  net with confusion.} More formally:

\begin{dfn}[\cite{DBLP:journals/tse/ChiolaMBC93}]
  The transition s $t_1$ and $t_2$ are in \emph{assymetric confusion}
  in the marking $M$ if and only if it is \emph{not} the case that
  $t_1 \EffC{M} t_2$, $t_1$ is enabled in $M$,
  $M \xrightarrow{t_1} M'$ and there exists a transition $t_3 \in T$
  such that $t_2 \EffC{M'} t_3$.
\end{dfn}

Algorithms that are used to determine $p_{\text{immediate}}$
\citep{DBLP:journals/tse/ChiolaMBC93, DBLP:journals/tse/TeruelFP03}
usually consider confusion a semantic error and help the modeller
choose priorities for the immediate transitions in order to avoid
confusion. If the net is not confused, immediate transitions not in
pairwise conflict can be fired in \emph{batches}, while weights
$\lambda$ are used to randomly resolve conflicts.

The trivial case of a net without confusion is one without any
immedaite transitions, i.e.~$\PI \equiv 0$ for all transitions. Such
$\gspn$s are called (ordinary) stochastic Petri nets or \spn s.

\section{Tensors and Kronecker algebra}

\chapter{Example Petri Net}

We will now consider the Generalized Stochastic Petri Net $\PN$ in
Figure~\ref{fig:example:pn}~(a). The net contains seven places, two
immediate transitions $t_1$ and $t_3$, which form an effective
conflict set, and six timed transitions.

\begin{figure}
  \hspace*{\fill}
  \parbox[c]{0.55\linewidth}{\centering \begin{tikzpicture}
  \petriP(p1)(0,0){$p_1$}<1>
  \petriT(t1)(2,-0.8){$t_1$}<1>{2}
  \petriT(t2)(-2,-0.8){$t_2$}<1>{1}
  \petriP(p2)(3,-1.5){$p_2$}
  \petriP(p4)(1,-2.4){$p_4$}
  \petriT(t3)(0,-2.4){$t_3$}{1}
  \petriP(p5)(-1,-1.5){$p_5$}<1>
  \petriP(p3)(-3,-1.5){$p_3$}
  \petriT(t4)(2,-2.4){$t_4$}{0.1}
  \petriT(t5)(-2,-2.4){$t_5$}{0.1}
  \petriP(p6)(1,-4.3){$p_6$}
  \petriT(t6)(0,-4.3){$t_6$}{1}
  \petriP(p7)(-1,-4.3){$p_7$}
  \petriT(t7)(-2,-4.3){$t_7$}{2}

  \begin{pgfonlayer}{connections}
    \draw [->]
    (p1) edge (t1)
    (p5) edge (t1)
    (t1) edge (p2)
    (p1) edge (t2)
    (p5) edge (t2)
    (t2) edge (p3)
    (p4) edge (t3)
    (t3) edge (p5)
    (p2) edge (t4)
    (t4) edge (p4)
    (t4) edge (p6)
    (p3) edge (t5)
    (t5) edge (p5)
    (t5) edge (p7)
    (p6) edge (t6)
    (t6) edge (p7)
    (p7) edge (t7)
    (t7) to ++(-1.5,0) |- (p1);
  \end{pgfonlayer}

  \begin{pgfonlayer}{bg}
    \path [rounded corners=5pt,fill=black!10,draw,dashed]
    (-3.8,0.9) rectangle (3.5,-3.2);
    \path [rounded corners=5pt,fill=black!10,draw,dashed]
    (-2.5,-3.4) rectangle (2.5,-5.1);
    \node [anchor=north east] at (3.5,0.9) {$\LN_1$};
    \node [anchor=north east] at (2.5,-3.4) {$\LN_2$};
  \end{pgfonlayer}
\end{tikzpicture}}
  \hfill
  \parbox[c]{0.4\linewidth}{\centering \begin{tikzpicture}
  \petriP(pi1)(0,0){$\PI_1$}<1>
  \petriT(t4)(1.5,-1){$t_4$}
  \petriT(t5)(-1.5,-1){$t_5$}
  \petriP(pi2)(0,-2){$\PI_2$}
  \petriT(t7)(-1.5,-3){$t_7$}

  \begin{pgfonlayer}{connections}
    \draw [->]
    (pi1) edge (t4)
    (t4) edge (pi2)
    (pi1) edge (t5)
    (t5) edge (pi2)
    (pi2) edge (t7)
    (t7) to ++(-0.5,0) |- (pi1);
  \end{pgfonlayer}
\end{tikzpicture}}
  \hspace*{\fill}
  \par\vspace{0.5\onelineskip}
  \hspace*{\fill}
  \parbox[t]{0.55\linewidth}{\subcaption{%
      Partition consisting of regions $\LN_1$ and $\LN_2$}}
  \hfill
  \parbox[t]{0.4\linewidth}{\subcaption{High level structure}}
  \hspace*{\fill}
  \caption{Example Generalized Stochastic Petri Net $\PN$ with its
    initial marking $M_0$}
  \label{fig:example:pn}
\end{figure}

\section{Decomposition into High Level and Low Level Nets}

\subsection{Partitioning Into Minimal Regions}

To facilitate Kronecker analysis we partition the net into
\emph{minimal regions} following
\citet{DBLP:journals/sigmetrics/BuchholzK98}.

A set of transitions $T_r \subset T$ defines a region if
$({}^{\bullet, \circ}T_r)^{\bullet, \circ} = T_r$, where
${}^{\bullet,\circ}T_r = {}^{\bullet}T_r \cup {}^{\circ}T_r$ is set of
places connected to transitions in $T_r$ by input or inhibitor edges,
and $P_r^{\bullet, \circ} = P_r^{\bullet} \cup P_r^{\circ}$ is the set
of transitions connected to places in $P_r$ by input or inhibitor
edges. In other words, the minimum regions partition the set of
transitions $T$ such that no place is shared by two transitions for
different regions as input on inhibitor.

We first partition $T$ into singleton sets $(\{p_i\})_{i = 1}^7$, then
iterate over the places. In the $i$th iteration, all the sets which
share $p_i$ as input (or inhibitor) are merged. We only need to merge
regions in iterations when $p_i$ has at least two adjacent input (or
inhibitor) edges.

\subsection{Merging Minimal Regions into Subnets}

The minimal region $\{t_1, t_2, t_3, t_4\}, \{t_5\}$ has immediate
output transitions $t_1$ and $t_2$ which must be turned into internal
transitions by region merging. Moreover, minimal regions $\{t_7\}$ and
$\{t_8\}$ contain no internal (timed) behaviour, therefore, they also
need to be merged. We merge them with each other in order to divide
$\PN$ into $J = 2$ low-level nets and $k = 3$ output transitions
between them.

Partitioning of the net is illustrated in
Table~\ref{tbl:example:partition}.

\begin{table}
  {\centering
    \begin{tabular}{@{}lll@{}}
      \toprule
      & Action & Regions \\
      \midrule
      & Initialize & $\{t_1\}, \{t_2\}, \{t_3\}, \{t_4\}, \{t_5\},
                     \{t_6\}, \{t_7\}, \{t_8\}$ \\[0.5em]
      $p_1$ & Merge $t_1, t_2, t_3, t_4$ & $\{t_1, t_2, t_3, t_4\}, \{t_5\},
                     \{t_6\}, \{t_7\}, \{t_8\}$ \\
      $p_2$ & & $\{t_1, t_2, t_3, t_4\}, \{t_5\},
                     \{t_6\}, \{t_7\}, \{t_8\}$ \\
      $p_3$ & & $\{t_1, t_2, t_3, t_4\}, \{t_5\},
                     \{t_6\}, \{t_7\}, \{t_8\}$ \\
      $p_4$ & $t_1, t_3$ ok & $\{t_1, t_2, t_3, t_4\}, \{t_5\},
                     \{t_6\}, \{t_7\}, \{t_8\}$ \\
      $p_5$ & $t_2, t_4$ ok & $\{t_1, t_2, t_3, t_4\}, \{t_5\},
                     \{t_6\}, \{t_7\}, \{t_8\}$ \\
      $p_6$ & & $\{t_1, t_2, t_3, t_4\}, \{t_5\},
                     \{t_6\}, \{t_7\}, \{t_8\}$ \\
      $p_7$ & & $\{t_1, t_2, t_3, t_4\}, \{t_5\},
                     \{t_6\}, \{t_7\}, \{t_8\}$ \\[0.5em]
      $t_1$ & Merge $t_1, t_5$ & $\{t_1, t_2, t_3, t_4, t_5\},
                     \{t_6\}, \{t_7\}, \{t_8\}$ \\
      $t_3$ & Merge $t_3, t_6$ & $\{t_1, t_2, t_3, t_4, t_5, t_6\},
                     \{t_7\}, \{t_8\}$ \\[0.5em]
      $t_7$ & Merge $t_7, t_8$ & $\LN_1 = \{t_1, t_2, t_3, t_4, t_5, t_6\},
                     \LN_2 = \{t_7, t_8\}$ \\
      \bottomrule
    \end{tabular}
    \par}
  \caption{Partitioning the set of transitions into regions}
  \label{tbl:example:partition}
\end{table}

\subsection{Partial Place Invariants and MSIPs}

We first compute the $P$-invariants of $\PN$:
\begin{itemize*}
\item $\{p_2, p_3, p_4, p_5\}$,
\item $\{p_1, p_2, p_3, p_6, p_7\}$.
\end{itemize*}
The $P$-invariants of $\LN_1$ with the output transitions removed:
\begin{itemize*}
\item $\PI_1 = \{p_1, p_2, p_3\}$,
\item $\{p_2, p_3, p_4, p_5\}$.
\end{itemize*}
The $P$-invariant of $\LN_2$ with the output transition removed:
\begin{itemize*}
\item $\PI_2 = \{p_6, p_7\}$.
\end{itemize*}

\emph{Partial} $P$-invariants are invariants of the low-level nets
which are not invariants of the full net. Therefore, both $\LN_1$ and
$\LN_2$ have only one partial invariants, $\PI_1$ and $\PI_2$,
respectively. These invariants correspond to MSIPs in the high level
net $\HN$ an the extended nets $\EN_1$ and $\EN_2$. The high-level net
is shown in Figure~\ref{fig:example:pn}~(b).

\section{High-level and Low-level TRS exploration}

Because both the high-level and low-level nets are $1$-bounded, we
will denote a marking by the set of places which have a token,
e.g.~$M_0 = \{p_1, p_5\}$ in the original net $\PN$ and $M_0 =
\{\PI_1\}$ in $\HN$.

The set of reachable high-level states (macro-markings) are
\begin{equation*}
  \TRS(\HN) = \{s^H_1 = \{\PI_1\}, s^H_2 = \{\PI_2\}\} \text.
\end{equation*}
When exploring the tangible state-spaces of the low-level nets, we
also note which high-level marking a given state corresponds to:
\begin{align*}
  \TRS(\LN_1) &= S^{(1)}_1 \cup S^{(1)}_2 \text, \\
  S^{(1)}_1 &= \{s^{(1)}_{1} = \{p_1, p_5\}, s^{(1)}_{2} = \{p_2\},
              s^{(1)}_{3} = \{p_3\}\} \text, \\
  S^{(2)}_2 &= \{s^{(1)}_{4} = \{p_4\}, s^{(1)}_{5} = \{p_5\}\} \text,
  \\
  \TRS(\LN_2) &= S^{(2)}_1 \cup S^{(2)}_2 \text, \\
  S^{(2)}_1 &= \{s^{(2)}_{1} = \emptyset\} \text, \\
  S^{(2)}_2 &= \{s^{(2)}_{2} = \{p_6\}, s^{(2)}_{3} = \{p_7\}\} \text.
\end{align*}

A na\"\i ve approximation of the state space of $\PN$ is $\TRS(\LN_1)
\times \TRS(\LN_2)$. However, this set of states contains many
markings unreachable from $M_0$.

We will analyse the behaviour of $\PN$ as a \ctmc\ over the state space
$S' = S'_1 \cup S'_2 = S^{(1)}_1 \times S^{(2)}_1 \cup S^{(1)}_2
\times S^{(2)}_2\}$.
Observe that the the global state $(s^{(1)}_{5}, s^{(2)}_{2})$ is
unreachable, because in order have a token in $p_5$, transition $t_6$
must be fired, however, in order to have a token in $p_6$, transition
$t_5$ must be fired. The true state space of $\PN$ is
$S = S' \setminus \{(s^{(1)}_{5}, s^{(2)}_{2})\}$.

While the introduction of low-level state-space partitioning by MSIPs
reduced the size of $S'$ considerably, the resulting Markov chain is
still not irreducible; we have to work around unreachable marking. In
addition, the reduction of state-space size---which reduces storage
requirements of probability vectors---increased the complexity of the
rate matrix, as we will see in the next Section.

\section{Rate Matrix Generation}

We will find the rate matrix as a block matrix of $\lvert S^H \rvert
\times \lvert S^H \rvert$ block, plus a diagonal correction term:
\begin{gather}
  Q_O = \left( \begin{array}{@{}c|c@{}}
                 Q_O[1, 1] & Q_O[1, 2] \\
                 \hline Q_O[2, 1] & Q_O[2, 2]
       \end{array} \right) \text, \\
  Q_D = \diag(\vec{e}^T Q_O) \text, \\
  Q = Q_O + Q_D \text,
  \intertext{where the $\lvert S'_{x} \rvert \times \lvert S'_{y}
    \rvert$ matrix $Q_O[x, y]$ describes the rate of transitions from
    macro-marking $s^H_{x}$ to $s^H_{y}$ and is of the form:}
  Q_O[x, y] = \begin{cases}
    \displaystyle \bigoplus_{j = 1}^J Q^{(j)}_L[x, x] +
    \sum_{i = 1}^{k_{x, x}} \bigotimes_{j = 1}^{J} Q^{j}_k[x, x] & \text{if
      $x = y$,} \\
    \displaystyle
    \sum_{i = 1}^{k_{x, y}} \bigotimes_{j = 1}^{J} Q^{j}_k[x, y] & \text{if
      $x \ne y$.}
  \end{cases} \label{eq:example:QO}
\end{gather}

\emph{Is it true in the general case that $k_{x,x} = 0$? An output
  transition adds at least one token to a place in a \emph{foreign}
  region, therefore increases the value of at least one partial
  P-invariant of that region. Therefore, no output transition may
  leave the MSIPs unchanged.}

The $\lvert S^{(j)}_{x} \rvert \times \lvert S^{(j)}_{x} \rvert$
matrix $Q^{(j)}_L[x, x]$ describes the effects of inner transitions in
$\LN_j$ when the current macro-marking is $s^H_x$. The
$\lvert S^{(j)}_{x} \rvert \times \lvert S^{(j)}_{y} \rvert$ matrix
$Q^{(j)}_i[x, x]$ describes the effects in $\LN_j$ of the $i$th
transition (out of $k_{x, x}$) which takes $\HN$ from $s^H_x$ to
$s^H_y$.

The local transition matrices are
\begin{align}
  Q^{(1)}_L[1, 1] &= \begin{pmatrix}
    0 & 2 & 1 \\
    0 & 0 & 0 \\
    0 & 0 & 0
  \end{pmatrix} \text,
  & Q^{(2)}_L[1, 1] &= \begin{pmatrix}
    0
  \end{pmatrix} \text, \\
  Q^{(1)}_L[2, 2] &= \begin{pmatrix}
    0 & 0 \\
    0 & 0
  \end{pmatrix} \text, &
  Q^{(2)}_L[2, 2] &= \begin{pmatrix}
    0 & 1 \\
    0 & 0
  \end{pmatrix} \text.
\end{align}

There are no output transitions that leave the macro-marking
unchanged, therefore $k_{1, 1} = k_{2, 2} = 0$. The transitions $t_5$
and $t_6$ take $s^H_1$ to $s^H_2$, while transition $t_8$ takes
$s^H_2$ to $s^H_1$, therefore $k_{1, 2} = 2$ and $k_{2, 1} = 1$.

First we construct the matrices corresponding to output transitions
$t_5$. Note that we write $\lambda(t_5)$ ad $\lambda(t_6)$ in the
appropriate cells of $Q^{(1)}_1[1, 2]$ and $Q^{(1)}_2[1, 2]$ instead
of $1$ in order to avoid multiplication by $\lambda$ in
\eqref{eq:example:QO}.
\begin{align}
  Q^{(1)}_1[1, 2] &= \begin{pmatrix}
    0 & 0 \\
    0.1 & 0 \\
    0 & 0
  \end{pmatrix} \text, &
  Q^{(2)}_1[1, 2] &= \begin{pmatrix}
    1 & 0
  \end{pmatrix} \text, \\
  Q^{(1)}_2[1, 2] &= \begin{pmatrix}
    0 & 0 \\
    0 & 0 \\
    0 & 0.1
  \end{pmatrix} \text, &
  Q^{(2)}_2[1, 2] &= \begin{pmatrix}
    0 & 1
  \end{pmatrix} \text. \\
\end{align}

To construct the matrices corresponding to $t_8$, we must model the
immediate random behaviour of $\LN_1$ due to $t_1$ and $t_3$ in
$Q^{(1)}_1[2, 1]$:
\begin{align}
  Q^{(1)}_1[2, 1] &= \begin{pmatrix}
    0 & \frac{1}{3} & \frac{2}{3} \\
    1 & 0 & 0
  \end{pmatrix} \text, &
  Q^{(2)}_1[2, 1] &= \begin{pmatrix}
    0 \\
    0.5
  \end{pmatrix} \text.
\end{align}

The full $Q$ matrix is
\begin{equation}
  Q = \left( \begin{array}{@{}ccc|cccc@{}}
               * & 2 & 1 & 0 & 0 & 0 & 0 \\
               0 & * & 0 & 0.1 & 0 & 0 & 0 \\
               0 & 0 & * & 0 & 0 & 0 & 0.1 \\
               \hline
               0 & 0 & 0 & * & 1 & 0 & 0 \\
               0 & \frac{1}{6} & \frac{1}{3} & 0 & * & 0 & 0 \\
               0 & 0 & 0 & 0 & 0 & * & 1 \\
               0.5 & 0 & 0 & 0 & 0 & 0 & * \\
             \end{array} \right) \text.
\end{equation}
Observe that the second-to-last column contains no positive entries,
therefore, the corresponding $(s^{(1)}_5, s^{(2)}_2)$ state is indeed
unreachable and the CMTC is not irreducible.

\section{Kronecker Analysis with Decomposition}

\chapter{Kronecker Analysis}

\begin{figure}
  {\centering
    \begin{tikzpicture}[
  process/.style={
    text width=2.5cm,
    align=center,
    draw
  }
  ]
  \matrix [column sep=20pt,row sep=15pt] {
    \node (start) [fill=black,circle,inner sep=5pt] {}; \\[-7pt]
    \node (decompose) [process] {Decomposition}; &
    \node (matrix) [process] {Matrix\\Inventory}; &
    \node (init) [process] {Initialize $\vec{\pi}$, $\vec{y}$}; \\
    \node (aggregate) [process] {Aggregate\\Subproblems}; &
    \node (factorize) [process] {Construct and\\Factorize $Q^{(j)}$}; &
    \node (solve) [process] {Solve for $\vec{\pi}^{(j)}$}; \\
    \node (disaggregate) [process] {Disaggregate\\into
      $\vec{\pi}$}; &
    \node (correct) [process] {Solve for $\vec{y}$}; &
    \node (converged) [draw,diamond,aspect=2,inner sep=2pt]
    {Converged?}; \\
    & & \node (end) [draw,double,double distance=2pt,fill=black,circle,inner sep=5pt] {}; \\
  };
  \node (beforeaggregate) [diamond,inner sep=3pt,draw] at
  ($(init.south)+(0,-13pt)$) {};
  \draw [->] (start) edge (decompose);
  \draw [->] (decompose) edge (matrix);
  \draw [->] (matrix) edge (init);
  \draw [->] (init) edge (beforeaggregate);
  \draw [->] (beforeaggregate) -| (aggregate);
  \draw [->] (aggregate) edge (factorize);
  \draw [->] (factorize) edge (solve);
  \draw [->] (solve.south) to ++(0,-13pt) -| (disaggregate);
  \draw [->] (disaggregate) edge (correct);
  \draw [->] (correct) edge (converged);
  \draw [->] (converged.east) to ++(0.75cm,0) |- node [near
  start,right] {No} (beforeaggregate);
  \draw [->] (converged) edge node [right] {Yes} (end);
\end{tikzpicture}
    \par}
  \caption{The exact decompositional algorithm}
  \label{fig:kronecker:flowchart}
\end{figure}

Rough outline for figuring out the exact decomposition algorithm
\citep{bao2008decompositional} in Figure~\ref{fig:kronecker:flowchart}:

\section{Construction of Initial Probability Vector}

Due to the reducibility of $Q$, the initial choice of $\vec{\uppi}$ will
determine the steady-state solution. Therefore, we need to choose
$\vec{\uppi}$ such that it is only nonzero at the states reachable from
$M_0$,
\begin{equation}
  \pi_i = \begin{cases}
    1 / \lvert \TRS(\PN) \rvert & \text{if $s_i \in \TRS(\PN)$,} \\
    0 & \text{if $s_i \notin \TRS(\PN)$.} \\
  \end{cases}
\end{equation}
This requires discovery of $\TRS(\PN)$.

\section{Solution of the Reducible Subproblem}

Due to the block structure of $Q^{(j)}$, the ``projection'' of $Q$ to
$\LN_j$ which is obtained by removing the Kronecker products from
\eqref{eq:example:QO}, it is never irreducible. State changes which
would allow $Q^{(j)}$ to move between different macro-markings would violate
the partial $P$-invariants of $\LN_j$ which are used to describe the
macro-markings.

Therefore, we have to disaggregate and aggregate the synchronized part
of $Q$ in the decompositional algorithm. This reduces the
decompositional algorithm of \citet{bao2008decompositional} to the
approximate decompositional algorithm of
\citet{DBLP:journals/sigmetrics/BuchholzK98}, plus a Gauss--Seidel
correction step.

\section{Disaggregation of the Probability Vector}

After solving the local equations with (dis)aggregated synchronized
transitions, we obtain local probability vectors $\pi^{(j)}$ for each
local net $\LN_j$. Due to the reducibility of $Q$, the
vector Kronecker product
\begin{equation}
  \vec{\uppi}' = \bigotimes_{j = 1}^J \vec{\uppi}^{(j)}
\end{equation}
cannot be used as input for the GS correction step, because it can
have support at states unreachable from $M_0$.

\begin{prop}
  Let $S = \{1, 2, \ldots, n\}$ and $S^{(j)} = \{1, 2, \ldots, n_j\}$
  for all $j = 1, 2, \ldots, J$ be state spaces, $\TRS \subseteq S$,
  $f^{(j)}: S \to S^{(j)}$ be surjective mappings and
  $\vec{\uprho}^{(j)}$ be positive probability vectors (that is,
  $\sum_{s^{(j)} \in S^{(j)}} \rho^{(j)}(s^{(j)}) = 1$ and
  $\rho^{(j)}(s^{(j)}) > 0$ for all
  $s^{(j)} \in S^{(j)}$). The system of equations
  \begin{empheq}[left=\empheqlbrace]{gather}
    \sum_{s \in S} \pi(s) = 1 \text, \qquad
    \pi(s) \ge 0 \text{ for all $s \in S$,}
    \label{eq:kronecker:disaggregate-cond1} \\
    \sum_{f^{(j)}(s) = s^{(j)}} \pi(s) = \rho^{(j)}(s^{(j)}) \text{
      for all $j = 1, 2, \ldots, j$ and $s^{(j)} \in S^{(j)}$,}
    \label{eq:kronecker:disaggregate-cond2} \\
    \pi(s) = 0 \text{ for all $s \notin \TRS$}
    \label{eq:kronecker:disaggregate-cond3}
  \end{empheq}
  has a solution of the form
  \begin{align}
    \pi(s) = a \prod_{j = 1}^{J} p^{(j)}(f^{(j)}(s))
    \text, \label{eq:kronecker:disaggregated-pi}
  \end{align}
  where $\vec{p}^{(j)}$ are nonnegative vectors and a is a positive
  constant. Solutions of this form have maximum entropy
  \begin{align}
    H(\vec{\uppi}) = \sum_{s \in \TRS} - \pi(s) \log \pi(s) \text.
  \end{align}

  Conversely, any probability vector $\vec{\uppi}$ of the form
  \eqref{eq:kronecker:disaggregated-pi} that satisfies
  \eqref{eq:kronecker:disaggregate-cond2} and
  \eqref{eq:kronecker:disaggregate-cond3} also satisfies
  \eqref{eq:kronecker:disaggregate-cond1} and has maximum entropy the
  entropy $H(\vec{\uppi})$.
\end{prop}

\begin{proof}
  Consider the Lagrange function associated with the optimization
  problem $\max_{\vec{\uppi}} H(\vec{\uppi})$ subject to
  \eqref{eq:kronecker:disaggregate-cond1}--%
  \eqref{eq:kronecker:disaggregate-cond3} by introducing Lagrange
  multiplicators $\alpha, \vec{\upbeta}, \vec{\upgamma}^{(j)}$ and
  $\vec{\updelta}$,
  \begin{equation}
    \begin{aligned}
      & \Lambda(\vec{\pi}, \alpha, \vec{\upbeta},
      \vec{\upgamma}^{(1)}, \vec{\upgamma}^{(2)}, \ldots,
      \vec{\upgamma}^{(J)},
      \vec{\updelta}) =\\
      &\qquad - \sum_{s \in \TRS} - \pi(s) \log \pi(s) + \alpha
      \Biggl( \sum_{s \in S} \pi(s) - 1 \Biggr) + \sum_{s \in S}
      \beta(s) \pi(s) \\
      &\qquad + \sum_{j = 1}^{J} \, \sum_{s' \in S^{(j)}}
      \gamma^{(j)}(s') \Biggl( \sum_{s \in F^{(j)}(s')} \pi(s) -
      \rho^{(j)}(s') \Biggr) + \sum_{s \notin \TRS} \delta(s) \pi(s)
      \text,
    \end{aligned}
  \end{equation}
  where we have written $F^{(j)}(s') = \{ s \in S \mid f^{(j)}(s) = s'
  \}$ for brevity.

  Suppose that the optimal solution has $\pi(s) > 0$ for all
  $s \in \TRS$. Hence, the gradient $\nabla \Lambda$ exists and the
  optimal solution satisfies the Karush--Kuhn--Tucker conditions
  \begin{gather}
    \frac{\partial \Lambda}{\partial \pi(s)}
    = \log \pi(s) + 1 + \alpha + \beta(s) + \sum_{j = 1}^{J}
    \gamma^{(j)}(f^{(j)}(s)) = 0 \text{ for all $s \in \TRS$,} \\
    \beta(s) \ge 0 \text{ and } \beta(s) \pi(s) = 0 \text{ for all $s
      \in S$.}
  \end{gather}
  We can conclude from $\pi(\TRS) > 0$ that $\beta(s) = 0$ for all
  $s \in \TRS$ and
  \begin{equation}
    \pi(s) = \exp (-1 - \alpha) \prod_{j = 1}^J \exp
    [ -\gamma^{(j)}(f^{(j)}(s)) ]
    = a \prod_{j = 1}^J p^{(j)}(f^{(j)}(s)) \text.
  \end{equation}
\end{proof}

\section{Correction Vector Calculation}

(Block) Gauss--Seidel, as presented in the book, only works on sums of
Kronecker products, not block matrices where each block is a sum of
Kronecker products. Therefore, another method is needed to calculate
the correction vector $\vec{y}$.

Could (Block) Jacobi or (Block) power iteration \citep[Section
3.2]{dayar2012analyzing} be used here? Another possibility is a
``hybrid'' Jacobi--Gauss--Seidel approach, where we use Gauss--Seidel
backward substitution is used for the diagonal block, but the
off-diagonal blocks are zero in the matrix we need to ``invert'' in
matrix splitting.

\citet[Equation~(16)]{DBLP:journals/questa/Buchholz94} gives a
modified form of block Gauss--Seidel iteration for block structured
matrices. Does it respect the unreachable global states?

\section{Exact Hierarchical Decomposition}

\subsection{Basic Algorithm}

Let $\LN_1, \LN_2, \ldots, \LN_J$ be a decomposition of $\PN$ into
subnets, i.e.~$P = P_1 \coprod P_2 \coprod \cdots \coprod P_j$ a
disjoint partition of the places $P$ in $\PN$ into subsets $P_j$,
which are the places of $\LN_j$, respectively.

Let $\TRS_j$ denote the tangible reachability set of $\LN_j$. The set
$\TRS$ of tangible states of $\PN$ is a subset of the \emph{potential}
reachability set $\PS$,
\begin{equation}
  \TRS \subseteq \PS = \TRS_1 \times \TRS_2 \times \cdots \times
  \TRS_J\text.
\end{equation}

Let us define the equivalence relation $\sim_j$ over $\TRS_j$ as
\begin{equation}\label{eq:kronecker:state-equivalence}
  s^{(j)} \sim_j s'^{(j)} \Longleftrightarrow \{ s \in \TRS : s[j] =
  s^{(j)} \} = \{ s' \in \TRS : s'[j] = s'^{(j)} \} \text,
\end{equation}
where $s[j] \in \TRS_j$ denotes the local state of $\LN_j$ in the
global state $s \in \TRS$, i.e.~the $j$th component of $s$. The set of
macro-states $\macroTRS_j$ is the set of equivalence classes $\TRS_j /
{\sim_j}$.

The set of reachable macro-state combinations
$\macroTRS \subset \macroStates{\PS} = \macroTRS_1 \times \macroTRS_2
\times \cdots \times \macroTRS_j$
is used as a set os block indices to the block-Kronecker infinitesimal
generatior matrix:
\begin{gather}
  Q = Q_O - Q_D \text, \\
  Q_O[\macroS{s}, \macroS{s}'] = \begin{cases} \displaystyle
    \bigoplus_{j = 1}^J Q_L^{(j)}[\macroS{s}, \macroS{s}] + \sum_{i =
      1}^{k_{\macroS{s}, \macroS{s}}} \bigotimes_{j = 1}^J
    Q_{i}^{j}[\macroS{s},
    \macroS{s}] & \text{if $\macroS{s} = \macroS{s}'$,} \\
    \displaystyle \sum_{i = 1}^{k_{\macroS{s}, \macroS{s}'}}
    \bigotimes_{j = 1}^J Q_{i}^{j}[\macroS{s}, \macroS{s}'] & \text{if
      $\macroS{s} = \macroS{s}'$,}
  \end{cases} \\
  Q_D = \diag (Q_O \vec{e}^T) \text,
\end{gather}
where $Q_L^{(j)}[\macroS{s}, \macroS{s}]$ corresponds to the local
transitions in $\LN_j$ and the macro-state $\macroS{s}$, while
$Q_{i}^{j}[\macroS{s}, \macroS{s}']$ corresponds to the effects of
$i$th synchronized transition that brings the system from global
macro-state $\macroS{s}$ to $\macroS{s}'$ in $\LN_j$.

In the original algorithm of \citet{DBLP:journals/tse/Buchholz99},
$\TRS$ is represented with a bit vector of length $\lvert \PS
\rvert$.
After the generation of macro-states $\macroTRS_1$, only a bit vector
of length
$\lvert \macroTRS_1 \times \TRS_2 \times \TRS_3 \times \cdots \times
\TRS_J \rvert$
is needed, because states in the macro-states of $\LN_1$ have
identical external behaviour. Likewise, the first $j - 1$ subnets may
be represented by their macro-states when we calculate $\macroTRS_j$.

\subsection{Reducing Storage Requirements by Preprocessing}

TODO Section~5 in \citet{DBLP:journals/tse/Buchholz99}.

\subsection{Efficient Macro-state Generation with MDDs}

Consider the quasi-reduced ordered MDD that describes $\TRS$ and
suppose that nodes on the $j$th level correspond to the subnet
$\LN_j$. Let
\begin{equation}
  \Env_j(s^{(j)}) = \bigcup_{\level(a) = j} \Above(a) \times a[s^{(j)}]
\end{equation}
be the \emph{environment} of the local state $s^{(j)} \in S^{(j)}$,
where $\Above(a)$ is the MDD of the upper $J - j$ levels of $r$
leading to $a$.

Local macro states consist of local states with the same
(quasi-reduced) environment, that is,
\begin{gather}
  s^{(j)}_1 \sim_j s^{(j)}_2 \Longleftrightarrow \Env_j(s^{(j)}_1) =
  \Env_j(s^{(j)}) \text, \\
  \macroS{S}^{(j)} = S^{(j)} / {\sim}_j \text.
\end{gather}

We can obtain the MDD representation of $\macroTRS$ by replacing the
local states with the corresponding local macro-states in $r$.

\subsection{Block iteration}

The block iteration technique \citep{DBLP:journals/tse/Buchholz99}
attempts to find the steady-state solution by iterating
\begin{gather}
  \vec{p}_k[\macroS{x}] Q[\macroS{x}, \macroS{x}] =
  \vec{b}_k[\macroS{x}] \text, \label{eq:kronecker:block:solve} \\
  \vec{b}_k = \sum_{\macroS{y} < \macroS{x}} \vec{p}_k[\macroS{y}]
  Q[\macroS{y}, \macroS{x}] + \sum_{\macroS{y} > \macroS{x}}
  \vec{p}_{k - 1}[\macroS{y}] Q[\macroS{y}, \macroS{x}] \text,
\end{gather}
which corresponds to \emph{grouped GS iteration} from 

The main complication comes from the fact that, even in the case of an
irreducible $Q[\macroS{x}, \macroS{x}]$,
\eqref{eq:kronecker:block:solve} may have infinitely many solutions.

\begin{dfn}[\citet{young1971iterative}, Chapter~2, Definition~6.1.]
  The $n \times n$ matrix $A$ has \emph{Property A} if there exists a
  permutation matrix $P$ such that
  \begin{align}
    P^{-1} A P = \begin{pmatrix}
      D_1 & H \\ K & D_2
    \end{pmatrix} \text,
  \end{align}
  where $D_1$ and $D_2$ are square diagonal matrices.
\end{dfn}

\begin{dfn}[\citet{young1971iterative}, Chapter~5, Definition~3.1.]
  Given a matrix $A = (a_{ij})$, the integers $i$ and $j$ are
  associated with respect to $A$ if $a_{ij} \ne 0$ or $a_{ji} \ne 0$.
\end{dfn}

\begin{dfn}[\citet{young1971iterative}, Chapter~5, Definition~3.2.]
  The $n \times n$ matrix $A$ is \emph{consistently ordered} if there
  exists a partition
  $S_1 \uplus S_2 \uplus \cdots \uplus S_t = \{1, 2, \ldots, n\}$
  such that if $i$ and $j$ are associated with respect to $A$ and
  $i \in S_k$, then $j \in S_{k + 1}$ if $j > i$ and $j \in S_{k - 1}$
  if $j < i$.
\end{dfn}

\begin{thm}[\citet{young1971iterative}, Chapter~5]
  If $A$ is consistently ordered, then $A$ has Property A. Moreover,
  if $A$ has Property A, there exists a permutation matrix $P$ such
  that $P^{-1} A P$ is consistently ordered.
\end{thm}

\section{TODO}

\begin{itemize}
\item Add citations to \emph{everything} including the activity
  diagram
\item TODO articles and workflow
\item Navigable activity diagram
\item More examples\ldots
\end{itemize}

\begin{gather}
  Q[\macroS{x}, \macroS{y}] = \sum_{t \in T(\macroS{x}, \macroS{y})}
  w(t) \bigotimes_{j = 1}^J Q_{t}^{(j)}[\macroS{x}^{(j)},
  \macroS{y}^{(j)}]
  - \delta(\macroS{x}, \macroS{y}) \sum_{\macroS{z} \in \macroS{S}} \sum_{t \in T(\macroS{x}, \macroS{z})}
  w(t) \bigotimes_{j = 1}^J D_{t}^{(j)}[\macroS{x}^{(j)},
  \macroS{z}^{(j)}] \text, \\
  D_{t}^{(j)}[\macroS{x}^{(j)}, \macroS{z}^{(j)}] = \diag
  Q_{t}^{(j)}[\macroS{x}^{(j)}, \macroS{z}^{(j)}] \vec{e}^T \text.
\end{gather}

\chapter{Low-Rank Probability Vector Representations}

\section{Hiearchical Tucker Tensor Format}

Hiearchical Tucker represenation of tensors are a tool for storing
vectors
$\vec{v} \in \mathbb{R}^{n_1} \otimes \mathbb{R}^{n_2} \otimes \cdots
\otimes \mathbb{R}^{n_k}$
with moderate memory requirements. The recursive structure of the
dimension trees allows for simple operations on hierarchical Tucker
structured vectors, while the \emph{truncation} operation can reduce
memory requirements whith bounded error.

\begin{dfn}
  A \emph{dimension tree}
  $T(\mathbb{R}^{n_1}, \mathbb{R}^{n_2}, \ldots, \mathbb{R}^{n_k})$ is
  an (almost) balanced binary tree with leaf nodes
  $\{l_j\}_{j = 1}^{k}$. Each non-leaf node $p$ has exactly two
  children, $a.\textit{left}$ and $a.\textit{right}$. We will denote
  the root of the dimension tree by $T.\textit{root}$.
\end{dfn}

\begin{dfn}
  A \emph{hiearchical Tucker tree} $[\cdot]_{\vec{v}}$
  maps additional information to the nodes of a dimension tree
  $T(\mathbb{R}^{n_1}, \mathbb{R}^{n_2}, \ldots, \mathbb{R}^{n_k})$.

  Each node $a$ has two associated vectors spaces
  $[a]_{\vec{v}}.\textit{super}$ and $[a]_{\vec{v}}.\textit{sub}$,
  where $[a]_{\vec{v}}.\textit{sub} \le [a]_{\vec{v}}.\textit{super}$,
  and the \emph{rank} $[a]_{\vec{v}}.r = \dim
  [a]_{\vec{v}}.\textit{sub}$. The \emph{hiearachical Tucker rank} of
  $\vec{v}$ is tuple of all ranks $[a]_{\vec{v}}.r$, $a \in T$.

  For each leaf node $l_j$,
  $[l_j]_{\vec{a}}.\textit{super} = \mathbb{R}^{n_j}$, and
  $[l_j]_{\vec{a}}.\textit{sub}$ is the span of $[l_j]_{\vec{a}}.r$
  basis vectors
  $\langle [l_j]_{\vec{a}}.\vec{b}[1], [l_j]_{\vec{a}}.\vec{b}[2],
  \ldots, [l_j]_{\vec{a}}.\vec{b}[l_j.r] \rangle$.

  For each non-leaf node $p$, $[p]_{\vec{a}}.\textit{super} =
  [p.\textit{left}]_{\vec{v}}.\textit{sub} \otimes
  [p.\textit{right}]_{\vec{v}}.\textit{sub}$. The associated subspace
  is determined as follows:
  \begin{itemize}
  \item Let
    $[p]_{\vec{v}}.A[1], [p]_{\vec{v}}.A[2], \ldots,
    [p]_{\vec{v}}.A[[p]_{\vec{v}}.r]$
    be
    $[p.\textit{left}]_{\vec{v}}.r \times
    [p.\textit{left}]_{\vec{v}}.r$ matrices, referred together as the
    \emph{transfer tensor};
  \item Calculate the $i$th basis vector $[p]_{\vec{v}}.\vec{b}[i]$ as
    \begin{equation}
      [p]_{\vec{v}}.\vec{b}[i] = \sum_{j_1 =
        1}^{[p.\textit{left}]_{\vec{v}}.r} \sum_{j_2 =
        1}^{[p.\textit{right}]_{\vec{v}}.r} [p]_{\vec{v}}.A[i][j_1,
      j_2] \cdot
      [p.\textit{left}]_{\vec{v}}.\vec{b}[j_1] \otimes
      [p.\textit{right}]_{\vec{v}}.\vec{b}[j_2] \text;
    \end{equation}
  \item $[p]_{\vec{a}}.\textit{sub}$ is the span of all
    $[p]_{\vec{v}}.r$ basis vectors.
  \end{itemize}
\end{dfn}

In practice, only the matrices $[p]_{\vec{v}}.A$ need to be stored in
the non-leaf nodes and the bases $[l_j]_{\vec{v}}.\vec{b}$ in the leaf
nodes.

\begin{dfn}
  A representation of a vector
  $\vec{v} \in \mathbb{R}^{n_1} \otimes \mathbb{R}^{n_2} \otimes
  \cdots \otimes \mathbb{R}^{n_k}$
  in \emph{hiearchical Tucker format} is a tuple
  $(T, [\cdot]_{\vec{v}}, \rho)$, where $[\cdot]_{\vec{v}}$ is a
  hiearchical tucker tree over the dimension tree
  $T(\mathbb{R}^{n_1}, \mathbb{R}^{n_2}, \ldots, \mathbb{R}^{n_k})$
  and $\rho \in \mathbb{R}$ is a scaling constant if
  $[T.r]_{\vec{v}}.r = 1$ and
  \begin{equation}
    \vec{v} = \rho \cdot [r]_{\vec{v}}.\vec{b}[1] \text.
  \end{equation}
\end{dfn}

\begin{dfn}
  A hierarchical Tucker tree is \emph{orthonormal} if all the
  bases $[a]_{\vec{v}}.\vec{b}$ are orthonormal. A hiearchinal Tucker
  vector is orthonormal if its tree is orthonormal.
\end{dfn}

While multiplication by scalar, multiplication by Kronekcer product of
matrices and addition can be implemented on any hiearchical Tucker
format vector, efficient truncation requires orthonormal
bases. Therefore, the orthonormalization operator is of crucial
importance.

\subsection{Multiplication by Scalar}

Multiplication by scalar is a trivial operation on hiearchical Tucker
vectors.

\begin{algorithm}
  \caption{Multiplication by Scalar}
  \begin{algorithmic}[1]
    \Function{Multiply}{$(T, [\cdot]_{\vec{v}}, \rho)$, $\lambda$}
    \State \Return $(T, [\cdot]_{\vec{v}}, \lambda \cdot \rho)$
    \EndFunction
  \end{algorithmic}
\end{algorithm}

\subsection{Multiplication by Kronecker Product}

Multiplication by a Kronecker product of matrices is done by
performing matrix multiplication in $\mathbb{R}^{n_j}$ for each leaf
node $l_j$ of $T$, while the transfer tensors are unchanged.

We call the function \textsc{Clone} to create an exact copy of a
hiearchical Tucker tree for modification. The returned hiearachical
Tucker vector is generally not orthonormal.

\begin{algorithm}
  \caption{Multiplication by Kronecker Product}
  \begin{algorithmic}[1]
    \Function{Multiply}{$(T, [\cdot]_{\vec{v}}, \rho)$, $A_1 \otimes A_2
      \otimes \cdots \otimes A_j$}
    \State $[\cdot]_{\vec{w}} \gets \Call{Clone}{[\cdot]_{\vec{v}}}$
    \ForAll{leaf node $l_j$}
    \For{$i \gets 1$ \To $[l_j]_{\vec{v}}.r$}
    \State $[l_j]_{\vec{w}}.\vec{b}[i] \gets
    [l_j]_{\vec{v}}.\vec{b}[i] \cdot A_j$
    \EndFor
    \EndFor
    \State \Return $(T, [\cdot]_{\vec{w}}, \rho)$
    \EndFunction
  \end{algorithmic}
\end{algorithm}

\subsection{Addition of Hierarchical Tucker Vectors}

Addition of hierarchical Tucker vectors may be perfomred
recursively. The symbol $\dirplus$ is used to denote the direct sum of
matrices.

The returned hiearachical Tucker vector is generally not
orthonormaland its hiearchical Tucker rank is the sum of the addends'
rank.

\begin{algorithm}
  \caption{Addition of Hierarchical Tucker Vectors}
  \begin{algorithmic}[1]
    \Function{Add}{$(T, [\cdot]_{\vec{v}}, \rho)$, $(T,
      [\cdot]_{\vec{w}}, \sigma)$}
    \State Initialize a new hierarchical Tucker tree $[\cdot]_{\vec{v}
      + \vec{w}}$ over $T$
    \State $[T.r]_{\vec{v} + \vec{w}}.r = [T.r]_{\vec{v}}.r +
    [T.r]_{\vec{w}}.r$
    \State \Call{RecursiveAdd}{$[\cdot]_{\vec{v}}$,
      $[\cdot]_{\vec{w}}$, $[\cdot]_{\vec{v} + \vec{w}}$,
      $T.r.\textit{left}$}
    \State \Call{RecursiveAdd}{$[\cdot]_{\vec{v}}$,
      $[\cdot]_{\vec{w}}$, $[\cdot]_{\vec{v} + \vec{w}}$,
      $T.r.\textit{right}$}
    \For{$i \gets 1$ \To $[T.r]_{\vec{v}}.r$}
    \State $[T.r]_{\vec{v} + \vec{w}}.A[i] \gets \rho \cdot ([T.r]_{\vec{v}}.A[i]
    \dirplus 0)$
    \EndFor
    \For{$i \gets 1$ \To $[T.r]_{\vec{w}}.r$}
    \State $[T.r]_{\vec{v} + \vec{w}}.A[i + [T.r]_{\vec{v}}.r] \gets
    \sigma \cdot (0 \dirplus [T.r]_{\vec{w}}.A[i])$
    \EndFor
    \State \Return $(T, [\cdot]_{\vec{v} + \vec{w}}, 1)$
    \EndFunction
    \Statex
    \Procedure{RecursiveAdd}{$[\cdot]_{\vec{v}}$,
      $[\cdot]_{\vec{v}}$, $[\cdot]_{\vec{v} + \vec{w}}$, $a$}
    \State $[a]_{\vec{v} + \vec{w}}.r = [a]_{\vec{v}}.r +
    [a]_{\vec{w}}.r$
    \If{$a$ is a leaf node}
    \State $[a]_{\vec{v} + \vec{w}}.\vec{b} = [a]_{\vec{v}}.\vec{b}
    \uplus [a]_{\vec{w}}.\vec{b}$
    \Else
    \State \Call{RecursiveAdd}{$[\cdot]_{\vec{v}}$,
      $[\cdot]_{\vec{w}}$, $[\cdot]_{\vec{v} + \vec{w}}$,
      $a.\textit{left}$}
    \State \Call{RecursiveAdd}{$[\cdot]_{\vec{v}}$,
      $[\cdot]_{\vec{w}}$, $[\cdot]_{\vec{v} + \vec{w}}$,
      $a.\textit{right}$}
    \EndIf
    \For{$i \gets 1$ \To $[a]_{\vec{v}}.r$}
    \State $[a]_{\vec{v} + \vec{w}}.A[i] \gets [a]_{\vec{v}}.A[i]
    \dirplus 0$
    \EndFor
    \For{$i \gets 1$ \To $[a]_{\vec{w}}.r$}
    \State $[a]_{\vec{v} + \vec{w}}.A[i + [a]_{\vec{v}}.r] \gets
    0 \dirplus [a]_{\vec{w}}.A[i]$
    \EndFor
    \EndProcedure
  \end{algorithmic}
\end{algorithm}

\backmatter

\printbibliography

\end{document}