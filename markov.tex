
\documentclass[a4paper,11pt,twoside,openright]{memoir}

\renewcommand*{\baselinestretch}{1.15}
\setbinding{2mm}
\isopage
%% Page layout by Andreas Mathias from memman.pdf
% \setlength{\trimtop}{0pt}
% \setlength{\trimedge}{\stockwidth}
% \addtolength{\trimedge}{-\paperwidth}
% \settypeblocksize{634pt}{448.13pt}{*}
% \setulmargins{4cm}{*}{*}
% \setlrmargins{*}{*}{1.5}
% \setmarginnotes{17pt}{51pt}{\onelineskip}
% \setheadfoot{\onelineskip}{2\onelineskip}
% \setheaderspaces{*}{2\onelineskip}{*}
\checkandfixthelayout

%% Avoid widows and orphans
%\clubpenalty=10000
%\widowpenalty=10000
%\raggedbottom

\usepackage[usenames,dvipsnames]{xcolor}

\usepackage[T1]{fontenc}
\usepackage[utf8]{inputenc}
\usepackage{XCharter}
\usepackage[scaled]{helvet}
%\usepackage{berasans}
\usepackage{beramono}
\usepackage[expert,bitstream-charter,
greekuppercase=italicized,greeklowercase=italicized]{mathdesign}

%% https://groups.google.com/d/msg/comp.text.tex/XULGbUPl5hs/ffvt5EkzpyoJ
\DeclareSymbolFont{xyz}{U}{txmia}{m}{it}
\SetSymbolFont{xyz}{bold}{U}{txmia}{bx}{it}
\DeclareSymbolFontAlphabet{\varmathbb}{xyz}
\makeatletter
\DeclareMathSymbol{\m@thbbch@rA}{\mathord}{xyz}{129}
\DeclareMathSymbol{\m@thbbch@rB}{\mathord}{xyz}{130}
\DeclareMathSymbol{\m@thbbch@rC}{\mathord}{xyz}{131}
\DeclareMathSymbol{\m@thbbch@rD}{\mathord}{xyz}{132}
\DeclareMathSymbol{\m@thbbch@rE}{\mathord}{xyz}{133}
\DeclareMathSymbol{\m@thbbch@rF}{\mathord}{xyz}{134}
\DeclareMathSymbol{\m@thbbch@rG}{\mathord}{xyz}{135}
\DeclareMathSymbol{\m@thbbch@rH}{\mathord}{xyz}{136}
\DeclareMathSymbol{\m@thbbch@rI}{\mathord}{xyz}{137}
\DeclareMathSymbol{\m@thbbch@rJ}{\mathord}{xyz}{138}
\DeclareMathSymbol{\m@thbbch@rK}{\mathord}{xyz}{139}
\DeclareMathSymbol{\m@thbbch@rL}{\mathord}{xyz}{140}
\DeclareMathSymbol{\m@thbbch@rM}{\mathord}{xyz}{141}
\DeclareMathSymbol{\m@thbbch@rN}{\mathord}{xyz}{142}
\DeclareMathSymbol{\m@thbbch@rO}{\mathord}{xyz}{143}
\DeclareMathSymbol{\m@thbbch@rP}{\mathord}{xyz}{144}
\DeclareMathSymbol{\m@thbbch@rQ}{\mathord}{xyz}{145}
\DeclareMathSymbol{\m@thbbch@rR}{\mathord}{xyz}{146}
\DeclareMathSymbol{\m@thbbch@rS}{\mathord}{xyz}{147}
\DeclareMathSymbol{\m@thbbch@rT}{\mathord}{xyz}{148}
\DeclareMathSymbol{\m@thbbch@rU}{\mathord}{xyz}{149}
\DeclareMathSymbol{\m@thbbch@rV}{\mathord}{xyz}{150}
\DeclareMathSymbol{\m@thbbch@rW}{\mathord}{xyz}{151}
\DeclareMathSymbol{\m@thbbch@rX}{\mathord}{xyz}{152}
\DeclareMathSymbol{\m@thbbch@rY}{\mathord}{xyz}{153}
\DeclareMathSymbol{\m@thbbch@rZ}{\mathord}{xyz}{154}
\DeclareMathSymbol{\varBbbk}{\mathord}{xyz}{171}
\long\def\DoLongFutureLet #1#2#3#4{%
  \def\@FutureLetDecide{#1#2\@FutureLetToken
    \def\@FutureLetNext{#3}\else
    \def\@FutureLetNext{#4}\fi\@FutureLetNext}
  \futurelet\@FutureLetToken\@FutureLetDecide}
\def\DoFutureLet #1#2#3#4{\DoLongFutureLet{#1}{#2}{#3}{#4}}
\def\@EachCharacter{\DoFutureLet{\ifx}{\@EndEachCharacter}%
  {\@EachCharacterDone}{\@PickUpTheCharacter}}
\def\m@keCharacter#1{\csname\F@ntPrefix#1\endcsname}
\def\@PickUpTheCharacter#1{\m@keCharacter{#1}\@EachCharacter}
\def\@EachCharacterDone \@EndEachCharacter{}
\DeclareRobustCommand*{\varmathbb}[1]{\gdef\F@ntPrefix{m@thbbch@r}%
  \@EachCharacter #1\@EndEachCharacter}
\makeatother
\let\mathbb\varmathbb

\setsecheadstyle{\Large\sffamily\bfseries}
\setsubsecheadstyle{\large\sffamily\bfseries}
\setsubsubsecheadstyle{\normalsize\sffamily\bfseries}
\setparaheadstyle{\normalsize\sffamily\bfseries}
\setsubparaheadstyle{\normalsize\sffamily\bfseries}

\copypagestyle{my}{ruled}
\makeevenhead{my}{\thepage}{\itshape\textls{\MakeUppercase{\leftmark}}}{}
\makeoddhead{my}{}{\itshape\rightmark}{\thepage}
\makeevenfoot{my}{}{}{}
\makeoddfoot{my}{}{}{}
\makeatletter
\makepsmarks{my}{%
  \def\chaptermark##1{%
    \markboth{##1}{}}%
  \def\sectionmark##1{%
    \markright{%
      \ifnum\c@secnumdepth > \z@
      \thesection. \ %
      \fi
      ##1}}}
\makeatother
\pagestyle{my}

\renewcommand*{\chapnamefont}{\normalfont\huge\textit}
\renewcommand*{\chapnumfont}{\normalfont\huge\textit}
\renewcommand*{\chaptitlefont}{\normalfont\Huge\sffamily\bfseries}
\renewcommand*{\cftchapterfont}{\normalfont\sffamily\bfseries\large}
\renewcommand*{\cftchapterpagefont}{\normalfont\sffamily\bfseries\large}

\setsecnumdepth{subsection}
\settocdepth{subsection}

\renewcommand*{\pfbreakdisplay}{\ensuremath{\bullet\quad\bullet\quad\bullet}}
\newcommand*{\fbreak}{\fancybreak{\pfbreakdisplay}}

\usepackage{array,booktabs,mdwlist}
\renewcommand*{\arraystretch}{1.2}
\usepackage{xspace}

\usepackage{amsmath,relsize,stackengine,mleftright,mathtools}
\allowdisplaybreaks
\usepackage[ntheorem]{empheq}
\usepackage{upgreek}
\usepackage{nicefrac}

%% http://tex.stackexchange.com/a/100151/8744
\makeatletter
\def\tagform@#1{\maketag@@@{\small(#1)\@@italiccorr}}
\makeatother

\newcommand*{\mathabbrev}[1]{\textit{\textls[45]{#1}}}
\DeclareMathOperator{\E}{\mathbb{E}}
\newcommand*{\PN}{\mathabbrev{PN}}
\newcommand*{\SPN}{\mathabbrev{SPN}}
\newcommand*{\GSPN}{\mathabbrev{GSPN}}
\newcommand*{\SGSPN}{\mathabbrev{SGSPN}}
\newcommand*{\TL}{\mathabbrev{TL}}
\newcommand*{\TS}{\mathabbrev{TS}}
\newcommand*{\LN}{\mathabbrev{LN}}
\newcommand*{\HN}{\mathabbrev{HN}}
\newcommand*{\EN}{\mathabbrev{EN}}
\newcommand*{\PI}{\mathabbrev{PI}}
\newcommand*{\xto}{}\let\xto\xrightarrow
\newcommand*{\vecarrow}{}\let\vecarrow\vec
\renewcommand*{\vec}[1]{\boldsymbol{\mathrm{#1}}}
\newcommand*{\EffC}[1]{\mathop{\textnormal{EC}\mathnormal{(#1)}}}
\DeclareMathOperator{\EME}{EME}
\DeclareMathOperator{\SME}{SME}
\DeclareMathOperator{\PriME}{\Pi ME}
\DeclareMathOperator{\HME}{HME}
\DeclareMathOperator{\MME}{MME}
\newcommand*{\MInv}[1]{\tau\vec{#1}}
\DeclareMathOperator{\StrC}{SC}
\DeclareMathOperator{\CasC}{CC}
\DeclareMathOperator{\SCC}{SCC}
\newcommand*{\CCS}{\mathabbrev{CCS}}
\newcommand*{\ECS}{\mathabbrev{ECS}}
\newcommand*{\RS}{\mathabbrev{RS}}
\newcommand*{\TRS}{\mathabbrev{TRS}}
\newcommand*{\PS}{\mathabbrev{PS}}
\newcommand*{\PSS}{\mathabbrev{PSS}}
\newcommand*{\lS}[1]{S^{(#1)}}
\newcommand*{\macroS}{}\let\macroS\tilde
\newcommand*{\macroStates}{}\let\macroStates\widetilde
\newcommand*{\macroTRS}{\macroStates{\TRS}}
\DeclareMathOperator{\diag}{diag}
\DeclareMathOperator{\level}{level}
\DeclareMathOperator{\Env}{Env}
\DeclareMathOperator{\Above}{Above}
\DeclareMathOperator{\Bag}{Bag}
\newcommand*{\To}{\textbf{to}\xspace}
\DeclareMathOperator{\dirplus}{\mathbin{\overline{\oplus}}}
\newcommand*{\leftTok}[1]{{}^\bullet #1}
\newcommand*{\leftInh}[1]{{}^\circ #1}
\newcommand*{\rightTok}[1]{#1{}^\bullet}
\newcommand*{\rightInh}[1]{#1{}^\circ}
\newcommand*{\sumprime}{\sideset{}{'}\sum}

\newcommand*{\loc}[1]{^{(#1)}}

%% http://tex.stackexchange.com/a/13035/8744
\newcommand*{\textabbrev}[1]{\textls[35]{\textsc{#1}}}
\newcommand*{\Ctmc}{\textabbrev{Ctmc}}
\newcommand*{\ctmc}{\textabbrev{ctmc}}
\newcommand*{\Gspn}{\textabbrev{Gspn}}
\newcommand*{\gspn}{\textabbrev{gspn}}
\newcommand*{\Spn}{\textabbrev{Spn}}
\newcommand*{\spn}{\textabbrev{spn}}
\newcommand*{\Sgspn}{\textabbrev{Sgspn}}
\newcommand*{\sgspn}{\textabbrev{sgspn}}
\newcommand{\paren}[1]{\textup(#1\textup)}

\usepackage{xparse}

\usepackage[final]{microtype}

\usepackage{tikz}
\usetikzlibrary{arrows.meta,calc,positioning,shapes,matrix}
\usepackage{tikz-timing}
\usepackage[compactlambda]{petriTikz}

\pgfdeclarelayer{bg}
\pgfdeclarelayer{connections}
\pgfsetlayers{bg,connections,main}

\usepackage{varioref}
\renewcommand*{\reftextfacebefore}{}
\renewcommand*{\reftextfaceafter}{}

\usepackage[
  hyperfootnotes=false,
  breaklinks=true,
  hypertexnames=false,
  plainpages=false,
  pdfpagelabels=true,
  hidelinks
]{hyperref}

\let\openbox\relax
\usepackage[amsmath,thmmarks,hyperref]{ntheorem}
\theoremheaderfont{\normalfont\sffamily\bfseries}
\makeatletter
% Lifted from ntheorem.sty
\newcounter{proof}
\newcounter{currproofctr}
\newcounter{endproofctr}
\newcommand*{\openbox}{\leavevmode
  \hbox to.77778em{%
    \hfil\vrule
  \vbox to.675em{\hrule width.6em\vfil\hrule}%
  \vrule\hfil}}
\gdef\proofSymbol{\openbox}
\newcommand{\proofname}{Proof}
\newenvironment{proof}[1][\proofname]{%
  \th@nonumberplain
  \def\theorem@headerfont{\itshape}%
  \normalfont
  \theoremsymbol{\ensuremath{_\blacksquare}}%
  \@thm{proof}{proof}{#1.}%
}{\@endtheorem}
\newtheoremstyle{my}%
{\item[\hskip\labelsep \theorem@headerfont ##1~##2\theorem@separator]}%
{\item[\hskip\labelsep {\theorem@headerfont ##1~##2} \paren{##3}\theorem@separator]}
\makeatother
\theoremstyle{my}
\theoremseparator{.}

\usepackage[capitalise,noabbrev]{cleveref}
\crefname{section}{\S}{\SS}
% Protect pages and equations from capitalization
\crefname{page}{{p}age}{{p}ages}
\crefname{equation}{{e}q.}{{e}qs.}

\makeatletter
\DeclareRobustCommand{\labelvref}{\cref@vref{labelcref}}
\DeclareRobustCommand{\labelfullref}{\cref@fullref{labelcref}}
\makeatother

\newtheorem{thm}{Theorem}[chapter]
\newtheorem{prop}[thm]{Proposition}
\newtheorem{dfn}[thm]{Definition}
\newtheorem{restr}[thm]{Restriction}
\newtheorem{obs}[thm]{Observation}

\usepackage{autonum}
% Patch varioref commands for autonum
\makeatletter
\autonum@generatePatchedReferenceCSL{Cref}
\autonum@generatePatchedReferenceCSL{labelcref}
\autonum@generatePatchedReferenceCSL{vref}
\autonum@generatePatchedReferenceCSL{Vref}
\autonum@generatePatchedReferenceCSL{labelvref}
\autonum@generatePatchedReferenceCSL{fullref}
\autonum@generatePatchedReferenceCSL{Fullref}
\autonum@generatePatchedReferenceCSL{labelfullref}
\makeatother

\renewcommand*{\reftextfaraway}[1]{on~\cpageref{#1}}

\usepackage{algorithm,algpseudocode}
\makeatletter
\@addtoreset{algorithm}{chapter}
\makeatother
\renewcommand{\thealgorithm}{\thechapter.\arabic{algorithm}} 

\usepackage[
  style=authoryear-comp,
  maxnames=2,
  maxbibnames=99,
  uniquename=init,
  uniquelist=false,
  backref=false,
  doi=true,
  isbn=true,
  eprint=true,
  natbib=true,
  hyperref=true,
  autolang=hyphen,
  backend=biber
]{biblatex}
\bibliography{markov}
\AtEveryBibitem{% Clean up the bibtex rather than editing it
  \clearlist{address}
  \clearfield{date}
  \clearlist{location}
  \clearfield{month}
  \ifentrytype{book}{}{% Remove editor except for books
    \ifentrytype{inbook}{}{
      \clearname{editor}
    }
  }
  \iffieldundef{doi}{}{
    \clearfield{url}
    \clearfield{isbn}
    \clearfield{issn}
    \clearfield{eprint}
  }
  \iffieldundef{eprint}{}{
    \clearfield{url}
    \clearfield{isbn}
    \clearfield{issn}
  }
  %  \clearfield{url}}{}
}
\DefineBibliographyStrings{english}{%
  bibliography = {References}
}
\renewcommand*{\bibopenparen}{\textup(}
\renewcommand*{\bibcloseparen}{\textup)}

\begin{document}

\frontmatter

%--------------------------------------------------------------------------------------
% The title page
%--------------------------------------------------------------------------------------
\thispagestyle{empty}
\begin{center}
  \includegraphics[width=60mm,keepaspectratio]{bme_logo}\\
  \vspace{0.3cm}
  \textbf{Budapest University of Technology and Economics}\\
  \textmd{Faculty of Electrical Engineering and Informatics}\\
  \textmd{Department of Measurement and Information Systems}\\[5cm]
  \vspace{0.4cm}
  {\Huge \sffamily \bfseries Stochastic Analysis of Petri Nets}\\[0.8cm]
  \vspace{0.5cm}
  \textit{\Large Project Report}\\[4cm]
  \noindent{
    \renewcommand{\arraystretch}{1}
    \begin{tabular}{@{}c@{}c@{}}
      \makebox[0.49\linewidth]{\emph{Author}} & \makebox[0.49\linewidth]{\emph{Advisor}} \\
      \makebox[0.49\linewidth]{Kristóf Marussy} & \makebox[0.49\linewidth]{András~Vörös} \\
    \end{tabular}
  }
  \vfill
  {\large \today}
\end{center}

%%% Local Variables:
%%% mode: latex
%%% TeX-master: "markov"
%%% End:


\begin{KeepFromToc}
  \tableofcontents
\end{KeepFromToc}

\mainmatter

\chapter{Background}

\section{Continous-time Markov chains}

\begin{dfn}
  A \emph{continous-time Markov chain} \paren{\emph{\ctmc}}
  $\{X(t) \in S : t \ge 0\}$ over the finite or countably infinite
  state space $S = \{s_1, s_2, \ldots\}$ is a stochastic process which
  satisfies the Markovian property
  \begin{equation}
    \Pr \{ X(t') = s_{j} \mid X(t) = s_{i}, \forall u \in U : X(u) \}
    = \Pr \{ X(t') = s_{j} \mid X(t) = s_{i} \}
  \end{equation}
  for all $t' > t$ and $U \subseteq [0, t)$.
\end{dfn}

\begin{dfn}
  A continous-time Markov chain is \emph{(time-)homogenous} if its
  transition probabilities are invariant with respect to time, i.e.
  \begin{equation}
    \Pr \{ X(t') = s_{j} \mid X(t) = s_{i} \} = p(t' - t)[i, j] \text.
  \end{equation}
\end{dfn}

We will assume that the \ctmc s we consider are homogenous and have a
finite state space, unless otherwise indicated.

Let us introduce to shorthand $\vec{\uppi}(t)$ to denote the vector of
probabilities
\begin{equation}
  \pi(t)[i] = \Pr\{X(t) = s_i\} \text,
\end{equation}
which will be called the \emph{probablity vector} at time $t$.

The row-stochastic matrix
\begin{equation}
  P(\tau) = \bigl(p(\tau)[i, j]\bigr)_{i, j = 1}^{\lvert S \rvert}
  \label{eq:intro:p_tau}
\end{equation}
describes the state transitions of the \ctmc\ over a time period
$\tau$. More concretely,
\begin{equation}
  \vec{\uppi}(t') = \vec{\uppi}(t) P(t' - t) \text.
\end{equation}

If $X(t)$ is a homogenous \ctmc\ over a finite state space, there exists
an \emph{infinitesimal generator matrix}
$Q \in \mathbb{R}^{\lvert S \rvert \times \lvert S \rvert}$ such that
\begin{equation}
  P(\tau) = \exp(\tau Q) = \sum_{k = 1}^\infty \frac{\tau^k Q^k}{k!} =
  I + \tau Q + o(\tau) \text,
\end{equation}
where $\exp(\cdot)$ denotes the matrix exponential function. This
matrix satisfies
\begin{enumerate}
\item If $i \ne j$, $q[i, j] \ge 0$;
\item
  $q[i, i] = -\bigl( q[i, 1] + q[i, 2] + \cdots + q[i, i - 1] + q[i, i + 1] +
  \cdots + q[i, \lvert S \rvert] \bigr)$
  for all $0 \ge i \ge \lvert S \rvert$, that is,
  $Q \vec{e}^T = \vec{0}^T$.
\end{enumerate}
Matricies having these two properties are called Q-matrices.

The \emph{sojourn time} from state $s_j$
\begin{equation}
  T(i) = \inf_{t > 0} \{ X(t) \ne s_i \mid X(0) = s_i \}
\end{equation}
has the memoryless property
\begin{equation}
  \Pr\{T(i) > t + \epsilon \mid T(i) > t\} = \Pr\{T(i) > \epsilon\} \text.
\end{equation}
and is exponentially distributed with rate $1 / \E[T(i)] = -q(i,
i)$.

Moreover, if a state change occurs at time $t$, that is,
$X(t-) \ne X(t+)$, then $X(t+) = s_j$ given $X(t-) = s_i$ with
probability $-q[i, j] / q[i, i]$. Thus we can use the matrix $Q$ to
visualize the behaviour of the \ctmc\ in state $s_i$ as a sojourn in
$s_i$ with exponentially distributed duration followed by a random
jump to some state $s_j \ne s_i$.

The configuration above is equivalent to a ``race'' between $n - 1$
exponentially distributed delays with rates
$q[i, 1], q[i, 2], \ldots, q[i, i - 1], q[i, i + 1], \ldots, q[i,
\lvert S \rvert]$, respectively, where the
first delay to complete causes the \ctmc\ to shift to the destination
state associated with the delay. The latter characterization turns out
to be the key intution for modelling asynchronous distributed systems
with continous-time Markov chains.

The time evolution of the probability vector satisfies the
differential equation
\begin{equation}
  \frac{d \vec{\uppi}(t)}{dt} = \vec{\uppi}(t) Q \text.
  \label{eq:intro:diffeq}
\end{equation}
This gives us a third way of understanding the structure of the
infintesimal generator matrix: the value $q[i, j]$ corresponds to a
``flow amount'' of probability mass from the state $s_i$ to
$s_j$. Off-diagonal entries $q[i, j]$ of $Q$ have positive flow amount
if the \ctmc\ can eventually leave $s_i$ towards $s_j$ and zero
otherwise, while negative entries $q[i, i]$ along a diagonal
correspond to ``leaking'' of probability mass from $s_i$ to other
states. A zero diagonal entry represents a final state which cannot be
left.

\begin{dfn}
  The state $s_j \in S$ of the \ctmc\ $X(t)$ is \emph{accessible} from
  $s_i \in S$ \paren{written as $s_i \to s_j$} if $p(\tau)[i, j] > 0$ for
  some $\tau > 0$.
\end{dfn}

\begin{dfn}
  The states $s_i, s_j \in S$ of the \ctmc\ $X(t)$ \emph{communicate}
  \paren{written as $s_i \leftrightarrow s_j$} if $s_i \to s_j$ and
  $s_j \to s_i$.
\end{dfn}

\begin{dfn}
  The \ctmc\ $X(t)$ is \emph{irreducible} if all of its states pairwise
  communicate.
\end{dfn}

If $X(t)$ is irreducible, then its infinitesimal generator matrix $Q$
has rank $\lvert S \rvert - 1$.

\subsection*{Analysis tasks}

The two tasks most frequently associated with continous-time Markov
chains are \emph{transient} and \emph{steady-state} analysis.

In transient problems, the probability distribution vector
$\vec{\uppi}(t)$ at time $t$ is sought given the initial probability
distribution vector $\vec{\uppi}(0) = \vec{\uppi}_0$. In practice,
instead of directly calculating the matrix exponential
\labelvref{eq:intro:p_tau}, either the initial value problem based on
\vref{eq:intro:diffeq}
\begin{equation}
  \frac{d \vec{\uppi}(t)}{dt} = \vec{\uppi}(t) Q \text{ subject
    to } \vec{\uppi}(0) = \vec{\uppi}_0 \label{eq:intro:diffeq-initial}
\end{equation}
is solved, or the \emph{uniformization algorithm}
\begin{equation}
  \vec{\uppi}(t) = \sum_{k = 1}^{\infty} \vec{\uppi}_0 \mleft( I +
  \frac{1}{\gamma} Q \mright)^k \frac{(\gamma t)^k}{k!} e^{-\gamma t}
\end{equation}
is used with some appropriately chosen $\gamma > 0$.

In steady-state analysis, the limit
\begin{equation}
  \vec{\uppi}_{\infty} = \lim_{t \to \infty} \vec{\uppi}(t)
\end{equation}
is computed. If the \ctmc\ is irreducible, this limit is well-defined
and independent of the choice of the initial probability distribution
$\vec{\uppi}(0)$.

The equlibirium condition
\begin{equation}
  \frac{d \vec{\uppi}}{dt} = 0
\end{equation}
together with \eqref{eq:intro:p_tau} yields the homogenous system of
linear equations
\begin{equation}
  \vec{\uppi}_\infty Q = \vec{0} \text{ subject to } \sum_{i =
    0}^{\lvert S \rvert} \pi_\infty[i] = 1 \text{ and } \forall 1 \le
  i \le \lvert S \rvert :  \pi_\infty[i] > 0 \label{eq:intro:steadystate}
\end{equation}
for finding the steady-state probability distribution vector. In other
words, $\vec{\uppi}_\infty$ is the (unique) vector in the null space
of $Q$ which is a valid probability distribution vector, that is,
whose components are all nonnegative and sum to $1$.

While these two task primitives are often useful on their own, more
complicated inquiries, such as model checking for stochastic temporal
logical formulae also use them as subroutines.

\section{Petri nets}

\begin{dfn}
  A \emph{Petri net} is a tuple
  \begin{equation}
    \PN = (P, T, \Pi, W^{-}, W^{+}, W^{H}, M_0) \text,
  \end{equation}
  where
  \begin{itemize}
  \item $P$ and $T$ are finite set of \emph{places} and
    \emph{transitions}, respectively;
  \item $\Pi: P \to \mathbb{N}$ is the \emph{priority function}, which
    map places to nonnegative integer priorities;
  \item $W^{-}, W^{+}, W^{H} : T \to \Bag(P)$ are the \emph{input,
      output} and \emph{inhibitor arc functions}, which map
    transitions to ``bags'' \paren{multisets} of places;
  \item $M_0 \in \Bag(P)$ is the \emph{initial marking}.
  \end{itemize}
\end{dfn}

For convenience, we will write $w^-[j, i]$, $w^+[j, i]$ and
$w^H[j, i]$ for the multiplicity of the place $p_i \in P$ in
$W^-(t_j)$, $W^+(t_j)$ and $W^H(t_j)$, respectively, where $t_j \in T$
is a transition. A \emph{marking} is a multiset of places
$M \in \Bag(P)$. We write $m(i)$ for the multiplicity of $p_i$ in the
marking $M$. The set of \emph{input places} $\leftTok{t_j}$,
\emph{output places} $\rightTok{t_j}$ and \emph{inhibitor places}
$\leftInh{t_j}$ of a transition $t_j$ are defined as
\begin{equation}
  \left\{ \begin{gathered}
    \begin{aligned}
      \leftTok{t_j} &= \{ p_i \in P : w^-[j, i] > 0 \}\text, &
      \rightTok{t_j} &= \{ p_i \in P : w^+[j, i] > 0 \}\text,
    \end{aligned} \\
    \leftInh{t_j} = \{ p_i \in P : w^H[j, i] > 0 \}\text,
  \end{gathered} \right.
\end{equation}
while set of \emph{input} $\leftTok{p_i}$,
\emph{output} $\rightTok{p_i}$ and \emph{inhibited transitions}
$\rightInh{p_i}$ of a place $p_i$ are defined symmetrically.

\subsection*{Token game}

\begin{dfn}
  The transition $t_j$ has \emph{concession} in the marking $M$ if
  \begin{equation}
    \forall p_i \in \leftTok{t_j} : m(i) \ge w^-[j, i] \text{ and }
    \forall p_i \in \leftInh{t_j} : m(i) < w^H[j, i] \text.
  \end{equation}
\end{dfn}

\begin{dfn}
  The transition $t_j$ is \emph{enabled} in the marking $M$ if it has
  concession in $M$ and no transition $t_{k}$ has concession in $M$
  such that $\Pi(t_{k}) > \Pi(t_j)$.
\end{dfn}

If the transition $t_j$ is enabled in $M$, it can \emph{fire} to yield
another marking
\begin{equation}
  M' = M - W^-(T_j) + W^+(T_j)
\end{equation}
which is written as $M \xrightarrow{t_j} M'$. We can visualize the
marking $M$ as a bag of ``tokens'' located on some of the places
$P$. The firing of $t_j$ removes $W^-(t_j)$ tokens from
$\leftTok{t_j}$ and places $W^{+}(t_j)$ tokens to $\rightTok{t_j}$,
but only if there are less than $W^H(t_j)$ tokens on $\leftInh{t_j}$
and no higher priority transition has concession.

\begin{dfn}
  The marking $M'$ is \emph{reachable} from the marking $M$ if there
  exists a \paren{possibly empty} sequence of transitions
  $t_{i_1}, t_{i_2}, \ldots, t_{i_n}$ such that
  \begin{equation}
    M \xrightarrow{t_{i_1}} M_1 \xrightarrow{t_{i_2}} M_2
    \xrightarrow{t_{i_3}} \cdots \xrightarrow{t_{i_{n - 1}}} M_{n - 1}
    \xrightarrow{t_{i_n}} M_n = M'\text.
  \end{equation}
\end{dfn}

\begin{dfn}
  The \emph{reachability set} $\RS$ of a Petri net is the set of all
  markings reachable from the initial marking $M_0$.
\end{dfn}

The rechability set may contain an enormous amount of markings even if
the Petri net is of modest size. This phenomenon is called \emph{state
  space explosition} \textbf{TODO citation?}, which is a significant
obstacle in the analysis of Petri nets. The maximum size of the models
which can be analyzed using a given pool of resources may be incresead
by chosing an efficient representation of the reachability set instead
of explicit storage of all reachable markings. \textbf{TODO citation!}

\section{Stochastic Petri nets}

\begin{dfn}
  A \emph{Generalized Stochastic Petri Net} \paren{\emph{\gspn}} is a
  tuple
  \begin{equation}
    \GSPN = (P, T, \Pi, \Lambda, W^-, W^+, W^H, M_0) \text,
  \end{equation}
  where $(P, T, \Pi, W^-, W^+, W^H, M_0)$ is a Petri net and $\Lambda:
  T \to \mathbb{R}^{+}$ is called the \emph{rate} or \emph{weight}
  function.
\end{dfn}

Transitions with $\Pi = 0$ are called \emph{timed} transitions, while
transitions with $\Pi \ge 1$ are called \emph{immediate}
transitions. The value $\Lambda(t_j)$, abbreviated as $\lambda_j$, is
referred to as the \emph{rate} associated with $t_j$ if $t_j$ is
timed or the \emph{weight} of $t_j$ if $t_j$ is immediate.

It is customary to draw timed transitions as open rectangles, while
immediate transition are shown as black bars or filled rectangles.

\begin{dfn}
  A marking $M$ is \emph{vanishing} if there exists an immediate
  transition with concession. Non-vanishing markings are called
  \emph{tangible}.
\end{dfn}

The \emph{tangible reachability set} $\TRS \subseteq \RS$ is the set
of all reachable tangible states. The dynamics of a \gspn\ are
described with by continous-time Markov chain $X(t)$ whose set of
states $S$ is the tangible reachability set $\TRS$ and the initial
state $X(0)$ is the initial marking $M_0$. The states
$S = \{s_0, s_1, s_2, \ldots\}$ and tangible reachable markings
$\TRS = \{M_0, M_1, M_2, \ldots\}$ will be used interchangeably in the
rest of this report, specifically, we will assume that the state $s_0$
represents the initial marking $M_0$ in the \ctmc.

In tangible states, timed transitions that are enabled fire with
delays exponentially distributed according to their $\lambda$
rates. If the resulting marking is vanishing, the enabled immediate
transitions fire until a tangible marking is reached, which will
become the next state of the \ctmc. If multiple immediate transitions
are enabled in a vanishing marking, one is selected and fired randomly
with probability proportional to its $\lambda$ weight. Thus, state
changes of the \ctmc\ corresponding to the \gspn\ consist of the
firing of a timed transition followed by a -- possibly empty, if the
output marking of the timed transition is tangible -- random sequence
of immiditate transition firings.

See \vref{fig:intro:gspn-timedomain} for an example of a \gspn
's time-domain behavior.

\begin{figure}
  \centering
  \begin{tikzpicture}
    \petriP(p1)(0,0){$p_1$}<1>
    \petriP(p2)(3,0){$p_2$}
    \petriP(p3)(6,0){$p_3$}

    \petriT(t1)(1.5,0){$t_1$}{\lambda_1}
    \petriT(t2)(4.5,0){$t_2$}{\lambda_2}
    \petriT(t3)(3,1.75){$t_3$}<1>{1}
    \petriT(t4)(4.5,-1.75){$t_4$}<1>{2}

    \begin{pgfonlayer}{connections}
      \begin{scope}[-{Straight Barb[angle=60:3pt 3]}]
        \draw (p1) edge (t1) edge (p2) edge node [above] {2} (t2)
        edge (p3);
        \draw (p3) edge [in=0,out=130] node [above] {2} (t3);
        \draw (t3) edge [out=180,in=50] (p1);
        \draw (p3) edge [in=0,out=-100] (t4);
        \draw (t4) edge [out=180,in=-80] (p2);
      \end{scope}
    \end{pgfonlayer}
  \end{tikzpicture}\par
  \vspace{1ex}
  \begin{tikzpicture}
    \timing [yscale=1.4,xscale=4.8,timing/font=\LARGE,timing/slope=0.05] at (0,0)
    {D{$(1,0,0)$}N(a)D{$(0,2,0)$}N(b)0.9D{$(0,1,1)$}N(c),Dd{$(0,1,1)$}N(d)D{$(1,0,0)$}};
    \coordinate (aa) at ($(a)+(0,1)$);
    \coordinate (bb) at ($(b)+(0,1)$);
    \coordinate (cc) at ($(c)+(0,1)$);
    \coordinate (dd) at ($(d)+(0,1)$);
    \begin{scope}[-{Bar[width=10pt]}]
      \path [draw,{Bar[width=10pt]}-{Bar[width=10pt]}] (aa) -- node
      [above] {$\mathbb{E}[T] = \lambda_1$} (bb);
      \path [draw] (bb) -- node [above] {$\mathbb{E}[T] = \lambda_2$} (cc);
      \path [draw] (cc) -- node [above] {$\mathbb{E}[T] = \lambda_2$} (dd);
    \end{scope}
    \begin{scope}[node distance=2pt]
      \node [below=of a,anchor=north west] {$t_1$ fire};
      \node [below=of b,anchor=north west] {$t_2$ fire};
      \node [below=of c,anchor=north west,text width=4cm] {$t_2$ and
        $t_4$ fire \newline
        (with $\Pr = \nicefrac{2}{3}$)};
      \node [below=of d,anchor=north west, text width=2cm] {$t_2$,
        $t_3$ fire \newline
        ($\Pr = \nicefrac{1}{3}$)};
    \end{scope}
    \begin{scope}[dashed]
      \draw (a) --  ++ (0,-1.2);
      \draw (b) --  ++ (0,-1.2);
      \draw (c) --  ++ (0,-1.2);
      \draw (d) --  ++ (0,-1.2);
    \end{scope}
  \end{tikzpicture}
  \caption{Behaviour of a \gspn. Markings are written in the form
    $(m_1, m_2, m_3)$ to show the number of tokens at the places
    $p_1$, $p_2$ and $p_3$, respectively. The trace starts from the
    initial marking shown on the net.}
  \label{fig:intro:gspn-timedomain}
\end{figure}

Let $p_{\text{immediate}}(M_i, M_j)$ denote the probabilty that the
\gspn\ reaches the tangible marking $M_j$ starting from the tangible
or transient marking $M_i$ by a possible empty sequence of immediate
transition firings. Observe that when $M_i$ is tangible,
$p_{\text{immediate}}(M_i, M_j) = 1$ if $i = j$ and $0$ otherwise.

The behaviour of the \gspn\ is well-defined if
$p_{\text{immediate}}(M, M')$ is well-defined for all $M \in \RS$,
$M' \in \TRS$ \citep{DBLP:journals/tse/TeruelFP03} and the net is free
of \emph{confusion}.

The infinitesimal generator of a \gspn\ can be written as
\begin{empheq}[left=\empheqlbrace]{gather}
  q_{O}[i, j] = \sideset{}{'}\sum_{M_i \xrightarrow{t_k} M'} \lambda_k
  p_{\text{immedidate}}(M', M_j)
  \text, \label{eq:intro:qd-element} \\
  Q = Q_{O} + Q_{D} = Q_{O} - \diag Q_O \vec{e}^T \text, \label{eq:intro:qd-qo}
\end{empheq}
where the summation is performed over all \emph{timed} transitions
$t_k$ enabled in the state $M_i$, while $Q_O$ and $Q_D$ stand for the
off-diagonal and diagonal parts of $Q$, respectively.
x
The storage of $Q$ for \gspn\ is particularly challenging due to state
space explosition, because the number of entries, therefore the amount
of memory required for the explicit storage of the inifinitesimal
generator matrix, grows quadratically in the number of tangible
reachable markings. This behaviour is opposed to the amount of memory
required for explicit storage of $\TRS$, which grows merely
linearly. Therefore, efficient representation of $Q$ becomes a problem
even in analysis task where explicit storate of $\TRS$ (or $\RS$) is
possible.

\subsection*{Conflict and confusion}

A pair of immediate transitions $t_1$, $t_2$ cannot fire
``simultaneously'' if only one of traces
$M \xrightarrow{t_1} M' \xrightarrow{t_2} M'''$ and
$M \xrightarrow{t_2} M'' \xrightarrow{t_1} M'''$ is possible. In this
case, we must evaluate the effects of $t_1$ and $t_2$ separately. This
notion is formalized as the effective conflict relation, which is
noncommutative, unreflexive and intransitive due to the presence on
priority levels and inhibitor arcs.

\begin{dfn}[\cite{DBLP:journals/tse/ChiolaMBC93}]
  The transitions $t_1$ and $t_2$ are in \emph{effective conflict} in
  the marking $M$ \paren{written as $t_1 \EffC{M} t_2$} if and only if
  $t_2$ has concession in $M$, $t_1$ is enabled in $M$, $M
  \xrightarrow{t_1} M'$ and $t_2$ has no concession in $M'$.
\end{dfn}

If $t_1$ and $t_2$ are not in conflict, but the final outcome of the
sequence of immediate firings depends on the order in which $t_1$ and
$t_2$ are fired, the \gspn\ is in \emph{confusion}. \textbf{TODO Petri
  net with confusion.} More formally:

\begin{dfn}[\cite{DBLP:journals/tse/ChiolaMBC93}]
  The transition s $t_1$ and $t_2$ are in \emph{assymetric confusion}
  in the marking $M$ if and only if it is \emph{not} the case that
  $t_1 \EffC{M} t_2$, $t_1$ is enabled in $M$,
  $M \xrightarrow{t_1} M'$ and there exists a transition $t_3 \in T$
  such that $t_2 \EffC{M'} t_3$.
\end{dfn}

Algorithms that are used to determine $p_{\text{immediate}}$
\citep{DBLP:journals/tse/ChiolaMBC93, DBLP:journals/tse/TeruelFP03}
usually consider confusion a semantic error and help the modeller
choose priorities for the immediate transitions in order to avoid
confusion. If the net is not confused, immediate transitions not in
pairwise conflict can be fired in \emph{batches}, while weights
$\lambda$ are used to randomly resolve conflicts.

The trivial case of a net without confusion is one without any
immedaite transitions, i.e.~$\Pi \equiv 0$ for all transitions. Such
$\gspn$s are called (ordinary) stochastic Petri nets or \spn s.

\subsection*{Superposed \textls[35]{GSPNs}}

\begin{dfn}[\cite{DBLP:journals/tse/Buchholz99}]
  A \emph{superposed \gspn\ \paren{\sgspn}} is a tuple
  \begin{equation}
    \SGSPN = (P, \mathcal{P}, T, \Pi, \Lambda, W^-, W^+, W^H, M_0)
    \text,
  \end{equation}
  where $(P, T, \Pi, \Lambda, W^-, W^+, W^H, M_0)$ is a \gspn\ and
  $\mathcal{P} = \{P\loc{1}, P\loc{2}, \ldots, P\loc{J}\}$ is a
  disjoint partition $P\loc{1} \uplus P\loc{2} \uplus \cdots \uplus
  P\loc{J}$ of $P$.
\end{dfn}

Transitions
\begin{equation}
  T\loc{j} = \TL\loc{j} \uplus \TS\loc{j} =
  \bigcup_{p \in P\loc{j}} \leftTok{p} \cup \rightTok{p} \cup
  \rightInh{p}
\end{equation}
along with the projected versions $\Pi\loc{j}$, $\Lambda\loc{j}$,
$W^{-(j)}$, $W^{+(j)}$, $W^{H(j)}$, $M\loc{j}_0$ of $\Pi$, $\Lambda$,
$W^-$, $W^+$, $W^H$, $M_0$ comprise the $j$th \emph{local net}
\begin{equation}
  \LN\loc{j} = (P\loc{j}, T\loc{j}, \Pi\loc{j}, \Lambda\loc{j},
  W^{-(j)}, W^{+(j)}, W^{H(j)}, M\loc{j}_0) \text.
\end{equation}
The transitions $\TL\loc{j} \subseteq T\loc{j}$ belong to only a
single local net and are called \emph{local transitions}, while
elements of $\TS\loc{j} = T\loc{j} \setminus \TL\loc{j}$ are
\emph{synchronized transitions}.

\begin{restr}
  Synchronized transitions of an \sgspn\ must be timed, that is,
  \begin{equation}
    \exists 1 \le j \le J: t \in \TS\loc{j} \implies \Pi(t) =
    \Pi\loc{j}(t) = 0 \text.
  \end{equation}
  Equivalently, every timed transition must be internal to its local
  net.
\end{restr}

\section{Tensors and Kronecker algebra}
\label{sec:intro:tensors}

\begin{dfn}
  The \emph{Kronecker product} of two matrices is defined as
  \begin{equation}
    \begin{gathered}
      \otimes: \mathbb{R}^{m_1 \times n_1} \times \mathbb{R}^{m_2 \times
        n_2} \to \mathbb{R}^{m_1 m_2 \times n_1 n_2}\text, \\
      (A \otimes B)[i_1 m_2 + i_2, j_1 n_2 + j_2] = a[i_1, j_1] b[i_2,
      j_2] \text.
    \end{gathered}
  \end{equation}
\end{dfn}

The matrix $A \otimes B$ has a block structure with $i_1 \times j_1$
block, where each block has $i_2 \times j_2$ elements. This can be
written as
\begin{equation}
  A \otimes B = \begin{pmatrix}
    a[1,1] B & a[1,2] B & \cdots & a[1,n_1] B \\
    a[2,1] B & a[2,2] B & \cdots & a[1,n_1] B \\
    \vdots & \vdots & \ddots & \vdots \\
    a[m_1,1] B & a[m_1,2] B & \cdots & a[m_1,n_1] B    
  \end{pmatrix} \text,
\end{equation}
where each entry of the right-hand side is a $m_2 \times n_2$ matrix.

It will be occasionally useful to refer to entries of $A \times B$ by
multi-indices, that is,
\begin{equation}
  (A \otimes B)[i_1, i_2, j_1, j_2] = (A \otimes B)[\vec{i}, \vec{j}]
  = (A \otimes B)[i_1 m_2 + i_2, j_1 n_2 + j_2] \text,
\end{equation}
where $\vec{i} = (i_1, i_2)$ and $\vec{j} = (j_1, j_2)$. Multi-index
notation can be extended to Kronecker products of more than two
matrices, for example,
\begin{multline}
  (A \otimes B \otimes C)[i_1, i_2, i_3, j_1, j_2, j_3] \\ = (A
  \otimes B \otimes C)[i_1 m_2 m_3 + i_2 m_3 + i_3, j_1 n_2 n_3 + j_2
  m_3 + j_3] \\ = a[i_1, j_1] b[i_2, j_2] c[i_3, j_3] \text,
\end{multline}
where $A \in \mathbb{R}^{m_1 \times n_1}$, $B \in \mathbb{R}^{m_2
  \times n_1=2}$, $C \in \mathbb{R}^{m_3 \times n_3}$.

The Kronecker product distributes over the matrix multiplication,
\begin{equation}
  \label{eq:intro:kronecker-distrib}
  (A \otimes B) (C \otimes D) = (AC) \otimes (BD)
\end{equation}
if the matrices $A$, $C$ and $B$, $D$ are compatible.

An important special case of the Kronecker product is the product of
two row or column vector, which is called the tensor product:

\begin{dfn}
  The \emph{tensor product} of two vectors is defined as
  \begin{equation}
    \otimes: \mathbb{R}^{n_1} \times \mathbb{R}^{n_2} \to
    \mathbb{R}^{n_1 n_2}\text ,
    \quad (\vec{u} \otimes \vec{v})[i_1 n_2 + i_2] = u[i_1] v[i_2] \text.
  \end{equation}
\end{dfn}

The tensor product can be used to form the product of two vector
spaces:

\begin{dfn}
  The \emph{tensor product} of two vector spaces is the vector space
  generated by the tensor products of their vectors,
  \begin{equation}
    \mathbb{R}^{n_1} \otimes \mathbb{R}^{n_2} = \langle \vec{u} \otimes
    \vec{v} : \vec{u} \in \mathbb{R}^{n_1}, \vec{v} \in
    \mathbb{R}^{n_2} \rangle \text.
  \end{equation}
\end{dfn}

If $\mathscr{B} = \{\vec{b}_1, \vec{b}_2, \ldots, \vec{b}_{n_1} \}$
and $\mathscr{C} = \{\vec{c}_1, \vec{c}_2, \ldots, \vec{c}_{n_2} \}$
are bases in $\mathbb{R}^{n_1}$ and $\mathbb{R}^{n_2}$, respectively,
then $\mathscr{D} = \{\vec{b} \otimes \vec{c} : \vec{b} \in
\mathscr{B}, \vec{c} \in \mathscr{C} \}$ is a basis in
$\mathbb{R}^{n_1} \otimes \mathbb{R}^{n_2}$.

%%% Local Variables:
%%% mode: latex
%%% TeX-master: "markov"
%%% End:


\chapter{Analysis based on Kronecker products}

\section{\textit{Q} matrix representation with Kronecker
  products}

The dynamic behaviour of the superposed generalized stochastic Petri
net $\SGSPN$ can be studied using the continous-time Markov whose
state space is tangible reachability set $\TRS$ of the $\SGSPN$ and
whose infinitesimal generator matrix $Q$ is
\begin{empheq}[left=\empheqlbrace]{gather}
  q_{O}[i, j] = \sumprime_{M_i \xrightarrow{t_k} M'} \lambda_k
  p_{\text{immedidate}}(M', M_j)
  \text, \tag{\ref{eq:intro:qd-element} revisited} \\
  Q = Q_{O} + Q_{D} = Q_{O} - \diag Q_O \vec{e}^T \text.
  \tag{\ref{eq:intro:qd-qo} revisited}
\end{empheq}
Unfortunately, this matrix requires space proportional to the square
of $\TRS$ if dense matrix storage is employed, and reductions os
required storage space by sparse matrix storage methods may not be
sufficient due to the large number of transitions that may be enabled
in a given state.

By exploiting the partitioned structure of superposed \gspn s, the
required space can be reduced. Consider the \emph{restriction} of
$\TRS$, $\TRS\loc{j} \subset \Bag(P\loc{j})$, which is the set of
possible markings the local net $\LN\loc{j}$ can have when the
superposed net $\SGSPN$ is in a tangible reachable state. Moreover,
let $M\loc{j} \in \Bag(P\loc{j})$ denote the restriction of the marking
$M$ to $\LN\loc{j}$. In set-builder notation,
\begin{equation}
  \TRS\loc{j} = \{ M\loc{j} : M \in \TRS \} \text.
\end{equation}

\begin{obs}\label{obs:kronecker:pimmediate-j}
  Let $p\loc{j}_{\text{immediate}}(M\loc{j}, M^{\prime (j)})$ denote
  the probability of state change from $M\loc{j} \in \Bag(P\loc{j})$ to
  $M^{\prime (j)} \in \TRS\loc{j}$ due to immediate transitions in
  $\LN\loc{j}$. Every immediate transition is local in an \sgspn and
  can cause state changes only in its local net, independently of
  other local nets. Therefore, we may write the global immediate
  state-change probabilities as
  \begin{equation}
    p_{\text{immediate}}(M, M') = \prod_{j = 1}^J
    p\loc{j}_{\text{immediate}}(M\loc{j}, M^{\prime (j)})
    \text. \label{eq:kronecker:pimmediate-j} 
  \end{equation}
\end{obs}

\subsection{Tangible reachability sets with Descartes product form}

Suppose that the tangible reachability set may be written as
\begin{equation}
  \TRS = \TRS\loc{1} \times \TRS\loc{2} \times \cdots \times
  \TRS\loc{j} \text,
\end{equation}
in other words, every combination of local net markings is tangible
and reachable. Then the off-diagonal part of $Q$ can be written as
\begin{equation}
  Q_O = \bigoplus_{j = 1}^J Q_L\loc{j} + \sum_{t \in \TS} \bigotimes_{j
  = 1}^J Q_t\loc{j} \text, \label{eq:kronecker:simple-local-global}
\end{equation}
where $Q_L\loc{j}$ describes the effect of \emph{local} transitions on
$\LN\loc{j}$, while $Q_t\loc{j}$ corresponds to a synchronized
transition $t \in \TS = \bigcup_{j = 1}^J \TS\loc{j}$ in
$\LN\loc{j}$. The matrices $Q_L\loc{j}$ and $Q_t\loc{j}$ are of size
$n_j \times n_j = \lvert \TRS\loc{j} \rvert \times \lvert \TRS\loc{j}
\rvert$ and together represent the $n \times n = \lvert \TRS \rvert
\times \lvert \TRS \rvert$ matrix $Q_O$. We have used \emph{Kronecker
  sum} notation to write the contribution of local transitions, i.e.
\begin{equation}
  \begin{aligned}
    \bigoplus_{j = 1}^J Q_L\loc{j} &= \sum_{j = 1}^J I^{n_1 \times
      n_1} \otimes I^{n_2 \times n_2} \otimes \cdots \otimes I^{n_{j -
        1} \times n_{j - 1}} \otimes Q_L\loc{j} \otimes I^{n_{j + 1}
      \times
      n_{j + 1}} \otimes \cdots \otimes I^{n_J \times n_J} \\
    &= \sum_{j = 1}^J I^{\prod_{k = 1}^{j - 1} n_k \times \prod_{k =
        1}^{j - 1}} \otimes Q_L\loc{j} \otimes I^{\prod_{k = j +
        1}^{J} n_k \times \prod_{k = j + 1}^{J}} \text.
  \end{aligned}
\end{equation}

Terms describing local transitions are the Kronecker products of all
but one identity matrices, because local transitions cause no state
change in local nets other than the one they belong to. That is,
\begin{equation}
  q_L\loc{j}[x, y] = \sumprime_{M_x \xrightarrow{t_k}
      M'} \lambda_k p_{\text{immediate}}\loc{j}(M', M_y) \text{, where
      $t_k \in \LN\loc{j},$} \label{eq:kronecker:ql}
\end{equation}
$M_x, M_y \in \TRS\loc{j}$ and the summation is done over all
enabled timed transitions local to $\LN_j$. In $\LN_j$ contains no
immediate transitions, \cref{eq:kronecker:ql} can be simplified to
\begin{equation}
  q_L\loc{j}[x, y] = \sumprime_{M_x \xrightarrow{t_k} M_y} \lambda_k
  \text{, where $t_k \in \LN\loc{j}$.}
\end{equation}

If the transition rates, are marking dependent, that is, $\Lambda: T
\times \Bag(P) \to \mathbb{R}^{+}$, then we can write
\begin{equation}
  q_L\loc{j}[x, y] = \sumprime_{M_x \xrightarrow{t_k}
    M'} \lambda_k(M_x) p_{\text{immediate}}\loc{j}(M', M_y) \text{, where
    $t_k \in \LN\loc{j},$}
\end{equation}
as long as the rate $\lambda_k$ only depends the marking of places in
$P\loc{j}$. Thus, the Kronecker product decomposition framework can be
also employed if (limited) marking dependency is used in the model.

The synchronized transition $\TS$ can be treated as follows. First,
let us write the transition rate of $t_k \in \TS$ as
\begin{equation}
  \lambda_k = \lambda\loc{j_1}_k \lambda\loc{j_2}_k \cdots
  \lambda\loc{j_\ell}_k \label{eq:kronecker:lambdak}
\end{equation}
where $1 \le j_1 < j_2 < \cdots < j_\ell \le J$ are the indices of all
the local transitions sets $P\loc{j_i}$ in which $t_k$ appears. That
is, there is $\lambda\loc{j_i}_k$ corresponding to each local net the
transition $t_k$ synchronizes. If there is no marking dependency in
the model, we can set $\lambda\loc{j_1}_k = \lambda_k$ while all the
others are $1$. If marking dependency is used, each
$\lambda\loc{j_i}_k$ is allowed to depend on the restriction of the
marking to $P\loc{j_i}$, so
\begin{equation}
  \lambda_k(M) = \lambda\loc{j_1}_k(M\loc{j_1})\;
  \lambda\loc{j_2}_k(M\loc{j_2}) \;\cdots\;
  \lambda\loc{j_\ell}_k(M\loc{j_\ell}) \text.
\end{equation}
Now the matrices corresponding to synchronized transitions can be
expressed as
\begin{equation}
  \begin{cases}
    q_{t_k}\loc{j}[x, y] = \lambda_k\loc{j} p_{\text{immediate}}\loc{j}(M', M_y)
    & \text{if $t_k \in T\loc{j}$ and $\exists M' : M_x \xrightarrow{t_k} M'$,} \\
    q_{t_k}\loc{j}[x, y] = 0
    & \text{if $t_k \in T\loc{j}$ and $M_x \not\xrightarrow{t_k}$,} \\
    Q_{t_k}\loc{j} = I^{n_j \times n_j} & \text{if $t_k \notin T\loc{j}$.}
  \end{cases}\label{eq:kronecker:qt}
\end{equation}

Recalling multi-index notation from \vref{sec:intro:tensors}, we may
verify that
\begin{multline}
  q_O[\vec{x}, \vec{y}] = q_O[x\loc{1}, x\loc{2}, \ldots, x\loc{J},
  y\loc{1}, y\loc{2}, \ldots, y\loc{J}] = \\ \sum_{j = 1}^J
  q_L\loc{j}[x\loc{j}, y\loc{j}] + \sum_{t \in \TS} \prod_{j = 1}^J
q_t\loc{j}[x\loc{j}, y\loc{j}] \text. \label{eq:kronecker:qd-multiindex}
\end{multline}

\subsubsection{Contribution from local transitions}

The first term can be treated easily,
\begin{align}
   \sum_{j = 1}^J q_L\loc{j}[x\loc{j}, y\loc{j}] &= \sum_{j = 1}^J\,
  \sumprime_{\substack{M_x \xrightarrow{t_k} M' \\ t_k \in
  \TL\loc{j}}} \lambda_k p_{\text{immediate}}\loc{j}(M^{\prime(j)},
  M_y\loc{j}) \label{eq:kronecker:tl-doublesum}\\
  &= \sumprime_{\substack{M_x \xrightarrow{t_k} M' \\ t_k \in \TL}}
  \lambda_k p_{\text{immediate}}(M', M_y)
  \text, \label{eq:kronecker:tl-singlesum}
\end{align}
where $\TL = \bigcup_{j = 1}^J \TL\loc{j}$. Each local transition
appears exactly once in the double summation in
\cref{eq:kronecker:tl-doublesum}, because it belongs to exactly one
local net, so the simplification \labelcref{eq:kronecker:tl-singlesum}
is possble. A local transition cannot cause immediate state change in
any local net other than its own, since timed transitions cannot be
synchronized, therefore
$p_{\text{immediate}}\loc{j}(M^{\prime(j)}, M_y\loc{j}) =
p_{\text{immediate}}(M', M_y)$ holds for local transitions.

\subsubsection{Contribution from synchronized transitions}

One may notice that the factors
\begin{equation}
  \prod_{j = 1}^J q_t\loc{j}[x\loc{j}, y\loc{j}]
\end{equation}
on the right side of \vref{eq:kronecker:qd-multiindex} are nonzero
only if $t$ is enabled in $M_x$. Additionally, if $M_x$ is a tangible
marking and $M_x \xrightarrow{t_k} M'$, $M^{\prime (j)}$ must be
tangible for all $j$ for which $t_k \notin T\loc{j}$, because the
firing of $t_k$ causes no marking change in such local nets. This means that
$M_x\loc{j} \xrightarrow{t_k} M^{\prime (j)} = M_x\loc{j}$ for
$t_k \notin T\loc{j}$ and
$p_{\text{immediate}}\loc{j}(M^{\prime (j)}, M_y\loc{j}) = 1$ if
$M_x\loc{j} = M^{\prime (j)} = M_y\loc{j}$, otherwise $0$.

Thus, we can rewrite the second term as
\begin{align}
  &\sum_{t_k \in \TS} \prod_{j = 1}^J q_{t_k}\loc{j}[x\loc{j}, y\loc{j}] =
    \sumprime_{\substack{M_x \xrightarrow{t_k} M' \\ t_k \in \TS}} \,
  \prod_{j = 1}^J q_t\loc{j}[x\loc{j}, y\loc{j}] \\
  &\qquad=\sumprime_{\substack{M_x \xrightarrow{t_k} M' \\ t_k \in
  \TS}} \, \prod_{t_k \in T\loc{j}} \lambda_k\loc{j}
  p_{\text{immediate}}\loc{j}(M^{\prime (j)}, M_y\loc{j}) \, \prod_{t_k
  \notin T\loc{j}} \delta_{x\loc{j}, y\loc{j}} \\
  &\qquad=\sumprime_{\substack{M_x \xrightarrow{t_k} M' \\ t_k \in
  \TS}} \, \prod_{t_k \in T\loc{j}} \lambda_k\loc{j}
  p_{\text{immediate}}\loc{j}(M^{\prime (j)}, M_y\loc{j}) \, \prod_{t_k
  \notin T\loc{j}} p_{\text{immediate}}\loc{j}(M^{\prime (j)},
  M_y\loc{j}) \text, \\
  \intertext{where we can use the factorization of $\lambda_k$
  from \vref{eq:kronecker:lambdak}
  and \cref{eq:kronecker:pimmediate-j}
  from \vref{obs:kronecker:pimmediate-j} to conclude}
  &\qquad=\sumprime_{\substack{M_x \xrightarrow{t_k} M' \\ t_k \in
  \TS}} \, \prod_{t_k \in T\loc{j}} \lambda_k\loc{j} \,\, \prod_{j = 1}^J
  p_{\text{immediate}}\loc{j} (M^{\prime (j)}, M_y\loc{j}) \\
  &\qquad= \sumprime_{\substack{M_x \xrightarrow{t_k} M' \\ t_k \in
  \TS}} \lambda_k p_{\text{immediate}}(M', M_y)
  \text, \label{eq:kronecker:qo-ts-result}
\end{align}
where $\delta_{x,y}$ is the $[x, y]$\thinspace th element of the
identity matrix.

\begin{thm}
  The decomposition \labelvref{eq:kronecker:simple-local-global},
  where the matrices $Q_L\loc{j}$ and $Q_t\loc{j}$ are defined by
  \cref{eq:kronecker:ql,eq:kronecker:qt}, is equivalent to the
  explicit generator matrix from
  \fullref{eq:intro:qd-element,eq:intro:qd-qo}.
\end{thm}

\begin{proof}
  Because $T = \TL \uplus \TS$ is a disjoint partition of the
  transitions, the addition of \vref{eq:kronecker:tl-singlesum} and
  \vref{eq:kronecker:qo-ts-result} yields
  \begin{equation}
    \sumprime_{\substack{M_x \xrightarrow{t_k} M' \\ t_k \in
        \TL \uplus \TS}} \lambda_k p_{\text{immediate}}(M', M_y) =
    \sumprime_{M_x \xrightarrow{t_k} M'} \lambda_k
    p_{\text{immediate}}(M', M_y) \text.
  \end{equation}
  This is a summation over all enabled timed transitions, as required
  by \cref{eq:intro:qd-element}.
\end{proof}

\fbreak

\begin{dfn}
  The \emph{potential state space} of an \sgspn is the Descartes
  product of the tangible reachability sets of its local nets,
  \begin{equation}
    \PSS = \TRS\loc{1} \times \TRS\loc{2} \times \cdots \times
    \TRS\loc{J} \subseteq \TRS \text.
  \end{equation}
\end{dfn}

Now we turn to the case when $\PSS \ne \TRS$, that is, when the
tangible reachability space of the \sgspn\ is \emph{not} in Descartes
product form. Approaches developed for this kind of systems may be
divided into the following three broad categories
\citet{DBLP:conf/ipps/BenoitPS03}:
\begin{enumerate}
\item Local nets may be aggregated (joined into larger local nets)
  until the resulting \sgspn's $\TRS$ becomes a Descartes
  product. Unfortunately, is is possible that the resulting aggregated
  \sgspn\ has only a single local net, therefore the approach gives no
  benefit over explicit treatment of the \gspn of the resulting local
  nets have state spaces $\TRS\loc{j}$ so large that computation
  become infeasible.
\item The infinitesimal generator 
\end{enumerate}

\section{Efficient vector--Kronecker product multiplication}

Both steady-state and transient analysis of \ctmc s, such as
\vref{eq:intro:diffeq-initial,eq:intro:steadystate}, often require the
multiplication of a probability distribution vector $\vec{\uppi}$ with
the infinitesimal generator matrix $Q$.

Because the vector--matrix product is associative,
\begin{align}
  \vec{\pi} Q &= \vec{\uppi} \, \mleft( \bigoplus_{j = 1}^J Q_L\loc{j}
                + \sum_{t \in \TS} \bigotimes_{j = 1}^J Q_t\loc{j}
                \mright) \\
              &= \sum_{j = 1}^J \vec{\uppi} \, I^{n_1 \times n_1} \otimes
                \cdots \otimes Q_L\loc{j} \otimes \cdots \otimes
                I^{n_J \times n_J} + \sum_{t \in \TS} \vec{\uppi}
                \bigotimes_{j = 1}^J Q_t\loc{j} \text,
\end{align}
thus only vector--Kronecker product multiplications and vector
additions are requred for probability vector--$Q$ matrix
multiplications.

An efficient vector--Kronecker product multiplication algorithm must
evaluate the product
\begin{equation}
  \vec{v} \bigotimes_{j = 1}^J A\loc{j}
  \text, \label{eq:kronecker:vector-kronecker-product}
\end{equation}
where each $A\loc{j}$ is of size $m_j \times n_j$, without explicitly
evaluating and storing the
$m \times n = \prod_{j = 1}^J m_j \times \prod_{j = 1}^J n_j$ matrix
$A = \bigotimes_{j = 1}^J A\loc{j}$. The simplest of these algorithm
is the \emph{extended shuffle algorithm} by
\citet{DBLP:conf/ipps/BenoitPS03}. The algorithm is an extension of
another, rather, \emph{extended} refers to the fact that the vector
$\vec{v}$ is stored in an \emph{extended} or \emph{dense} form by
explicitly storing all the zero elements in addition to the
nonzeroes. In contrast, the \emph{partially reduced} and \emph{fully
  reduced shuffle algoritms}
\citep{DBLP:conf/ipps/BenoitPS03,DBLP:journals/fgcs/BenoitPS06} can
handle multiplication of vectors is sparse format and thus may be
employed in potential state-space Kronecker methods even when the size
of $\PSS$ prevents explicit dense storage.

The shuffle algoritms are based on a rewriting of
\cref{eq:kronecker:vector-kronecker-product}
\begin{align}
  \vec{v} \bigotimes_{j = 1}^J A\loc{j}
  &= \vec{v} \prod_{j = 1}^J
    I^{n_1 \times n_1} \otimes \cdots \otimes
    I^{n_{j - 1} \times n_{j - 1}} \otimes A\loc{j} \otimes I^{m_{j + 1}
    \times m_{j + 1}} \otimes \cdots \otimes I^{m_J \otimes m_J} \\
  &= \vec{v} \prod_{j = 1}^J I^{\textit{nleft}_j \times
    \textit{nleft}_j} \otimes A\loc{j} \otimes I^{\textit{mright}_j
    \times \textit{mright}_j} \text,
\end{align}
where
\begin{equation}
  \begin{aligned}
    \textit{nleft}_j &= \textstyle \prod_{k = 1}^{j - 1} n_k \text,
    & \textit{nright}_j &= \textstyle \prod_{k = j + 1}^{J} m_k \text. \\
  \end{aligned}
\end{equation}

\Vref{alg:kronecker:shuffle} shows the pseudocode of the extended
shuffle algorithm.

\begin{algorithm}
  \caption{Extended shuffle
    \citep[Algorithm~2.1]{DBLP:conf/ipps/BenoitPS03}}
  \label{alg:kronecker:shuffle}
  \begin{algorithmic}[1]
    \Function{Shuffle}{$\vec{v}, A\loc{1}, A\loc{2}, \ldots,
      A\loc{J}$}
    \State Allocate buffers $\vec{x}$ and $\vec{y}$ of length $\max_{0
    \le j \le J} \prod_{k = 1}^j n_k \prod_{k = j}^J m_k$ each
    \State Copy $\vec{v}$ to the first $m$ elements of $\vec{x}$
    \State $\textit{nleft} = 1$, $\textit{mright} = \prod_{j = 2}^J m_j$
    \For{$j \gets 1$ \To $J$}
    \Comment{For each Kronecker factor}
    \For{$l \gets 0$ \To $\textit{nleft} - 1$}
    \Comment{For each ``slice'' of $\vec{x}$}
    \For{$r \gets 0$ \To $\textit{mright} - 1$}
    \Comment{For each ``subslice''}
    \State Allocate a vector $\vec{z}_{\textrm{in}}$ of length $m_j$
    \State $\textit{index} \gets l \cdot \textit{mright} \cdot m_j + r$
    \Comment{Find the input subslice}
    \For{$k \gets 1$ \To $m_j - 1$}
    \Comment{Copy elements to be multiplied to $\vec{z}_{\textrm{in}}$}
    \State $z_{\textrm{in}}[k] \gets x[\textit{index}]$
    \State $\textit{index} \gets \textit{index} + \textit{mright}$
    \EndFor
    \State $\vec{z}_{\textrm{out}} \gets \vec{z}_{\textrm{in}}
    A\loc{j}$
    \Comment{Perform the multiplication}
    \State $\textit{index} \gets l \cdot \textit{mright} \cdot n_j +
    r$
    \Comment{Find the output subslice}
    \For{$k \gets 1$ \To $n_j - 1$}
    \Comment{Copy results from $\vec{z}_{\textrm{out}}$ to $\vec{y}$}
    \State $y[\textit{index}] \gets z_{\textrm{out}}[k]$
    \State $\textit{index} \gets \textit{index} + \textit{mright}$
    \EndFor
    \EndFor
    \EndFor
    \State Swap $\vec{x}$ and $\vec{y}$
    \State $\textit{nleft} \gets \textit{nleft} \cdot n_j$
    \Comment{Update $\textit{nleft}$ and $\textit{mright}$}
    \State \algorithmicif\ $j < J$ \algorithmicthen\
    $\textit{mright} \gets \textit{mright} / m_{j + 1}$
    \EndFor
    \State Copy the first $n$ elements of $\vec{x}$ into a new vector
    $\vec{w}$
    \State \Return $\vec{w} = \bigotimes_{j = 1}^J A\loc{j}$
    \EndFunction
  \end{algorithmic}
\end{algorithm}

%%% Local Variables:
%%% mode: latex
%%% TeX-master: "markov"
%%% End:


\backmatter

\printbibliography

\end{document}